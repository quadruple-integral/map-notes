\documentclass{article}
\usepackage{geometry}
\usepackage{amsmath}
\usepackage{amsfonts}
\usepackage{amssymb}
\usepackage{amsthm}
\usepackage{parskip}
\usepackage{multicol}
\usepackage{xcolor}
\usepackage{fancyhdr}
\usepackage{physics}
\usepackage{graphicx} % Required for inserting images
\usepackage{hyperref}
\usepackage{enumitem}

% margin settings
\geometry{
    a4paper,
    left=7mm,
    right=7mm,
    top=2cm,
    bottom=7mm
}

% proof environments
\newtheorem{definition}{Definition}[section]
\newtheorem{theorem}{Theorem}[section]
\newtheorem{corollary}{Corollary}[theorem]
\newtheorem{lemma}[theorem]{Lemma}
\newtheorem*{remark}{Remark} % unnumbered remarks

% header and footer
\pagestyle{fancy}
\fancyfoot{} % removes footer
\fancyhf{}
\renewcommand{\headrulewidth}{0.5pt}
\fancyhead[L]{Electromagnetism and relativity}
\fancyhead[R]{\thepage}

\begin{document}

\begin{multicols*}{3}
% starred environment ensures text remains in same column
\noindent

\subsubsection*{Vector products}
$$\boldsymbol{a}\cdot\boldsymbol{b}=ab\cos\theta$$
$$\boldsymbol{a}\times
\boldsymbol{b}=ab\sin\theta\hat{\boldsymbol{n}}$$
$$\boldsymbol{a}\times\boldsymbol{b}
=-\boldsymbol{b}\times\boldsymbol{a}$$
$$\boldsymbol{a}\times(\boldsymbol{b}\times\boldsymbol{c})
=\boldsymbol{b}(\boldsymbol{a}\cdot\boldsymbol{c})
-\boldsymbol{c}(\boldsymbol{a}\cdot\boldsymbol{b})$$

\subsubsection*{Suffix notation}
\begin{enumerate}
    \item A suffix that appears \underline{twice}
    implies a summation.
    \item Any suffix \underline{cannot appear} \\ 
    \textbf{more than twice} in any term.
\end{enumerate}
We define the \textbf{Kronecker delta} as:
$$\delta_{ij}=\left\{\begin{array}{ll}
    1 & i=j \\
    0 & i\neq j
\end{array}\right.$$
and the \textbf{Levi-Civita} as:
$$\epsilon_{ijk}=\left\{
\begin{array}{lll}
    +1 & 123,312,231 \\
    -1 & 132,213,321 \\
    0 & \text{repeat indices.}
\end{array}\right.$$
Consequently:
\begin{align*}
    &\epsilon_{ijk}=\epsilon_{kij}=\epsilon_{jki} \\
    &=-\epsilon_{ijk}=-\epsilon_{ijk}=-\epsilon_{ijk}
\end{align*}
and we have the following identities:
$$\boldsymbol{a}
=\sum_{i=1}^{3}a_i\boldsymbol{e}_i
=a_i\boldsymbol{e}_i$$
$$\delta_{ii}=3$$
$$[\dots]_j\delta_{jk}
=[\dots]_k$$
$$\boldsymbol{e}_i\cdot
\boldsymbol{e}_j=\delta_{ij}$$
$$\boldsymbol{e}_i\times
\boldsymbol{e}_j=
\epsilon_{ijk}\boldsymbol{e}_k$$
$$\boldsymbol{a}\times\boldsymbol{b}
=\epsilon_{ijk}a_j b_k\boldsymbol{e}_i$$
$$\boldsymbol{a}\cdot
(\boldsymbol{b}\times\boldsymbol{c})
=\epsilon_{ijk}a_i b_j c_k$$
$$\epsilon_{ijk}\epsilon_{klm}
=\delta_{il}\delta_{jm}
-\delta_{im}\delta_{jl}$$
$$\epsilon_{ijk}\epsilon_{ijl}
=2\delta_{kl}
\hspace{0.05in}\text{and}\hspace{0.05in}
\epsilon_{ijk}\epsilon_{ijk}=6.$$

\subsubsection*{Transformations}
$$\boldsymbol{e}'_i
=\ell_{ij}\boldsymbol{e}_j$$
$$\ell_{ik}\ell_{jk}
=\ell_{ki}\ell_{kj}=\delta_{ij}$$
$$L^T L=LL^T=I
\hspace{0.05in}\text{where}\hspace{0.05in}
(L)_{ij}=\ell_{ij}$$

\subsubsection*{Tensors}
A rank $3$ tensor is defined as:
$$T'_{ijk}=\ell_{ip}\ell_{jq}\ell_{kr}T_{pqr}$$
which relates frame $S$ in $\{\boldsymbol{e}_i\}$ to
frame $S'$ in $\{\boldsymbol{e}'_i\}$ with
rule $\boldsymbol{e}'_i=\ell_{ij}\boldsymbol{e}_j$, etc.

Properties of tensors:
\begin{enumerate}
    \item The \underline{addition} of two rank
    $n$ tensors is also a rank $n$ tensor.
    
    \item The \underline{multiplication} of a
    rank $m$ tensor with a rank $n$ tensor yields
    a rank $m+n$ tensor.

    \item If $T_{ijk\dots s}$ is a rank $m$ tensor
    then $T_{\textcolor{red}{ii}k\dots s}$ is a rank $m-2$ tensor.

    \item If $T_{ij}$ is a tensor then
    $T_{ji}$ is also a tensor.
\end{enumerate}

\subsubsection*{Symmetric tensors}
$T_{ij}$ is a \underline{symmetric} tensor
when $T_{ij}=T_{ji}$ in frame $S$.
Then $T'_{ij}=T'_{ji}$ in frame $S'$.

Similarly $T_{ij}$ is an \underline{anti-symmetric}
tensor if $T_{ij}=-T_{ji}$ and
$T'_{ij}=-T'_{ji}$.

Finally \textcolor{red}{any tensor} can be written as
a sum of symmetric and anti-symmetric parts:
$$T_{ij}=\frac{1}{2}(T_{ij}+T_{ji})
+\frac{1}{2}(T_{ij}-T_{ji}).$$

\subsubsection*{Quotient theorem}
Consider $9$ entities $T_{ij}$ in frame $S$
and $T'_{ij}$ in frame $S'$. Let $b_i=T_{ij}a_j$
where $a_j$ is a vector.
If $b_i$ \underline{always} transforms as a vector
then $T_{ij}$ is a rank $2$ tensor.

Generalising, let $R_{ijk\dots r}$ be a rank $m$
tensor and $T_{ijk\dots s}$ a set of $3^n$ numbers
where $n>m$. If $T_{ijk\dots s}R_{ijk\dots r}$
is a rank $n-m$ tensor then $T_{ijk\dots s}$
is a rank $n$ tensor.

\subsubsection*{Matrices}
We define a $m\times n$ matrix $A$ as $(A)_{ij}=a_{ij}$
where $i=1,\dots,m$ and $j=1,\dots,n$.
\begin{itemize}
    \item $\Trace A=a_{ii}$
    \item $(A^T)_{ij}=a_{ji}$
    \item $(AB)^T=B^T A^T$
    \item $(I)_{ij}=\delta_{ij}$
\end{itemize}
The determinant of a $3\times3$ matrix $A$ is:
\begin{align*}
    \det A
    &=\left|
        \begin{array}{lll}
            a_{11} & a_{12} & a_{13} \\
            a_{21} & a_{22} & a_{23} \\
            a_{31} & a_{32} & a_{33}
        \end{array}
    \right| \\
    &=\epsilon_{lmn}a_{1l}a_{2m}a_{3n} \\
    &=\epsilon_{lmn}a_{l1}a_{m2}a_{n3}
\end{align*}
$$\epsilon_{ijk}\det A=
\epsilon_{lmn}a_{il}a_{jm}a_{kn}.$$

Furthermore:
$$\epsilon_{lmn}\det A=
\epsilon_{ijk}a_{il}a_{jm}a_{kn}$$
$$\det A=\frac{1}{3!}\epsilon_{ijk}
\epsilon_{lmn}a_{il}a_{jm}a_{kn}.$$

Properties of determinants:
\begin{enumerate}
    \item Adding rows to each other \\
    \underline{does not} change the determinant.
    
    \item Interchanging two rows \\
    \underline{changes determinant signs}.

    \item $\det A=\det A^T$
    
    \item $\det(AB)=\det A\cdot\det B$
\end{enumerate}
These also apply to columns. Finally:
$$\epsilon_{ijk}\epsilon_{lmn}\det A
=\left|
    \begin{array}{lll}
        a_{il} & a_{im} & a_{in} \\
        a_{jl} & a_{jm} & a_{jn} \\
        a_{kl} & a_{km} & a_{kn}
    \end{array}
\right|$$
and setting $A=I$ yields:
$$\epsilon_{ijk}\epsilon_{lmn}
=\left|
    \begin{array}{lll}
        \delta_{il} & \delta_{im} & \delta_{in} \\
        \delta_{jl} & \delta_{jm} & \delta_{jn} \\
        \delta_{kl} & \delta_{km} & \delta_{kn}
    \end{array}
\right|.$$

\subsubsection*{Linear equations}
Consider $\boldsymbol{y}=A\boldsymbol{x}$.
$\therefore x_i=A^{-1}_{ij}y_i$
$$A^{-1}_{ij}=\frac{1}{2}\frac{1}{\det A}
\epsilon_{imn}\epsilon_{jpq}a_{pm}a_{qn}$$

\subsubsection*{Orthogonal matrices}

\subsubsection*{Pseudotensors}

\subsubsection*{Invariant tensors}

\end{multicols*}

\end{document}