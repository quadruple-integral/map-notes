\documentclass{article}
\usepackage{amsmath}
\usepackage{amsfonts}
\usepackage{amssymb}
\usepackage{amsthm}
\usepackage{parskip}
\usepackage{multicol}
\usepackage{xcolor}
\usepackage{fancyhdr}
\usepackage{physics}
\usepackage{graphicx} % Required for inserting images
\usepackage{hyperref}
\newcommand{\matr}[1]{\mathbf{#1}}

\begin{document}

\pagestyle{fancy}
\fancyhead{}
\fancyhead[L]{Electromagnetism and Relativity}
\fancyhead[R]{Winter 2023 - Summer 2024}

\tableofcontents

\newpage

\section{Suffix notation}

\section{Cartesian tensors}

\subsection{True tensors}
tensor algebra

\subsubsection{Rank 2 quotient theorem}
The \textbf{quotient theorem} is as an alternative definition for tensors. In the context of \underline{rank 2} tensors it states that if $b_i$ always transforms as a \underline{vector} in
$$b_i=T_{ij}a_j$$
and that $a_j$ is also a vector then $T_{ij}$ is a rank 2 tensor.
\begin{proof}
    We egregiously define entity $T_{ij}$ in frame $S$ and $T'_{ij}$ in frame $S'$.

    The usual transformation laws apply, namely $\boldsymbol{e}'_i=\ell_{ij}\boldsymbol{e}_j$. By definition:
    \begin{align*}
        b'_i
        &=T'_{ij}a'_j \\
        &=T'_{ij}\ell_{jk}a_k
    \end{align*}
    Also directly from transformation laws:
    \begin{align*}
        b'_i
        &=\ell_{ij}b_j \\
        &=\ell_{ij}T_{jk}a_k
    \end{align*}
    $$\therefore\bigl(T'_{ij}\ell_{jk}-\ell_{ij}T_{jk}\bigl)a_k=0$$
    Since $a_k$ are constants of our vector it must then be that:
    $$T'_{ij}\ell_{jk}=\ell_{ij}T_{jk}$$
    $$\therefore T'_{ij}\ell_{jk}\ell_{mk}=\ell_{ij}\ell_{mk}T_{jk}$$
    Where here we aim to eliminate the first two $\ell$s. Finally:
    $$T'_{im}=\ell_{ij}\ell_{mk}T_{jk}$$
\end{proof}

\subsubsection{General quotient theorem}
Let $R_{ij\dots r}$ be a rank $m$ tensor, and $T_{ij\dots s}$ be a set of $3^n$ numbers where $n>m$.

If $R_{ij\dots r}T_{ij\dots s}$ is a rank $n-m$ tensor then $T_{ij\dots s}$ is a rank $n$ tensor.

symmetric and anti symmetric tensors

\subsection{Matrices as tensors}

\subsection{Pseudotensors}
Firstly note that $\det L=+1$ for \underline{rotations}, and $\det L=-1$ for \underline{reflections} and \underline{inversions}. Recall the transformation law $\boldsymbol{e}'_i=\ell_{ij}\boldsymbol{e}_j$.

A \underline{second} rank \textbf{pseudotensor} is defined:
$$T'_{ij}=(\det L)\ell_{ip}\ell_{jq}T_{pq}.$$
Furthermore a \underline{rank 1} pseudotensor is a \textbf{pseudovector} and is defined as:
$$T'_{i}=(\det L)\ell_{ip}T_{p}.$$
Finally a \textbf{pseudoscalar} is a \underline{rank 0} pseudotensor:
$$a'=(\det L)\cdot a,$$
and changes sign under transformation.

\subsection{Invariant tensors}

\subsection{Rotation tensors}

\subsection{Reflections, inversions and projections}
active and passive transformations

maybe merge with rotations?

\subsection{Inertia tensors}

\section{Taylor expansions}

\newpage

\section{Vector calculus}

\subsection{Vector operators}

\subsubsection{Gradient}

\subsubsection{Divergence}

\subsubsection{Curl}

chain rules, important identities

\subsection{Integrals theorems}

\subsubsection{Line, volume and surface integrals}

\subsubsection{Divergence theorem}

\subsubsection{Stokes's theorem}
Consider surface $S$ \underline{enclosed}
by line $C$. We then have that:
$$\int_S\boldsymbol{\nabla}\times\boldsymbol{E}\cdot
\dd\boldsymbol{S}=
\oint_C\boldsymbol{E}\cdot\dd\boldsymbol{r}.$$

\newpage

\section{Curvilinear coordinates}

\subsection{Orthogonal curvilinear coordinates}

\subsubsection{Scale factors and basis vectors}
Consider change of variables:
$$(x_1,x_2,x_3)\leftrightarrow(u_1,u_2,u_3)$$
where $u_i$ are our curvilinear coordinates, and
$$u_i=u_i(x_1,x_2,x_3)$$
$$x_i=x_i(u_1,u_2,u_3).$$
Then we define:
\begin{align*}
    \dd\boldsymbol{r}_i
    &=\frac{\partial\boldsymbol{r}}{\partial u_i}\dd u_i \\
    &=h_i\boldsymbol{e}_i \dd u_i
\end{align*}
where $h_i=|\displaystyle\frac{\partial\boldsymbol{r}}{\partial u_i}|$ is our \textbf{scale factor} and 
$$\boldsymbol{e}_i=\frac{1}{h_i}\frac{\partial\boldsymbol{r}}{\partial u_i}$$
is our \textbf{basis vector} of \underline{unit length} for a specific set of curvilinear coordinates.

Now if the basis vectors satisfy
$$\boldsymbol{e}_i\cdot\boldsymbol{e}_j=\delta_{ij}$$
we have an \underline{orthogonal} set of curvilinear coordinates.

\subsubsection{Cylindrical coordinates}
We define cylindrical coordinates as
$$(u_1,u_2,u_3)=(\rho,\phi,z)$$
and with the following relation to Cartesian coordinates:
$$\boldsymbol{r}=\rho\cos\phi\boldsymbol{e}_x
+\rho\sin\phi\boldsymbol{e}_y+z\boldsymbol{e}_z.$$
Furthermore:
$$h_{\rho}=1\hspace{0.1in}\text{and}\hspace{0.1in}\boldsymbol{e}_{\rho}=\cos\phi\boldsymbol{e}_{x}
+\sin\phi\boldsymbol{e}_{y}$$
$$h_{\phi}=\rho\hspace{0.1in}\text{and}\hspace{0.1in}
\boldsymbol{e}_{\phi}=-\sin\phi\boldsymbol{e}_x+\cos\phi\boldsymbol{e}_y$$
$$h_{z}=1\hspace{0.1in}\text{and}\hspace{0.1in}\boldsymbol{e}_z=\boldsymbol{e}_z.$$
Here $\phi$ is the \underline{anticlockwise} rotation of the $xy$-plane.

\newpage

\subsubsection{Spherical coordinates}
We define the spherical coordinates as
$$(u_1,u_2,u_3)=(r,\theta,\phi)$$
$$\boldsymbol{r}=r\sin\theta\cos\phi\boldsymbol{e}_x
+r\sin\theta\sin\phi\boldsymbol{e}_y
+r\cos\theta\boldsymbol{e}_z$$
where $\boldsymbol{e}_x$, $\boldsymbol{e}_y$ and $\boldsymbol{e}_z$
represent the Cartesian unit vectors.

Now $\phi\in[0,2\pi]$ is the \underline{rotation} angle in $xy$-plane,
and $\theta\in[0,\pi]$ in $z$-plane. We also have that:
$$h_{r}=1\hspace{0.1in}\text{and}\hspace{0.1in}\boldsymbol{e}_{r}
=\sin\theta\cos\phi\boldsymbol{e}_x
+\sin\theta\sin\phi\boldsymbol{e}_y+\cos\theta\boldsymbol{e}_z$$
$$h_{\theta}=r\hspace{0.1in}\text{and}\hspace{0.1in}
\boldsymbol{e}_{\theta}
=\cos\theta\cos\phi\boldsymbol{e}_x+\cos\theta\sin\phi\boldsymbol{e}_y
-\sin\theta\boldsymbol{e}_z$$
$$h_{\phi}=r\sin\theta\hspace{0.1in}\text{and}\hspace{0.1in}\boldsymbol{e}_{\phi}
=-\sin\phi\boldsymbol{e}_x+\cos\phi\boldsymbol{e}_y.$$

\newpage

\subsection{Length, area and volume}

\subsubsection{Vector and arc length}
Firstly the \textbf{vector length} due to infinitesimal change in all directions is
$$\dd\boldsymbol{r}=\sum_{i=1}^{3}h_i\dd u_i\boldsymbol{e}_i.$$
It is important to note that summation notation does not work here.

Now the \textbf{arc length} of $\dd\boldsymbol{r}$ is:
\begin{align*}
    \dd s
    &=|\dd\boldsymbol{r}| \\
    &=\sqrt{\dd\boldsymbol{r}\cdot\dd\boldsymbol{r}}
\end{align*}
and we define the \textbf{metric tensor} as
\begin{align*}
    g_{ij}
    &=\frac{\partial x_k}{\partial u_i}\frac{\partial x_k}{\partial u_j} \\
    &=\frac{\partial\boldsymbol{r}}{\partial u_i}
    \cdot\frac{\partial\boldsymbol{r}}{\partial u_j}.
\end{align*}
Since $\dd\boldsymbol{r}=\dd x_k$ we then the following relation:
$$(\dd s)^2=g_{ij}\dd u_i\dd u_j.$$

\newpage

\subsubsection{Vector area}

\subsubsection{Volume}
The volume of the infinitesimal parallelepiped defined by
$\dd\boldsymbol{r}_1$, $\dd\boldsymbol{r}_2$ and $\dd\boldsymbol{r}_3$ is:
\begin{align*}
    \dd V
    &=|(\dd\boldsymbol{r}_1\times\dd\boldsymbol{r}_2)
    \cdot\dd\boldsymbol{r}_3| \\
    &=h_1\hspace{0.03in}h_2\hspace{0.03in}h_3\hspace{0.03in}
    \dd u_1\hspace{0.03in}\dd u_2\hspace{0.03in}\dd u_3\hspace{0.03in}
    |(\boldsymbol{e}_1\times\boldsymbol{e}_2)
    \cdot\boldsymbol{e}_3| \\
    &=\sqrt{g}\hspace{0.03in}
    \dd u_1\hspace{0.03in}\dd u_2\hspace{0.03in}\dd u_3\hspace{0.03in}
\end{align*}
where $g$ is the \underline{determinant} of the metric tensor.
\newpage

\section{Electrostatics}

\subsection{Dirac delta function}
The one dimensional \textbf{Dirac delta} is defined:
$$\delta(x) =
    \left\{
	\begin{array}{ll}
		\infty  & \mbox{} x=0 \\
		0 & \mbox{} x\neq0,
	\end{array}
    \right.$$
and can be thought of as infinitely sharp at $x=0$ and zero elsewhere. 

It satisfies some useful properties:
\begin{itemize}
    \item $\delta(x-a) =
    \displaystyle\lim_{\sigma\rightarrow0}\left[\frac{1}{|\sigma|\sqrt{\pi}}
    \exp\left(-\frac{(x-a)^2}{\sigma^2}\right)\right]$

    i.e. an infinitely sharp Gaussian. (\underline{generalised functions})

    \item \textbf{Sift property}
    $$\int_{\mathbb{R}}f(x)\delta(x-a)dx=f(a)$$

    \item Let $x_i$ be the solutions to $g(x_i)=0$. Then:
    $$\int_{\mathbb{R}}f(x)\delta[g(x)]dx
    =\sum_i\frac{f(x_i)}{|g'(x_i)|}$$
\end{itemize}

Now we consider the \textbf{3D Dirac delta}, which is defined as follows:
$$\delta(\boldsymbol{r}-\boldsymbol{r}_0)=\delta(x-x_0)\delta(y-y_0)\delta(z-z_0)$$
given Cartesian coordinates $(x_1, x_2, x_3)$. It also satisfies the \textbf{sift} property:
$$\int_{\mathbb{R}^3}f(\boldsymbol{r})\delta(\boldsymbol{r}-\boldsymbol{r}_0)
=f(\boldsymbol{r}_0).$$
The three dimensional Dirac delta defined in a orthogonal \underline{curvilinear} coordinate system $(u_1, u_2, u_3)$ is as follows:
$$\delta(\boldsymbol{r}-\boldsymbol{a})=\frac{1}{h_1 h_2 h_3}
\delta(u_1-a_1)\delta(u_2-a_2)\delta(u_3-a_3)$$
for $h_1, h_2$ and $h_3$ are the \underline{scale factors}.

\newpage

\subsection{Coulomb's law}
Consider the force on charge $q$ at $\boldsymbol{r}$ due to charge $q_1$
at $\boldsymbol{r}_1$:
$$\boldsymbol{F}_1(\boldsymbol{r})=\frac{1}{4\pi\epsilon_0}
\frac{qq_1(\boldsymbol{r}-\boldsymbol{r}_1)}{|\boldsymbol{r}-\boldsymbol{r}_1|^3},$$
for here $\epsilon_0=8.85\times10^{-12}C^2N^{-1}m^{-2}$ in vacuum. 

Physically, like charges ($qq_1>0$) repel while opposite charges ($qq_1<0$) attract.

We then define an \textbf{electric field} as the force on a small positive test charge:
$$\boldsymbol{E}(\boldsymbol{r})=
\lim_{q\rightarrow 0}\left(\frac{1}{q}\boldsymbol{F}(\boldsymbol{r})\right).$$
The force on a charge $q$ at $\boldsymbol{r}$ from the origin in this electric field is:
$$\boldsymbol{F}(\boldsymbol{r})=q\boldsymbol{E}(\boldsymbol{r}).$$
A negative point charge is a sink whereas a positive point charge is a source.

Consider a collection of charges $q_i$ at position $\boldsymbol{r}_i$. The \textbf{principle of superposition} tells us that:
$$\boldsymbol{E}(\boldsymbol{r})=\frac{1}{4\pi\epsilon_0}
\sum_i \left(\frac{q_i (\boldsymbol{r}-\boldsymbol{r}_i)}
{|\boldsymbol{r}-\boldsymbol{r}_i|^3}\right).$$
Now consider a continuous charged object with volume $V$
and \textbf{charge density} $\rho(\boldsymbol{r}')$.
It generates the following electric field:
$$\boldsymbol{E}(\boldsymbol{r})=\frac{1}{4\pi\epsilon_0}
\int_V \rho(\boldsymbol{r}')\frac{\boldsymbol{r}-\boldsymbol{r}'}
{|\boldsymbol{r}-\boldsymbol{r}'|^3}dV'.$$
Returning to the electric field generated by a point charge $q_1$
at position $\boldsymbol{r}_1$:
$$\boldsymbol{E}(\boldsymbol{r})=\frac{q_1}{4\pi\epsilon_0}
\frac{\boldsymbol{r}-\boldsymbol{r}_1}{|\boldsymbol{r}-\boldsymbol{r}_1|^3},$$
this is a \textbf{conservative field}, and we may write it as:
$$\boldsymbol{E}(\boldsymbol{r})=-\boldsymbol{\nabla}\phi(\boldsymbol{r}),$$
where:
$$\phi(\boldsymbol{r})=\frac{q_1}{4\pi\epsilon_0}\frac{1}{|\boldsymbol{r}-\boldsymbol{r}_1|}.$$
Conservative fields have zero curl, and their line integrals are path independent. This namely applies to finding work done.

\newpage

\subsection{Electrostatic Maxwell's equations}

Firstly for a continuous charge distribution:
\begin{align*}
    \boldsymbol{E}(\boldsymbol{r})
    &=-\frac{1}{4\pi\epsilon_0}
    \int_V\rho(\boldsymbol{r}')
    \frac{\boldsymbol{r}-\boldsymbol{r}'}{|\boldsymbol{r}-\boldsymbol{r}'|^3}
    \dd V' \\
    &=-\boldsymbol{\nabla}
    \left(\frac{1}{4\pi\epsilon_0}
    \int_V\rho(\boldsymbol{r}')
    \frac{1}{|\boldsymbol{r}-\boldsymbol{r}'|}
    \dd V'\right)
\end{align*}
and therefore $\boldsymbol{\nabla}\times
\boldsymbol{E}=\boldsymbol{0}$ for \underline{static electric fields}.

Hence electrostatic fields are \underline{conversative fields}:
$$\int_{C_1}\boldsymbol{E}\cdot\dd\boldsymbol{r}=
\int_{C_2}\boldsymbol{E}\cdot\dd\boldsymbol{r}$$
and that
$$-\int_{a}^{b}\boldsymbol{E}\cdot\dd\boldsymbol{r}
=\phi(b)-\phi(a)$$
where $\boldsymbol{E}(\boldsymbol{r})
=-\boldsymbol{\nabla}\phi(\boldsymbol{r})$. Therefore:
$$\phi(\boldsymbol{r})
=\frac{1}{4\pi\epsilon_0}
\int_V\rho(\boldsymbol{r}')
\frac{1}{|\boldsymbol{r}-\boldsymbol{r}'|}
\dd V'.$$


\newpage

\subsection{Electric dipoles}
\textbf{Dipoles} consist of \underline{two} \underline{equal} and \textbf{opposite point charges} that are $\boldsymbol{d}$ apart. 

An \textbf{ideal dipole} is defined as when the following \textbf{dipole limit} is \underline{finite} and \underline{constant}:
$$\boldsymbol{p}=\lim_{\begin{smallmatrix} q\to\infty & \\ \boldsymbol{d}\to 0 \end{smallmatrix}} q\boldsymbol{d}.$$
A \textbf{dipole moment} is simply $\boldsymbol{p}=q\boldsymbol{d}$. The \textbf{dipole potential} at $\boldsymbol{r}_0$ is:
\begin{align*}
    \phi(\boldsymbol{r})
    &= \frac{q}{4\pi\epsilon_0}\left(\frac{1}{|\boldsymbol{r}-\boldsymbol{r}_0-\boldsymbol{d}|}-\frac{1}{|\boldsymbol{r}-\boldsymbol{r}_0|}\right)\\
    &=\frac{1}{4\pi\epsilon_0}\frac{\boldsymbol{p}\cdot(\boldsymbol{r}-\boldsymbol{r}_0)}{|\boldsymbol{r}-\boldsymbol{r}_0|^3},
\end{align*}
where we have Taylor expanded the \underline{first term} about $|\boldsymbol{r}-\boldsymbol{r}_0|$. For simplicity we set $\boldsymbol{r}_0=\boldsymbol{0}$. Then the \textbf{electric field} generated by our dipole at the origin is:
$$\boldsymbol{E}(\boldsymbol{r})=\frac{1}{4\pi\epsilon_0}\left(\frac{3\boldsymbol{p}\cdot\boldsymbol{r}}{r^5} \boldsymbol{r}-\frac{1}{r^3} \boldsymbol{p}\right),$$
since $\boldsymbol{E}=-\boldsymbol{\nabla}\phi(\boldsymbol{r})$. Note that these formulae are in \underline{Cartesian} coordinates.

Now we repeat this in spherical.

Force, torque and energy.

\subsubsection{Multidipole expansion}

potential

work done

\newpage

\subsection{Gauss's law}
Gauss's law is the integral form of Maxwell's first equation:
$$\int_S\boldsymbol{E}\cdot d\boldsymbol{S}
=\frac{Q_{enc}}{\epsilon_0}$$
where $Q_{enc}$ is the total charge enclosed by volume $V$. This result follows from the application of the \underline{divergence theorem} and is useful in problems with symmetry. 

\subsubsection{Boundaries}

\subsubsection{Conductors}
special case for electrostatics

\newpage

\subsection{Poisson's equation}
In electrostatics we have:
$$\boldsymbol{\nabla}^2\phi=\frac{\rho}{\epsilon_0}$$
where $\rho$ is our charge density. This is the \textbf{Poisson's equation}
and is a consequence of the fact that
$\boldsymbol{\nabla}\times\boldsymbol{E}=\boldsymbol{0}$ and
$\boldsymbol{\nabla}\cdot\boldsymbol{E}=\displaystyle\frac{\rho}{\epsilon_0}$.

\subsubsection{Existence and uniqueness of solutions}
The existence of solutions is given by the fact that:
$$\boldsymbol{E}=-\boldsymbol{\nabla}\phi.$$
Poisson's equation has \textbf{unique} solution $\phi$
if we have volume $V$ bounded by surface $S$
and one of the following boundary conditions:
\begin{enumerate}
    \item 
\end{enumerate}

method of images

\subsection{Capacitors}

\newpage

\section{Magnetostatics}

charge distribution $\implies$ electric field

current $\implies$ magnetic field

\subsection{Currents}

Elementary current

Bulk current density

Surface current density

Line current

\underline{units}!

Infinitesimal current element (dependent on material)\\
units: $Cs^{-1}m=Am$

Note that $J=Am^{-2}$.

Current flowing through \underline{surface} and \underline{line}.




\end{document}
