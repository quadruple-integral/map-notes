\documentclass{article}
\usepackage[margin=0.3in]{geometry}
\usepackage{amsmath}
\usepackage{amsfonts}
\usepackage{amssymb}
\usepackage{amsthm}
\usepackage{parskip}
\usepackage{multicol}
\usepackage{xcolor}
\usepackage{fancyhdr}
\usepackage{physics}
\usepackage{graphicx} % Required for inserting images
\usepackage{hyperref}
\usepackage{enumitem}
\newcommand{\matr}[1]{\mathbf{#1}}
\def\dbar{{\mathchar'26\mkern-12mu d}}

\newtheorem{definition}{Definition}[section]
\newtheorem{theorem}{Theorem}[section]
\newtheorem{corollary}{Corollary}[theorem]
\newtheorem{lemma}[theorem]{Lemma}
\newtheorem*{remark}{Remark}

\pagestyle{fancy}
\fancyhf{}
\renewcommand{\headrulewidth}{1pt}
\fancyhead[R]{\thepage}

\begin{document}

\begin{multicols*}{3}
\noindent

\subsubsection*{D: Supremum and infimum}

\subsubsection*{T: Approximation lemma}

\subsubsection*{D: Completeness of $\mathbb{R}$}
Every nonempty \underline{bounded} subset
of $\mathbb{R}$ has an infimum and supremum.

\subsubsection*{T: Archimedean property}
$\forall a,b\in\mathbb{R}; a>0;\exists n\in\mathbb{N}
: na>b$

\subsubsection*{D1.1: Nested intervals}
A sequence of sets $(I_n)_{n\in\mathbb{N}}$
is nested \\ if $I_1\supset I_2\supset I_3\dots$.

\subsubsection*{T1.1: Nested interval property}
Let $(I_n)_{n\in\mathbb{N}}$ be a sequence of
\underline{nonempty}, \underline{closed}
and \underline{bounded} nested intervals.
Then:
$$E=\bigcap_{n\in\mathbb{N}}I_n
\neq\emptyset.$$
If $\lambda(I_n)\rightarrow0$
then $E$ contains one number,
where $\lambda$ denotes length.

\subsubsection*{T1.2}
Let $E=[a,b]$ and that there exists an \underline{open}
collection of nested intervals
$(I_{\alpha})_{\alpha\in A}$ such that:
$$E\subset\bigcup_{\alpha\in A}I_{\alpha}.$$
Then $\exists\{\alpha_1,\alpha_2,\dots,\alpha_n\}
\subset A$ such that:
$$E\subset I_{\alpha_1}\cup I_{\alpha_2}\cup
\dots\cup I_{\alpha_n}.$$

\subsubsection*{D1.2: $\epsilon$-$N$ convergence}
Let $\displaystyle\lim_{n\rightarrow\infty}x_n=a$. Then:
$$\forall\epsilon>0; \exists N\in\mathbb{N}:
\forall n\geq N\implies |x_n-a|<\epsilon.$$

\subsubsection*{D1.3: Cauchy sequences}
The sequence $(x_n)$ is Cauchy if:
\begin{align*}
    &\forall\epsilon>0;\exists N\in\mathbb{N}:
    \forall n,m\geq N \\ &\implies |x_n-x_m|<\epsilon.
\end{align*}

\subsubsection*{D2.1: Pointwise convergence}
$f_n\rightarrow f$ pointwise on $E$ if:
$$f(x)=\lim_{n\rightarrow\infty}f_n(x).$$
Here $f_n:E\rightarrow\mathbb{R}$.
\begin{align*}
    &\forall x\in E;\forall\epsilon>0;
    \exists N\in\mathbb{N}:\forall n\geq N \\
    &\implies |f_n(x)-f(x)|<\epsilon
\end{align*}

\subsubsection*{T1.3 and T1.4}
Cauchy $\iff$ $\epsilon$-$N$ convergent.

\subsubsection*{D1.4: Subsequences}
The subsequence of $(x_n)_{n\in\mathbb{N}}$ is
a sequence of form $(x_{n_k})_{k\in\mathbb{N}}$
and is a selection of the original sequence 
\textbf{taken in order}.

\subsubsection*{T1.5: Bolzano-Weierstrass}
Every \underline{bounded} real sequence has \textbf{a}
convergent subsequence.

\subsubsection*{D1.5: Limit inferior and superior}
Let $(x_n)$ be a bounded real sequence. Then:
$$\limsup_{n\rightarrow\infty}x_n
=\lim_{n\rightarrow\infty}\left(\sup_{k\geq n}x_k\right)$$
$$\liminf_{n\rightarrow\infty}x_n
=\lim_{n\rightarrow\infty}\left(\inf_{k\geq n}x_k\right).$$

\subsubsection*{T1.6}
The real sequence $(x_n)$ is convergent \textbf{iff}:
$$\limsup_{n\rightarrow\infty}x_n
=\liminf_{n\rightarrow\infty}x_n.$$

\subsubsection*{D1.6: Convergence of infinite series}
Let $\displaystyle S=\sum_{k=1}^{\infty}a_k$ is
convergent if:
$$\lim_{n\rightarrow\infty}\sum_{k=1}^{n}a_k<\infty$$
The infinite series $S$ is \textbf{absolutely convergent} if
$\displaystyle S=\sum_{k=1}^{\infty}|a_k|$ is also convergent.
Otherwise $S$ is conditionally convergent.

\subsubsection*{T1.7: Cauchy criterion for series}
$\displaystyle S=\sum_{k=1}^{\infty}a_k$ is convergent \textbf{iff}:
\begin{align*}
    &\forall\epsilon>0; \exists N\in\mathbb{N}:
    \forall m\geq n\geq N \\
    &\implies \left|\sum_{k=n+1}^{m}a_k\right|<\epsilon.
\end{align*}

\subsubsection*{T1.8}
Let $\displaystyle S=\sum_{k=1}^{\infty}a_k$
be absolutely convergent. Let $z:\mathbb{N}
\rightarrow\mathbb{N}$ be a bijection. Then:
$$\sum_{k=1}^{\infty}a_k
=\sum_{k=1}^{\infty}a_{z(k)}.$$

\subsubsection*{T1.9: Riemann rearrangement}
Let $\displaystyle S=\sum_{k=1}^{\infty}a_k$
be conditionally convergent. Then there exists rearrangements
such that $S$ can take on any value.

\subsubsection*{D1.7: Sequential continuity}

\subsubsection*{T1.10}

\subsubsection*{D1.8: Composition of functions}

\subsubsection*{T1.11}

\subsubsection*{T1.12: $\epsilon$-$\delta$ continuity}

\subsubsection*{T1.13: Intermediate value theorem}

\subsubsection*{T1.14: Extreme value theorem}

\subsubsection*{T: Mean value theorem}

\subsubsection*{D: Differentiability}

\subsubsection*{T: Continuity test}

\subsubsection*{D2.2: Uniform convergence}
$f_n\rightarrow f$ uniformly on $E$ if:
\begin{align*}
    &\forall\epsilon>0;
    \exists N\in\mathbb{N}:\forall n\geq N
    \hspace{0.05in}\text{and}\hspace{0.05in}
    \forall x\in E \\
    &\implies |f_n(x)-f(x)|<\epsilon
\end{align*}

\subsubsection*{P2.1}
The following statements are equivalent.
\begin{enumerate}
    \item $f_n\rightarrow f$ uniformly on $E$
    \item $\displaystyle\lim_{n\rightarrow\infty}
    \sup_{x\in E}|f_n(x)-f(x)|=0$
    \item $\exists a_n\rightarrow0$ s.t.
    $|f_n(x)-f(x)|\leq a_n$ for $\forall x\in E$.
\end{enumerate}

\subsubsection*{T2.1}
If $f_n$ is continuous on $E$ \textbf{and}
$f_n\rightarrow f$ uniformly on $E$
then $f$ is continuous on $E$.

\subsubsection*{Remark}
If $f$ is \underline{not continuous} on $E$
then $f_n$ \underline{cannot} be uniform on $E$.

\subsubsection*{T2.5: Weierstrass M-test}
Let $E\subset\mathbb{R}$ and
$f_k:E\rightarrow\mathbb{R}$. \\
$\exists M_k>0 : \displaystyle
\sum_{k=1}^{\infty}M_k<\infty$. \\
If $\forall k\in\mathbb{N}$
and $\forall x\in E$;
$|f_k(x)|\leq M_k$
then: \\
$\displaystyle\sum_{k=1}^{\infty}f_k(x)$
converges uniformly on $E$.

\subsubsection*{D: Power series}
Let $(a_n)$ be a real sequence and $c\in\mathbb{R}$. Then:
$$f_{PS}(x)=\sum_{n=0}^{\infty}a_n(x-c)^n$$
is a power series centered at $c$, with
\textbf{radius of convergence}:
$$R=\sup\{r\geq0:(a_n r^n)\hspace{0.05in}
\text{is bounded}\}$$
where $R=\infty$ implying that series converges
everywhere.

\subsubsection*{T3.1: Convergence of power series}
Let $0<R<\infty$.
If $|x-c|<R$ then
$f_{PS}(x)$ converges absolutely.

If $|x-c|>R$ then $f_{PS}(x)$ diverges.

\subsubsection*{T3.2: Continuity of power series}
Let $0<r<R$ where $R$ is the radius of
convergence of $f_{PS}(x)$.

Then for $|x-c|\leq r$, $f_{PS}(x)$
converges absolutely and uniformly
to a \underline{continuous} function $f(x)$.

\subsubsection*{L3.1}
$\displaystyle\sum_{n=1}^{\infty}a_n(x-c)^n$
and $\displaystyle\sum_{n=1}^{\infty}n a_n(x-c)^{n-1}$
have the same radius of convergence.

\subsubsection*{T3.3}
Let $R$ be the radius of convegence of
$f_{PS}(x)$. Then for $\forall x:|x-c|<R$,
$f_{PS}(x)$ is \textbf{infinitely differentiable}.
$$f_{PS}(x)=\sum_{n=0}^{\infty}a_n(x-c)^n$$
$$\therefore a_n=\frac{f^{(n)}(c)}{n!}$$

\subsubsection*{Elementary expansions}
\begin{itemize}
    \item $\displaystyle E(x)
    =\sum_{n=0}^{\infty}\frac{x^n}{n!}$
\end{itemize}
\end{multicols*}

\end{document}