\documentclass{article}
\usepackage[margin=0.3in]{geometry}
\usepackage{amsmath}
\usepackage{amsfonts}
\usepackage{amssymb}
\usepackage{amsthm}
\usepackage{parskip}
\usepackage{multicol}
\usepackage{xcolor}
\usepackage{fancyhdr}
\usepackage{physics}
\usepackage{graphicx} % Required for inserting images
\usepackage{hyperref}
\usepackage{enumitem}
\newcommand{\matr}[1]{\mathbf{#1}}
\def\dbar{{\mathchar'26\mkern-12mu d}}

\newtheorem{definition}{Definition}[section]
\newtheorem{theorem}{Theorem}[section]
\newtheorem{corollary}{Corollary}[theorem]
\newtheorem{lemma}[theorem]{Lemma}
\newtheorem*{remark}{Remark}

\pagestyle{fancy}
\fancyhf{}
\renewcommand{\headrulewidth}{1pt}
\fancyhead[R]{\thepage}

\begin{document}

\begin{multicols*}{3}
\noindent

\subsubsection*{D: Supremum}

\subsubsection*{T: Approximation lemma}

\subsubsection*{D: Completeness of $\mathbb{R}$}
Every nonempty \underline{bounded} subset
of $\mathbb{R}$ has an infimum and supremum.

\subsubsection*{T: Archimedean property}
$\forall a,b\in\mathbb{R}; a>0;\exists n\in\mathbb{N}
: na>b$

\subsubsection*{D1.1: Nested intervals}

\subsubsection*{T1.1: Nested interval property}

\subsubsection*{D2.1: Pointwise convergence}
$f_n\rightarrow f$ pointwise on $E$ if:
$$f(x)=\lim_{n\rightarrow\infty}f_n(x).$$
Here $f_n:E\rightarrow\mathbb{R}$.
\begin{align*}
    &\forall x\in E;\forall\epsilon>0;
    \exists N\in\mathbb{N}:\forall n\geq N \\
    &\implies |f_n(x)-f(x)|<\epsilon
\end{align*}

\subsubsection*{D2.2: Uniform convergence}
$f_n\rightarrow f$ uniformly on $E$ if:
\begin{align*}
    &\forall\epsilon>0;
    \exists N\in\mathbb{N}:\forall n\geq N
    \hspace{0.05in}\text{and}\hspace{0.05in}
    \forall x\in E \\
    &\implies |f_n(x)-f(x)|<\epsilon
\end{align*}

\subsubsection*{P2.1}
The following statements are equivalent.
\begin{enumerate}
    \item $f_n\rightarrow f$ uniformly on $E$
    \item $\displaystyle\lim_{n\rightarrow\infty}
    \sup_{x\in E}|f_n(x)-f(x)|=0$
    \item $\exists a_n\rightarrow0$ s.t.
    $|f_n(x)-f(x)|\leq a_n$ for $\forall x\in E$.
\end{enumerate}

\subsubsection*{T2.1}
If $f_n$ is continuous on $E$ \textbf{and}
$f_n\rightarrow f$ uniformly on $E$
then $f$ is continuous on $E$.

\subsubsection*{Remark}
If $f$ is \underline{not continuous} on $E$
then $f_n$ \underline{cannot} be uniform on $E$.

\subsubsection*{T2.5: Weierstrass M-test}
Let $E\subset\mathbb{R}$ and
$f_k:E\rightarrow\mathbb{R}$. \\
$\exists M_k>0 : \displaystyle
\sum_{k=1}^{\infty}M_k<\infty$. \\
If $\forall k\in\mathbb{N}$
and $\forall x\in E$;
$|f_k(x)|\leq M_k$
then: \\
$\displaystyle\sum_{k=1}^{\infty}f_k(x)$
converges uniformly on $E$.

\subsubsection*{D: Power series}
Let $(a_n)$ be a real sequence and $c\in\mathbb{R}$. Then:
$$f_{PS}(x)=\sum_{n=0}^{\infty}a_n(x-c)^n$$
is a power series centered at $c$, with
\textbf{radius of convergence}:
$$R=\sup\{r\geq0:(a_n r^n)\hspace{0.05in}
\text{is bounded}\}$$
where $R=\infty$ implying that series converges
everywhere.

\subsubsection*{T3.1: Convergence of power series}
Let $0<R<\infty$.
If $|x-c|<R$ then
$f_{PS}(x)$ converges absolutely.

If $|x-c|>R$ then $f_{PS}(x)$ diverges.

\subsubsection*{T3.2: Continuity of power series}
Let $0<r<R$ where $R$ is the radius of
convergence of $f_{PS}(x)$.

Then for $|x-c|\leq r$, $f_{PS}(x)$
converges absolutely and uniformly
to a \underline{continuous} function $f(x)$.

\subsubsection*{L3.1}
$\displaystyle\sum_{n=1}^{\infty}a_n(x-c)^n$
and $\displaystyle\sum_{n=1}^{\infty}n a_n(x-c)^{n-1}$
have the same radius of convergence.

\subsubsection*{T3.3}
Let $R$ be the radius of convegence of
$f_{PS}(x)$. Then for $\forall x:|x-c|<R$,
$f_{PS}(x)$ is \textbf{infinitely differentiable}.
$$f_{PS}(x)=\sum_{n=0}^{\infty}a_n(x-c)^n$$
$$\therefore a_n=\frac{f^{(n)}(c)}{n!}$$

\subsubsection*{Elementary expansions}
\begin{itemize}
    \item $\displaystyle E(x)
    =\sum_{n=0}^{\infty}\frac{x^n}{n!}$
\end{itemize}
\end{multicols*}

\end{document}