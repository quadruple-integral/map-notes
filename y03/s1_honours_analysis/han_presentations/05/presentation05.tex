\documentclass{beamer}
\usetheme{Boadilla}
\usepackage{graphicx} % Required for inserting images
\usepackage{amsmath}
\usepackage{amsfonts}
\usepackage{amssymb}
\usepackage{tikz}
\usepackage{chemformula}

\title{Rearrangement of series}
\author{Christopher Shen}
\date{October 2023}

\begin{document}

\maketitle

\begin{frame}{Example 1: An incorrect argument}
    \begin{center}
        Let $S=1+2+3+4+\dots$ and let:

        $$S_1=1-1+1-1+1-1+\dots\stackrel{?}{=}\frac{1}{2}.$$
    \end{center}
\end{frame}

\begin{frame}{Example 1: An incorrect argument}
    \begin{center}
        Consider $S_2=1-2+3-4+5-6+\dots$.
    \end{center}
    \begin{align*}
        2S_2
        &=1-2+3-4+5-6+\dots \\
        &\quad\quad+1-2+3-4+5-6+\dots \\
        &=1-1+1-1+1-1+\dots \\
        &\stackrel{?}{=}\frac{1}{2}
    \end{align*}
    $$\therefore S_2=1-2+3-4+5-6+\dots=\frac{1}{4}$$
\end{frame}

\begin{frame}{Example 1: An incorrect argument}
    And finally:
    \begin{align*}
        S-S_2
        &=1+2+3+4+5+\dots \\
        &-(1-2+3-4+5-6+\dots) \\
        &=4(1+2+3+4+\dots) \\
        &=4S
    \end{align*}
    $$\therefore S=1+2+3+4+5+\dots=-\frac{1}{12}$$
\end{frame}

\begin{frame}{Example 1: An incorrect argument}
    This is completely wrong!
    \begin{center}
        $S=\displaystyle\sum_{k=1}^{\infty}a_k$ exists
        \textbf{if and only if}
        $\displaystyle\lim_{n\rightarrow\infty}
        \sum_{k=1}^{n}a_k<\infty$.
    \end{center}  
    In our case:
    $$\displaystyle\lim_{n\rightarrow\infty}
    \Bigl[1-1+1-1+1-1+\dots\Bigr]\hspace{0.1in}\text{DNE}$$
    and
    \begin{align*}
        S
        &=1+2+3+4+\dots \\
        &=\infty \\
        &\neq -\frac{1}{12}.
    \end{align*}
\end{frame}

\begin{frame}{Example 2: A conditionally convergent series}
    \begin{align*}
        \sum_{n=1}^{\infty}\left((-1)^{n+1}\frac{1}{n}\right)
        &=1-\frac{1}{2}+\frac{1}{3}-\frac{1}{4}+\dots \\
        &=\ln 2
    \end{align*}
    \begin{center}
        This is the alternating harmonic series.
    \end{center}
\end{frame}

\begin{frame}{Example 2: A conditionally convergent series}
    \begin{center}
        Know:
        $1-\frac{1}{2}+\frac{1}{3}-\frac{1}{4}+\dots=\ln 2$
    \end{center}
    $$\therefore \frac{1}{2}-\frac{1}{4}+\frac{1}{6}
    -\frac{1}{8}+\frac{1}{10}-\frac{1}{12}+\dots=\frac{1}{2}\ln 2$$
    \begin{align*}
        &1-\frac{1}{2}+\frac{1}{3}-\frac{1}{4}+\frac{1}{5}-\frac{1}{6}+\dots \\
        &\ \ +\frac{1}{2}\quad\quad-\frac{1}{4}\quad\quad+\frac{1}{6}+\dots \\
        &=\frac{3}{2}\ln 2
    \end{align*}
    $$\therefore 1-\frac{1}{2}+\frac{1}{3}-\frac{1}{4}+\dots=\frac{3}{2}\ln 2$$
\end{frame}

\begin{frame}{Example 2: A conditionally convergent series}
    \begin{center}
        Our previous argument was only valid because:
    \end{center}
    \begin{align*}
        \sum_{n=1}^{\infty}\left(|(-1)^{n+1}\frac{1}{n}|\right)
        &=\sum_{n=1}^{\infty}\frac{1}{n} \\
        &=\infty
    \end{align*}
\end{frame}

\begin{frame}
    \begin{definition}[Conditional convergence]
        Let $S=\sum_{k=1}^{\infty}a_k$. Series S is
        conditionally convergent if $\sum_{k=1}^{\infty}a_k<\infty$
        yet $\sum_{k=1}^{\infty}|a_k|=\infty$.
    \end{definition}

    \begin{theorem}[Riemann rearrangement]
        Let $S=\sum_{k=1}^{\infty}a_k$ be a conditionally
        convergent series. Then there exists rearrangements
        $z:\mathbb{N}\rightarrow\mathbb{N}$ such that:
        $$\sum_{k=1}^{\infty}a_{z(k)}$$
        may take on any value.
    \end{theorem}
\end{frame}

\end{document}