\pagestyle{fancy}
\fancyhead{}
\fancyhead[L]{Honours Analysis Workshop 7}
\fancyhead[R]{Winter 2023}

\section{Workshop 7}
\begin{enumerate}
    \item Let $f(x)=[x]$ for $\forall x\in\mathbb{R}$.    Find the following integrals:
    $$\int_{(0,5)}f$$
    and
    $$\int_{(-\frac{7}{3},\frac{12}{5}]}f.$$
    
    Note that here we denote $[x]$ as the \textbf{floor function}. The floor function \textbf{rounds down} its input to the closest integer. So for example we have that $[3.5]=3$ and $[-2.5]=-3$.

    Let's first consider the open interval $(0,5)$. Notice that we can write $f$ as a sum of characteristic functions each with interval of length $1$:
    $$f(x)=\sum_{j=1}^{5}(j-1)\chi_{[j-1,j)}(x).$$
    Recall the characteristic function definition:
    $$\chi_{[j-1,j)} =
    \left\{
    \begin{array}{ll}
	1  & \mbox{} x \in [j-1,j) \\
	0 & \mbox{otherwise}.
    \end{array}
    \right.$$
    Integrating this gives us:
    \begin{align*}
        \int_{(0,5)}f
        &=\int\sum_{j=1}^{5}(j-1)\chi_{[j-1,j)}(x) \\
        &=\sum_{j=1}^{5}(j-1)\int\chi_{[j-1,j)}(x) \\
        &=\sum_{j=1}^{5}(j-1) \\
        &=10.
    \end{align*}

    \newpage

    So now consider the semi-open interval $(-\frac{7}{3},\frac{12}{5}]$, for $f(x)=[x]$.

    We write this function as a sum of characteristic functions:
    \begin{align*}
        f(x)
        &=-3\chi_{(-\frac{7}{3},-2)}+-2\chi_{[-2,-1)}+-1\chi_{[-1,0)}+0
        +1\chi_{[1,2)}+2\chi_{[2,\frac{12}{5}]} \\
        &=-3\chi_{(-\frac{7}{3},-2)}+\sum_{j=-2}^{1}j\chi_{[j,j+1)}+2\chi_{[2,\frac{12}{5}]}.
    \end{align*}
    Then integrating gives:
    \begin{align*}
        \int_{(-\frac{7}{3},\frac{12}{5}]}f
        &=-1+-2+\frac{4}{5} \\
        &=-\frac{11}{5}.
    \end{align*}

    \newpage

    \item Let $f(x)=[nx]^2$ for $\forall x\in\mathbb{R}$ and $n\in\mathbb{N}$.

    Show that:
    $$\int_{(0,1)}f=\frac{1}{n}\sum_{j=1}^{n-1}j^2.$$ \\

    So first consider:
    $$x\in [\frac{j}{n},\frac{j+1}{n})$$
    for $j\in\{1,\dots,n-1\}$. Multiplying each of these intervals by $n$ gives:
    $$nx\in[j,j+1),$$
    and this also works for negative $n$ values. Taking the floor for each interval:
    $$[nx]=j\hspace{0.1in}\text{for}\hspace{0.1in}\forall nx\in[j,j+1)$$
    and squaring this gives $[nx]^2=j^2$ for $\forall j\in\{1,\dots,n-1\}$.

    So we can now write our function $f$ as the sum of characteristic functions:
    $$f(x)=0\cdot\chi_{(0,\frac{j}{n})}+\sum_{j=1}^{n-1}j^2 \chi_{
    [\frac{j}{n},\frac{j+1}{n})}(x)$$
    and integrating this gives:
    \begin{align*}
        \int_{(0,1)}f
        &=\sum_{j=1}^{n-1}j^2\int\chi_{[\frac{j}{n},\frac{j+1}{n})}(x) \\
        &=\sum_{j=1}^{n-1}j^2\cdot\frac{1}{n},
    \end{align*}
    since we have defined $n$ intervals each of length $\frac{1}{n}$. Finally:
    \begin{align*}
        \int_{(0,1)}f
        &=\frac{1}{n}\sum_{j=1}^{n-1}j^2 \\
        &=\frac{1}{n}\cdot\frac{n(n-1)(2n-1)}{6} \\
        &=\frac{1}{6}(n-1)(2n-1).
    \end{align*}

    \newpage

    \item Let $f(x)=\frac{1}{[x]^2}$ for $\forall x\geq1$. Show that $f$ is integrable on $[1,\infty)$ and:
    $$\int_{[1,\infty)}f=\sum_{j=1}^{\infty}\frac{1}{j^2}$$ \\

    Choose $J_j=[j,j+1)$ for $j\in\mathbb{N}$. Firstly we verify that:
    \begin{align*}
        \sum_{j=1}^{\infty}|c_j|\lambda(J_j)
        &=\sum_{j=1}^{\infty}\frac{1}{j^2} \\
        &=\frac{\pi^2}{6}<\infty
    \end{align*}
    with $c_j=\frac{1}{j^2}$ and our interval of choice being of length $1$.

    Now $\forall x\in J_i$ where $i\in\mathbb{N}$, we have that:
    \begin{align*}
        \sum_{j=1}^{\infty}|c_j| \chi_{J_j}(x)
        &=\sum_{j=1}^{\infty}\frac{1}{j^2}\chi_{J_j}(x) \\
        &=\frac{1}{i^2}<\infty
    \end{align*}
    Hence we have proven that $f$ is Lebesgue integrable on $[1,\infty)$, and:
    \begin{align*}
        \int_{[1,\infty)} f
        &=\sum_{j=1}^{\infty}\frac{1}{j^2}
        \int\chi_{J_j}(x) \\
        &=\sum_{j=1}^{\infty}\frac{1}{j^2}\lambda(J_j) \\
        &=\sum_{j=1}^{\infty}\frac{1}{j^2},
    \end{align*}
    and we are finished.
    
    \newpage

    \item So now consider the function:
    $$f(x)=\left\{
    \begin{array}{ll}
	1  & \mbox{} x \notin\mathbb{Q} \\
	0 & \mbox{} x\in\mathbb{Q}
    \end{array}.
    \right.$$
    Show that $f$ is integrable on every \textbf{bounded} interval $I$ and:
    $$\int_I f=\lambda(I).$$ \\

    \begin{proof}
    Firstly choose:
    $$c_j=\left\{
    \begin{array}{ll}
	1  & \mbox{} j=1 \\
	-1 & \mbox{} j>1
    \end{array},
    \right.$$
    and
    $$J_j=\left\{
    \begin{array}{ll}
	I  & \mbox{} j=1 \\
	q_{j-1} & \mbox{} j>1
    \end{array},
    \right.$$
    where we define $I\cap\mathbb{Q}=\{q_1,q_2,\dots\}$ and $j\in\mathbb{N}$.

    Then:
    \begin{align*}
        \sum_{j=1}^{\infty}|c_j|\lambda(J_j)
        &=\lambda(I)+\sum_{j=2}^{\infty}-\lambda(\{q_{j-1}\}) \\
        &=\lambda(I) \\
        &<\infty
    \end{align*}
    since interval $I$ is bounded and hence of finite length.

    Finally $\forall x\in I$:
    \begin{align*}
        f(x)
        &=\sum_{j=1}^{\infty}|c_j|\chi_{J_j}(x) \\
        &=\chi_I(x)+\sum_{j=1}^{\infty} -\chi_{\{q_j\}}(x) \\
        &<\infty
    \end{align*}
    So if $x\in\mathbb{Q}$ then $f(x)=0$, and vice versa.
    
    Therefore our function $f$ is integrable on bounded $I$ with formula:
    $$\int_I f=\lambda(I).$$
    \end{proof}

    \newpage

    \item Let $f:[a,b]\rightarrow\mathbb{R}$ be continuous.

    Let $M=\displaystyle\sup_{x\in[a,b]}|f(x)|$ and $p>0$. For part ($a$) show that:

    $\forall\epsilon:(0<\epsilon<M/2);\exists(\alpha,\beta)\subset[a,b]:$
    $$(M-\epsilon)^p(\beta-\alpha)\leq\int_{a}^{b}|f(x)|^p dx\leq M^p(b-a).$$ \\

    \begin{proof}
    Direct proof.

    Using the approximation property for suprema, $\exists x_0\in[a,b]$:
    $$\sup_{x\in[a,b]}|f(x)|-\epsilon<|f(x_0)|.$$
    Then choose a $(\alpha,\beta)\subset[a,b]$ such that $\forall\epsilon>0; \forall x\in(\alpha,\beta):$
    $$\sup_{x\in[a,b]}|f(x)|-\epsilon<|f(x)|.$$
    Since these are strictly positive values taking the power of $p$ preserves signs:
    $$(M-\epsilon)^p<|f(x)|^p.$$
    Also by the definition of supremum, for $\forall x\in[a,b]:$
    $$|f(x)|\leq\sup_{x\in[a,b]}|f(x)|$$
    and taking the $p$th gives:
    $$|f(x)|^p\leq M^p.$$
    Assuming the integrability of $f$ we use the integral comparison test:
    $$\int_{a}^{b}|f(x)|^p\leq M^p(b-a).$$
    Similarly:
    $$(M-\epsilon)^p(\beta-\alpha)<\int_{\alpha}^{\beta}|f(x)|^pdx.$$
    But because $(\alpha,\beta)\subset[a,b]$:
    $$\therefore(M-\epsilon)^p(\beta-\alpha)<\int_{\alpha}^{\beta}|f(x)|^pdx
    \leq\int_{a}^{b}|f(x)|^pdx\leq M^p(b-a).$$
    \end{proof}

    \newpage

    For part ($b$) we want:
    $$\lim_{p\rightarrow\infty}\left(
    \int_{a}^{b}|f(x)|^p dx\right)^{1/p}=M$$ \\

    \begin{proof}
        From part ($a$) we have that $\forall\epsilon:0<\epsilon<M/2$:
        $$(M-\epsilon)^p(\beta-\alpha)\leq\int_{a}^{b}|f(x)|^p dx\leq M^p(b-a)$$
        where $a<\alpha<\beta<b$. Taking the $p$th root gives:
        $$(M-\epsilon)(\beta-\alpha)^{1/p}\leq\left(\int_{a}^{b}|f(x)|^p dx\right)^{1/p}\leq M(b-a)^{1/p}.$$
        Now by definition $\beta-\alpha>0$ and $b-a>0$. So taking $p\rightarrow\infty$:
        $$M-\epsilon\leq\left(\int_{a}^{b}|f(x)|^p dx\right)^{1/p}\leq M<M+\epsilon$$
        for all $0<\epsilon<M/2$, by monotone convergence theorem. Then:
        $$|\left(\int_{a}^{b}|f(x)|^p dx\right)^{1/p}-M|<\epsilon,$$
        or that
        $$\lim_{p\rightarrow\infty}\left(\int_{a}^{b}|f(x)|^p dx\right)^{1/p}=M.$$
    \end{proof}

    \newpage

    \item Let $f(x)=n$ for $\forall x\in((n+1)^{-2},n^{-2}]$ and $n\in\mathbb{N}$. Show that:
    $$\int_{(0,1]}f=\sum_{j=1}^{\infty}\frac{1}{j^2}$$ \\

    We write our function as the following sum:
    $$f(x)=\sum_{j=1}^{\infty}\chi_{(0,\frac{1}{j^2}]}(x).$$
    This expression is clearly finite. We then check that:
    \begin{align*}
        \sum_{j=1}^{\infty}|c_j|\lambda(J_j)
        &=\sum_{j=1}^{\infty}\lambda((0,\frac{1}{j^2}]) \\
        &=\sum_{j=1}^{\infty}\frac{1}{j^2} \\
        &=\frac{\pi^2}{6}<\infty.
    \end{align*}
    Finally:
    $$\therefore\int_{(0,1)}f=\sum_{j=1}^{\infty}\frac{1}{j^2}.$$
    But \textcolor{red}{why not}:
    $$f(x)=\sum_{j=1}^{\infty}j\cdot\chi_{(\frac{1}{(j+1)^2},\frac{1}{j^2}]}(x)?$$
\end{enumerate}