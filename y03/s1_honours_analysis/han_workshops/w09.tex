\pagestyle{fancy}
\fancyhead{}
\fancyhead[L]{Honours Analysis Workshop 9}
\fancyhead[R]{Winter 2023}

\section{Workshop 9}
\begin{enumerate}
    \item Show that $\displaystyle\chi_E$
    is not Riemann-integrable, where $E=\mathbb{Q}\cap[0,1]$. \\

    This is known as the Dirichlet function. Firstly let:
    $$\mathbb{Q}\cap[0,1]=\{q_0,q_1,\dots\}$$
    and is the set of rationals between zero and one.
    Clearly we have that $q_0=0$ and $q_j\rightarrow1$.
    Then let $I_j=(q_{j-1},q_j)$ where $j\in\mathbb{N}$
    which implies:
    $$\sup_{x,y\in I_j}|f(x)-f(y)|=1$$
    if $f(x)=\chi_E$. We know that $f(x)$ in Riemann-integrable
    \underline{if and only if}:
    $$\sum_{j=1}^{n}\sup_{x,y\in I_j}|f(x)-f(y)|
    \lambda(I_j)<\epsilon$$
    yet we have that:
    $$\sum_{j=1}^{n}\sup_{x,y\in I_j}|f(x)-f(y)|
    \lambda(I_j)
    =\sum_{j=1}^{n}\lambda(I_j)=1$$
    and hence our function is not Riemann-integrable. \\

    \item For part ($i$) show that: \\
    If $f$ is Riemann-integrable
    then $|f|$ is also Riemann-integrable. \\

    Let $f$ be Riemann-integrable. Then by L4.1:
    $$\sum_{j=1}^{n}\sup_{x,y\in I_j}
    |f(x)-f(y)|\lambda(I_j)<\epsilon$$
    where $I_j=(x_{j-1},x_j)$
    and $a=x_0<\dots<x_n=b$.

    Now using the \textbf{reverse triangle inequality}:
    $$\bigl||f(x)|-|f(y)|\bigr|\leq|f(x)-f(y)|$$
    for $\forall x,y\in I_j$ we then have that:
    $$\sup_{x,y\in I_j}\bigl||f(x)|-|f(y)|\bigr|
    \leq\sup_{x,y\in I_j}|f(x)-f(y)|$$
    and therefore:
    $$\sum_{j=1}^{n}\sup_{x,y\in I_j}
    \bigl||f(x)|-|f(y)|\bigr|\lambda(I_j)<\epsilon$$
    or that $|f|$ is also Riemann-integrable.

    \newpage

    For part ($ii$) \underline{disprove} that: \\
    Let $|f|$ be Riemann-integrable. Then
    $f$ is also Riemann-integrable. \\

    So consider the following function:
    $$\chi_E(x)=
    \left\{
    \begin{array}{ll}
	1  & \mbox{$x$ is rational}\\
	-1 & \mbox{otherwise}
    \end{array}
    \right.$$
    where $E=\mathbb{Q}\cap[0,1]$. Taking the modulus
    of this function gives:
    $$|\chi_E(x)|=1$$
    for $\forall x\in[0,1]$ and clearly:
    $$\sup_{x,y\in I_j}||\chi_E(x)|-|\chi_E(y)||=0$$
    where $I_j=(x_{j-1},x_j)$, $j=1,2,\dots$ and
    $0=x_0<\dots<x_n=1$. Then by L4.1,
    $|\chi_E|$ is Riemann-integrable.
    However this is not true without the modulus:
    $$\sup_{x,y\in I_j}|\chi_E(x)-\chi_E(y)|=2$$
    and hence again via L4.1 this function is not
    Riemann-integrable. \\

    \item Let $-\infty\leq a<b<\infty$ and let
    $f$ be integrable on $(u,b)$ for $\forall u\in(a,b)$. \\
    Then $f$ is integrable on interval $(a,b)$ \textbf{if and only if}:
    $$\exists m<\infty:\forall u\in(a,b);\int_{u}^{b}|f|<m.$$

    \newpage

    \item Show that the following statements are equivalent:
    \begin{itemize}
        \item $\exists M<\infty$ such that $\forall v\in(a,b)$:
        $$\int_{a}^{v}|f|\leq M$$

        \item Consider partition of $(a,b)$:
        $$a<v_1<v_2<\dots<b$$
        and define $I_1=(a,v_1]$, $I_j=(v_{j-1},v_j]$
        where $j=2,3,\dots$.

        We then have that $\exists M<\infty$ such that:
        $$\sum_{j=1}^{n}\int_{I_j}|f|\leq M$$
        where $n\in\mathbb{N}$. \\
    \end{itemize}

    We begin by assuming the first condition. Let:
    $$\exists M<\infty:\forall v\in(a,b);
    \int_{a}^{v}|f|\leq M.$$
    Since this holds for all elements in $(a,b)$ we
    order our elements as $v_m$ where $m=1,2,\dots$
    and notice the following equality:
    $$\int_{a}^{v_m}|f|=\sum_{j=1}^{m}\int_{I_j}|f|\leq M.$$
    For the opposite direction assume that $\forall I_j$:
    $$\sum_{j=1}^{n}
    \int_{I_j}|f|\leq M<\infty$$
    and using T4.8(d) in lecture notes implies:
    $$\int_{I=(a,b)}|f|=\sum_{j=1}^{\infty}\int_{I_j}|f|.$$
    Then T4.8(a) implies that the integral of a subinterval
    is also defined and bounded.

    \newpage

    \item Now show the converse of question 4.
    Let $f$ be integrable on $(a,b)$. Then:
    $$\exists M<\infty:\forall v\in(a,b);
    \int_{a}^{v}|f|\leq M.$$ \\

    If $f$ is integrable
    then its modulus $|f|$ must be also integrable.
    Now its integral value must be defined and bounded.
    Pick $M$ to bound our integral:
    $$\int_{(a,b)}|f|<M<\infty$$
    and using T4.8(c) gives us that every subinterval is also
    integrable and bounded. \\
\end{enumerate}

Questions 4 and 5 constitutes the proof to the following result.
\begin{theorem}
    \hfill \\
    Let $-\infty\leq a<b\leq\infty$ and let
    $f$ be integrable on $(a,v)$ for $\forall v\in(a,b)$. \\
    Then $f$ is integrable on $(a,b)$ \textbf{if and only if}:
    $$\exists M<\infty:\forall v\in(a,b);
    \int_{a}^{v}|f|\leq M$$
\end{theorem}

Similarly we have that:
\begin{theorem}
    \hfill \\
    Let $-\infty\leq a<b<\infty$ and let
    $f$ be integrable on $(u,b)$ for $\forall u\in(a,b)$. \\
    Then $f$ is integrable on interval $(a,b)$ \textbf{if and only if}:
    $$\exists m<\infty:\forall u\in(a,b);\int_{u}^{b}|f|<m.$$
\end{theorem}

\newpage

\begin{enumerate}[resume]
    \item Show the following: \\
    Let $-\infty\leq a<b<\infty$ and let
    $f$ be integrable on $(u,b)$ for $\forall u\in(a,b)$. \\
    Then $f$ is integrable on interval $(a,b)$ \textbf{if and only if}:
    $$\exists M<\infty:\forall u\in(a,b);\int_{u}^{b}|f|<M.$$ \\

    \begin{proof}
        $\leftarrow$ direction. \\
        Firstly we show that the following are equivalent:
        \begin{itemize}
            \item $\exists M<\infty$ such that $\forall u\in(a,b)$:
            $$\int_{u}^{b}|f|\leq M$$
    
            \item Consider partition of $(a,b)$:
            $$a<\dots<u_2<u_1<b$$
            and define $I_1=[u_1,b)$, $I_i=[u_i, u_{i-1})$
            where $i=2,3,\dots$.
    
            We then have that $\exists M<\infty$ such that:
            $$\sum_{j=1}^{n}\int_{I_j}|f|\leq M$$
            where $n\in\mathbb{N}$.
        \end{itemize}
        We begin by assuming the first condition. Let:
        $$\exists M<\infty:\forall u\in(a,b);
        \int_{u}^{b}|f|\leq M.$$
        Since this holds for all elements in $(a,b)$ we
        order our elements as $u_m$ where $m=1,2,\dots$
        and the following equality holds via T4.9:
        $$\int_{u_m}^{b}|f|=\sum_{j=1}^{m}\int_{I_j}|f|\leq M.$$
        For the opposite direction assume that
        $\forall n\in\mathbb{N}$:
        $$\sum_{j=1}^{n}
        \int_{I_j}|f|\leq M<\infty$$
        and using T4.8(d):
        $$\sum_{j=1}^{\infty}\int_{I_j}|f|
        =\int_{I=(a,b)}|f|<\infty.$$
        Then T4.8(a) implies that the integral of a subinterval
        is also defined and bounded.
        This also implies that
        $f$ is integrable on $(a,b)$.
    \end{proof}

    \newpage

    \begin{proof}
        $\rightarrow$ direction. \\
        time is up.
    \end{proof}

    \newpage

    \item
\end{enumerate}