\pagestyle{fancy}
\fancyhead{}
\fancyhead[L]{Honours Analysis Workshop 8}
\fancyhead[R]{Winter 2023}

\section{Workshop 8}
\begin{enumerate}
    \item 1
    \item 2
    \item 3
    \item 4
    \item 5
    \item 6
    \newpage
    \item Define $L(x)=\displaystyle\int_{1}^{x}\frac{\dd t}{t}$
    for $\forall x>0$. Show:
    \begin{itemize}
        \item $L(xy)=L(x)+L(y)$
        \item $L'(x)=\displaystyle\frac{1}{x}$
        \item $L_{inv}(x)=E(x)$, where we define
        $E(x)=\displaystyle\sum_{k=0}^{\infty}\frac{x^k}{k!}$. \\
    \end{itemize}

    For the first part we want to show:
    $$\int_{1}^{yx}\frac{\dd t}{t}=\int_{1}^{x}\frac{\dd t}{t}
    +\int_{1}^{y}\frac{\dd t}{t}.$$
    Beginning from the left hand side let $t=x\alpha$.
    $$\therefore\int_{t=1}^{t=xy}\implies
    \int_{\alpha=\frac{1}{x}}^{\alpha=y}$$
    $$\therefore\dd t=x\dd \alpha$$
    $$\therefore\frac{1}{t}=\frac{1}{x\alpha}$$
    Now splitting this integral via T4.9 gives:
    \begin{align*}
        \int_{t=1}^{t=yx}\frac{\dd t}{t}
        &=\int_{\alpha=\frac{1}{x}}^{\alpha=y}
        \frac{\dd\alpha}{\alpha} \\
        &=\int_{\alpha=1}^{\alpha=y}
        \frac{\dd\alpha}{\alpha}+
        \int_{\alpha=\frac{1}{x}}^{\alpha=1}
        \frac{\dd\alpha}{\alpha} \\
        &=\int_{\alpha=1}^{\alpha=y}
        \frac{\dd\alpha}{\alpha}+
        \int_{\beta=1}^{\beta=x}
        \frac{\dd\beta}{\beta}
    \end{align*}
    where we set $\alpha=\frac{1}{x}\beta$ in the second integral.
    $$\therefore L(xy)=L(x)+L(y)$$

    Using the fundamental theorem of calculus:
    \begin{align*}
        L(x)=\displaystyle\int_{1}^{x}\frac{\dd t}{t}
        &\implies \dv{x}L(x)=\frac{1}{x}
    \end{align*}
    since $\forall t>0$, $\frac{1}{t}$ is continuous.

    \newpage

    For the final part let's first define our functions:
    $$E:\mathbb{R}\rightarrow\mathbb{R}$$
    $$L:\mathbb{R^+}\rightarrow\mathbb{R}$$
    where $\mathbb{R^+}=\mathbb{R}\backslash\{0,\dots\}$ represents the positive reals. Then define:
    $$E(x)=z$$
    for $x,z\in\mathbb{R}$ and:
    $$L(y)=x$$
    for $y\in\mathbb{R^+}$.
    
    For these two functions to be inverses
    of each other we must show that:
    $$E(L(y))=y$$
    and
    $$L(E(x))=x.$$
    Consider
    $$\dv{y}E(L(y))=E(L(y))\frac{1}{y}.$$
    Rearranging this and taking integrals:
    $$\int_{1}^{E(L(y))}\frac{1}{E(L(y))}\dd E(L(y))
    =\int_{1}^{y}\frac{1}{y}\dd y.$$
    This gives:
    $$\Bigl[L(E(L(y)))\Bigl]_{E(L(y))=1}^{E(L(y))=E(L(y))}
    =\left[L(y)\right]_{1}^{y}$$
    or that:
    $$L(E(L(y)))=L(y).$$
    $$\therefore E(L(y))=y$$
    This is fine since $y\in\mathbb{R^+}\subset\mathbb{R}$.
    Similarly consider the following:
    $$\dv{x}L(E(x))=\frac{1}{E(x)}E(x)=1.$$
    Here $L(E(x))$ is defined as $\forall x\in\mathbb{R};
    E(x)>0$.
    
    Integrating our expression as an indefinite integral:
    $$L(E(x))=x+k$$
    and we find that $k=0$ by setting $x=0$.
    $$\therefore L(E(x))=x$$

    \newpage

    \item Let $g:[a,b]\rightarrow\mathbb{R}$ be continuous, and that
    $g\geq0$ for $\forall x\in[a,b]$. Then let:
    $$\int_{a}^{b}g(x)\dd x=0.$$
    Show that $\forall x\in[a,b]$ we have $g(x)=0$. \\

    Firstly because $g\geq0$ splitting the integral using T4.9:
    $$\int_{a}^{b}g(x)\dd x
    =\int_{a}^{c}g(x)\dd x+\int_{c}^{b}g(x)\dd x=0$$
    implies that $\forall c\in[a,b]$:
    $$\int_{a}^{c}g(x)\dd x=0$$
    as areas of positive functions are always positive.

    Since $g(x)$ is continuous we can use the fundamental
    theorem of calculus.
    
    Let:
    $$G(x)=\int_{a}^{x}g(t)\dd t=0$$
    for $\forall x\in[a,b]$ as shown above.
    We then have that:
    $$g(x)=\dv{x}G(x)=0$$
    for $\forall x\in[a,b]$.
\end{enumerate}