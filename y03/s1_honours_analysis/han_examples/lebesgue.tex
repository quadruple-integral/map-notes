\section{Lebesgue integrals}

\subsection{Characteristic and step functions}
\begin{enumerate}
    \item Let $\phi$ and $\psi$ be step functions. Show that the following are also step functions:
    \begin{itemize}
        \item $\alpha\phi+\beta\psi$
        \item $\max\{\phi,\psi\}$ and $\min\{\phi,\psi\}$
        \item $\phi\psi$. \\
    \end{itemize}

    Firstly define step function $\phi$ wrt finite set $\{x_0,x_1\dots,x_n\}$
    and step function $\psi$ wrt finite set $\{y_0,y_1\dots,y_n\}$.
    Their linear combination is a step function wrt the union of these two finite sets.

    The rest are proven similarly via construction.

    \item Show that $\phi$ is a step function \textbf{if and only if}:
    $$\phi(x)=\sum_{j=1}^{n}c_j\chi_{J_j}(x)$$
    where $c_j$ is some constant and $J_j$ some interval. \\

    From the definition of $\phi$ we can write:
    $$\phi(x)=\sum_{j=1}^{n}c_j\chi_{(x_{j-1},x_j)}(x)
    +\sum_{j=0}^{n}\phi(x_j)\chi_{\{x_j\}}$$
    and the opposite direction is shown by the first example
    and the fact that a characteristic function is a step function. \\

    \item Write step function $\phi=\chi_{[0,1]}$
    in two ways. \\

    We can write it as piecewise function. (def of $chi$)

    But also as def 4.1 constants.

    So integral either length of $[0,1]$
    or sum of constants times subintervals.

    \newpage

    \item Show that the following function
    is Lebesgue integrable:
    $$f(x)=\frac{1}{[x][x+1]}$$
    where $[x]$ denotes the floor function. \\

    So:
    $$f=\sum_{j=1}^{\infty}|\frac{1}{j(j+1)}|\chi_{[j,j+1)}(x)$$
    and integral value is $1$. \\

    \item Show that $\chi_E$ is integrable where $E\subset I$
    and is bounded and \underline{countable}. \\

    $E=\{e_1,e_2,\dots\}$ and so:
    $$\chi_E=\sum_{j=1}^{\infty}e_j\chi_{\{e_j\}}(x)$$
    which has integral of zero. \\

    \item Cantor set is Lebesgue integrable.
\end{enumerate}

\newpage

\subsection{Riemann integrals}
\begin{enumerate}
    \item Show that \underline{step functions}
    are Riemann-integrable. \\

    Because $\phi\leq\phi\leq\phi$ and using definition. \\

    \item Show that if $f$ is Riemann-integrable on $[a,b]$
    then $f$ is bounded and has bounded support.

    Further show that if $f(x)=0$ when $x\notin[a,b]$
    then our step functions are \underline{zero} outside $[a,b]$. \\

    \item Show that the following functions are \textbf{not}
    Riemann-integrable:
    \begin{itemize}
        \item $f(x)=e^{-|x|}$
        \item $g(x)=x^{-\frac{1}{2}}\chi_{(0,1)}(x)$ \\
    \end{itemize}

    Not quite sure about $f$ yet, maybe because we
    cannot define a step function wrt to an infinite set?

    The function $g(x)$ has no upper step function. \\

    \item Find \underline{infinitely countable} $E\subset\mathbb{R}$ such that
    $\chi_E$ is Riemann-integrable.

    Is every such $\chi_E$ is Riemann-integrable?

    Consider $E=\{\frac{1}{n}:n\in\mathbb{N}\}$.

    Consider the Dirichlet function.
\end{enumerate}

\newpage

\subsection{Lebesgue integrals}

\subsubsection{Basic properties}
\begin{enumerate}
    \item 
\end{enumerate}