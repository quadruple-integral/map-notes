\documentclass{article}
\usepackage{amsmath}
\usepackage{amsfonts}
\usepackage{amssymb}
\usepackage{amsthm}
\usepackage{parskip}
\usepackage{multicol}
\usepackage{xcolor}
\usepackage{fancyhdr}
\usepackage{physics}
\usepackage{graphicx} % Required for inserting images
\usepackage{hyperref}
\usepackage{enumitem}
\newcommand{\matr}[1]{\mathbf{#1}}
\def\dbar{{\mathchar'26\mkern-12mu d}}

\newtheorem{definition}{Definition}[section]
\newtheorem{theorem}{Theorem}[section]
\newtheorem{corollary}{Corollary}[theorem]
\newtheorem{lemma}[theorem]{Lemma}
\newtheorem*{remark}{Remark}

\title{Honours Analysis}
\author{Notes by Christopher Shen}
\date{Winter 2023}

\begin{document}

\maketitle
\newpage

\tableofcontents
\newpage

\pagestyle{fancy}
\fancyhead{}
\fancyhead[L]{Honours Analysis}
\fancyhead[R]{Winter 2023}

\section{Real numbers}

\subsection{Properties of real numbers}

\subsection{Nested interval property and compactness}
    
\subsection{Triangle inequalities}

\section{Real sequences}

\section{Infinite series}
    
\section{Continuity and differentiability}

\newpage
    
\section{Pointwise and uniform convergence}
definition for pointwise and uniform convergence
    
uniform convergence supremum
    
limits and integration applications
    
weierstrass m test

uniform continuity - if $\delta$ is purely in $\epsilon$ form

\newpage
  
\section{Power series}

\newpage

\section{Lebesgue integration}

\subsection{Characteristic and step functions}
\begin{definition}[Characteristic functions]
    \hfill \\
    The characteristic function is defined as
    a real function such that:
    $$\chi_E(x) =
    \left\{
    \begin{array}{ll}
        1  & \mbox{} x \in E \\
        0 & \mbox{otherwise}
    \end{array}
    \right.$$
    where $E\subset\mathbb{R}$.
\end{definition}

\begin{remark}
    Therefore the integral of a characteristic function is:
    $$\int\chi_{E} = \lambda(E)$$
    for $\lambda(E)$ is the length of an internal $E$.
\end{remark}

\begin{definition}[Step functions]
    \hfill \\
    The step function with respect to
    finite set $\{x_0, \dots, x_n\}$
    for some $n \in \mathbb{N}$ is:
    $$\phi(x) =
    \left\{
    \begin{array}{ll}
	0  &  x < x_0 \hspace{0.06in} \text{or} \hspace{0.06in} x > x_n \\
	c_j & \mbox{if} \hspace{0.06in} x \in (x_{j-1}, x_j); \hspace{0.06in} 1 \leq j \leq n.
    \end{array}
    \right.$$
\end{definition}

\begin{remark}
    Step functions are a sum of characteristic functions:
    $$\phi(x)
    =\sum_{j=1}^{n} c_j \chi_{(x_{j-1}, x_j)}(x)$$
    and its integral is
    $$\int\phi=\sum_{j=1}^{n} c_j (x_{j-1}-x_j).$$
\end{remark}

Importantly the \underline{sum} of two step functions is another step function.
    
\newpage

\subsection{Lebesgue integrals}
Consider function $f:I \rightarrow \mathbb{R}$. This function is \textbf{Lebesgue integrable} on our interval I if:
\begin{enumerate}
    \item $\displaystyle\sum_{j=1}^{\infty} |c_j| \lambda(J_j) < \infty$

    \item $\forall x\in I; f(x)=\displaystyle\sum_{j=1}^{\infty} |c_j| \chi_{J_j}(x)<\infty$
\end{enumerate}
Here $c_j \in \mathbb{R}$, $J_i \subset I$ and is bounded for $j \in \{1, 2, 3, \dots\}$.

i.e. that our function's area and height are defined. Therefore:
$$\int_I f = \sum_{j=1}^{\infty} |c_j| \lambda(J_j)$$
and integral value is \underline{invariant of interval type}. (open, semi-open or closed)

\subsubsection{Properties of Lebesgue integrals}
Let functions $f$, $g$ be Lebesgue integrable on I and $\alpha, \beta\in\mathbb{R}$. Then:
\begin{enumerate}
    \item $\alpha f+\beta g$ is Lebesgue integrable on I, and:
    $$\int_I\alpha f+\beta g=\alpha\int_I f + \beta\int_I g.$$

    \item If $f\geq g$ on $I$ then:
    $$\int_I f\geq\int_I g.$$

    \item $$\int_I|f|\geq|\int_I f|$$

    \item $\max\{f,g\}$ and $\min\{f,g\}$ are integrable on $I$. Furthermore:
    $$\max\{f,g\}=\frac{f+g}{2}+\frac{|f-g|}{2}$$
    and
    $$\min\{f,g\}=\frac{f+g}{2}-\frac{|f-g|}{2}.$$

    \item $fg$ is integrable on $I$ if one of the functions is \underline{bounded}.

    \item Let $f\geq0$ where $\displaystyle\int_I f=0$.
    
    The function $h$ is integrable on $I$ if $0\leq h\leq f$.
\end{enumerate}

\newpage

\subsubsection{Integration on subintervals}
Let $J\subset I$. We then have the following statements.
\begin{enumerate}
    \item If $f$ is integrable on $I$ then $f$ is integrable on $J$.

    \item Let $f(x)=0$ for $\forall x\in I\setminus J$ and $f$ integrable on $J$. Then:
    $$\int_J f=\int_I f.$$

    \item Assume that $\forall x\in I; f(x)\geq0$. If $f$ is integrable on $I$ then:
    $$\int_I f\geq\int_J f.$$

    \item Let $I=\displaystyle\bigcup_{n=1}^{\infty}I_n$ where $I_n$ are all disjoint sets.
    
    Let $f$ be integrable on each $I_n$. We have that:
    $$\text{$f$ is integrable on $I$}\iff\sum_{n=1}^{\infty}\int_{I_n}f$$
    and that the following equality holds:
    $$\int_I f=\sum_{n=1}^{\infty}\int_{I_n}f.$$
\end{enumerate}
The regular integral calculus properties hold:
\begin{enumerate}
    \item $$\int_{a}^{b}f(x)dx=-\int_{b}^{a}f(x)dx$$

    \item $$\int_{a}^{b}f(x)dx+\int_{b}^{c}f(x)dx=\int_{a}^{c}f(x)dx$$
\end{enumerate}

\subsubsection{Maclaurin-Cauchy integral test}
Now let $f$ be a \underline{non-negative}, \textbf{monotone decreasing} function on $[p,\infty)$. Then:
$$\sum_{n=p}^{\infty}f(n)<\infty\iff\text{$f$ is integrable on $[p,\infty)$}$$
where $p\in\mathbb{Z}$. Furthermore:
$$\sum_{n=p}^{\infty}f(n)<\infty\iff\int_{p}^{\infty}f(x)dx<\infty.$$

\newpage

\subsection{Riemann integrals}
A real function $f$ is \textbf{Riemann-integrable} if it has \underline{bounded support}. i.e:
$$\forall\epsilon>0; \exists\phi,\psi:\phi\leq f\leq\psi\hspace{0.1in}\text{\underline{and}}
\hspace{0.1in}\int\psi-\int\phi<\epsilon,$$
where $\psi$ and $\phi$ are step functions.

Furthermore the following statements are equivalent:
\begin{enumerate}
    \item $f$ is Riemann-integrable, where $f$ is a real bounded function with bounded support $[a,b]$.

    \item $\sup\biggl\{\displaystyle\int\phi\biggl\}=\inf\biggl\{\displaystyle\int\psi\biggl\},
    \hspace{0.05in}\text{and is the integral value.}$

    \item $\forall\epsilon>0;\exists\{a=x_0<\dots<x_n=b\}:$

    $$\sum_{j=1}^{n}\left(\sup_{x\in(x_{j-1}-x_j)}f(x)-\inf_{x\in(x_{j-1}-x_j)}f(x)\right)(x_j-x_{j-1})
    <\epsilon$$
    and
    $$\sum_{j=1}^{n}\sup_{x,y\in I_j}|f(x)-f(y)|\cdot\lambda(I_j)<\epsilon$$
    where we define $I_j=(x_{j-1},x_j)$ and $j\in\{1,\dots,n\}$.
    
    Now let:
    $$m_j=\inf_{x\in I_j} f(x)$$
    $$M_j=\sup_{x\in I_j} f(x)$$
    and it makes sense to define step functions
    $$\phi_{*}\leq f\leq\phi^*(x)$$
    with respect to $\{x_0,\dots,x_n\}$ where:
    $$\phi_{*}(x)=\sum_{j=1}^{n}m_j\chi_{I_j}(x)
    +\sum_{j=1}^{n}f(x_j)\chi_{\{x_j\}}(x)$$
    and
    $$\phi^*(x)=\sum_{j=1}^{n}M_j\chi_{I_j}(x)
    +\sum_{j=1}^{n}f(x_j)\chi_{\{x_j\}}(x).$$
\end{enumerate}

If $f$ is Riemann-integrable then it is automatically Lebesgue-integrable, but not necessarily the opposite way.
So Lebesgue-integrals are a \underline{superset} of Riemann-integrals.

\newpage

Note that \underline{closed} intervals are \textbf{uniformly continuous}.

Let $g:[a,b]\rightarrow\mathbb{R}$ and that:
$$f(x) =
\left\{
\begin{array}{ll}
	g(x)  &  x\in[a,b] \\
	0 & \mbox{otherwise.}
\end{array}
\right.$$
We then have that:
\begin{enumerate}
    \item If $g$ is continuous on $[a,b]$ then $f$ is Riemann-integrable.
    \item If $g$ is a \underline{montone} function then $f$ is Riemann-integrable.
\end{enumerate}

\subsection{Fundamental theorem of calculus}
Let $g:I\rightarrow\mathbb{R}$ be integrable on $I$ and that
$$G(x)=\int_{x_0}^{x}g(x)\dd x$$
for $\forall x\in I$ and \underline{fixed} $x_0\in I$.

If $g(x)$ is continuous at $x\in I$ then:
$$\dv{x} G(x)=g(x).$$
Furthermore if $G(x)$ and $g(x)$ are continuous on the interval $I$:
$$\int_{a}^{b}g(x)\dd x=G(b)-G(a)$$
for $\forall a,b\in I$.

\subsection{Integration of sequences}
Consider $(f_n)_{n\in\mathbb{N}}$ that are integrable on $I$. Assume the following:
\begin{itemize}
    \item $\displaystyle\sum_{n=1}^{\infty}
    \int_I |f_n|<\infty$
    \item $\displaystyle\sum_{n=1}^{\infty}
    |f_n(x)|<\infty$ for $\forall x\in I$.
\end{itemize}
Let $f(x)=\displaystyle\sum_{n=1}^{\infty}f_n(x)$.
Then $f$ is integrable on $I$ and
$$\int_I f=\sum_{n=1}^{\infty}\int_I f_n.$$

\newpage

The following result is a useful test for integrability.

Let $f_n\geq0$ on $I$ and that
$f(x)=\displaystyle\sum_{n=1}^{\infty}f_n(x)$. Then:
$$\text{$f$ is integrable on $I$}
\iff\sum_{n=1}^{\infty}\int_I f_n<\infty.$$

\subsubsection{Monotone convergence for integration}
Now consider a monotone increasing sequence of functions $(f_n)_{n\in\mathbb{N}}$:
$$f_1\leq f_2\leq f_3\leq\dots.$$
Let $f(x)=\displaystyle\lim_{n\rightarrow\infty}f_n(x)$. Then:
$$\int_I f=\lim_{n\rightarrow\infty}\int_I f_n.$$
and furthermore:
$$\sup_{n\in\mathbb{N}}\int_I f_n
=\lim_{n\rightarrow\infty}\int_I f_n <\infty.$$

\subsubsection{Fatoux's lemma}
Let $f_n>0$ be integrable functions on $I$ and that:
$$f(x)=\liminf_{n\rightarrow\infty} f_n(x)$$
for $\forall x\in I$. If
$$\liminf_{n\rightarrow\infty}\int_I f_n(x)<\infty$$
then $f$ is integrable on $I$ and:
$$\int_I f\leq
\liminf_{n\rightarrow\infty}\int_I f_n(x).$$
An immediate result is the following. 

Let $f_n$ be integrable on the interval $I$ and that:
$$f(x)=\lim_{n\rightarrow\infty}f_n(x).$$
If $|f_n(x)|\leq g(x)$ where $\displaystyle\int_I g<\infty$ then:
$$\int_I f=\lim_{n\rightarrow\infty}\int_I f_n.$$
A final result is that if $f_n:(a,b)\rightarrow\mathbb{R}$ are integrable functions, and that:
$$f_n\rightarrow f\hspace{0.05in}\text{uniformly on}\hspace{0.05in}(a,b),$$
we then have that:
$$\int_I f=\lim_{n\rightarrow\infty}\int_I f_n.$$

\newpage

\section{Fourier analysis}

\subsection{$L^2$ space}
l2 norm of a function

inner product

cauchy schwarz inequalities

minkowski inequalities

convergence in l2

orthonormal systems

T5.2

bessel's inequality

riemann lemma

complete orthonormal systems

T5.4

\subsection{Fourier series}

trigonometric polynomial (fs)

complex fourier series

fourier coefficients

euler formula

lemma 5.1: orthgonality of FS

convolution of fs

dirichlet kernel

\subsection{Convergence of Fourier series}

\subsubsection{Approximations}

\subsubsection{$L^2$ convergence}

\subsubsection{Pointwise convergence}

\end{document}
