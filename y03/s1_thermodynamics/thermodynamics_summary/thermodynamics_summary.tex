\documentclass{article}
\usepackage{geometry}
\usepackage{amsmath}
\usepackage{amsfonts}
\usepackage{amssymb}
\usepackage{amsthm}
\usepackage{parskip}
\usepackage{multicol}
\usepackage{xcolor}
\usepackage{fancyhdr}
\usepackage{physics}
\usepackage{graphicx} % Required for inserting images
\usepackage{hyperref}
\usepackage{enumitem}

% margin settings
\geometry{
    a4paper,
    left=7mm,
    right=7mm,
    top=2cm,
    bottom=7mm
}

% testing
\usepackage{blindtext}

% proof environments
\newtheorem{definition}{Definition}[section]
\newtheorem{theorem}{Theorem}[section]
\newtheorem{corollary}{Corollary}[theorem]
\newtheorem{lemma}[theorem]{Lemma}
\newtheorem*{remark}{Remark} % unnumbered remarks

% header and footer
\pagestyle{fancy}
\fancyfoot{} % removes footer
\fancyhf{}
\renewcommand{\headrulewidth}{0.5pt}
\fancyhead[L]{Thermodynamics}
\fancyhead[R]{\thepage}

\begin{document}

\begin{multicols*}{3}
\noindent

\subsubsection*{Definitions}

\textbf{Isolated system}: No exchanges

\textbf{Closed system}: Only energy exchange

\textbf{Open system}: Energy \& mass exchange

\textbf{Intensive state variables}: \\
Independent of mass

\textbf{Extensive state variables}: \\
Proportional to mass

\textbf{Reservoirs}: Infinite/very large system
that remains unchanged when in contact
with finite system.

\textbf{Mechanical equilibrium}: \\
No unbalanced forces

\textbf{Thermal equilibrium}: \\
No temperature differences

\textbf{Thermodynamic equilibrium}: \\
Intensive state variables of system are \underline{constant}.
Alternatively our system is in \underline{mechanical} and
\underline{thermal} equilibrium.

\textbf{Reversible processes}: \\
Every intermediate is an equilibrium state.

\textbf{Quasi-static processes}: \\
Process sufficiently slow such that only \\
infinitesimal temperature or pressure gradients exist.

Frictionless quasi-static processes are \underline{reversible}.

\textbf{Cyclic processes}:
$$\Delta U=0
\hspace{0.05in}\text{and}\hspace{0.05in}
W=Q$$
For conservative forces:
$$\oint\dd X=0$$
where $X$ is a state variable.

\textbf{Adiabatic processes}: $\Delta Q=0$

\textbf{Isothermal processes}: $\Delta T=0$

\subsubsection*{Zeroth law}
If $A$ is in thermal equilibrium with $B$ and $C$
\underline{seperately} then $B$ and $C$
are also in thermal equilibrium.

\subsubsection*{Ideal gas state equation}
Given $n$ moles of gas at temperature $T$:
$$PV=nRT$$
where $R=8.314$JK$^{-1}$mol$^{-1}$.

\subsubsection*{First law}
Total energy $E$ is conserved and:
$$\dd U=\dd Q-\dd W.$$
Note $Q>0$ represents energy transferred
\underline{into} system.
When system does work on surroundings $W>0$.

Work done \underline{by} fluid
in reversible \\
adiabatic processes:
$$\dd W=P\dd V$$
and is in Joules (J).

\subsubsection*{Isochoric heat capacity}
$$C_V(T)=\left(\frac{\dd Q}{\dd T}\right)_V
=\left(\frac{\partial U}{\partial T}\right)_V$$

\subsubsection*{Isobaric heat capacity}
\begin{align*}
    C_P
    &=\left(\frac{\dd Q}{\dd T}\right)_P \\
    &=C_V+\left[P+
    \left(\frac{\partial U}{\partial V}\right)_T\right]
    \left(\frac{\partial V}{\partial T}\right)_P
\end{align*}
Heat capacity has units JK$^{-1}$.

For ideal gases we have that:
$$C_P-C_V=nR.$$
Under \underline{reversible adiabatic} processes:
$$TV^{\gamma-1}=\text{constant}$$
$$PV^{\gamma}=\text{constant}$$
$$T^{\frac{1}{\gamma-1}}V=\text{constant}$$
where $\gamma$ is the \underline{adiabatic exponent}:
$$U=\frac{f}{2}nRT$$
$$\gamma=\frac{C_P}{C_V}
=\frac{f+2}{f}$$
regardless of degrees of freedom $f$.

\subsubsection*{State function enthalpy}
$$H=U+PV$$
\begin{align*}
    \dd H
    &=\dd U+V\dd P+P\dd V \\
    &=\dd Q+V\dd P
\end{align*}
$$\therefore C_P
=\left(\frac{\partial H}{\partial T}\right)_P$$

\subsubsection*{Chemical reactions}
$$Q=\Delta U+P_0\Delta V=\Delta H$$
Here $P_0$ is constant.
\begin{itemize}
    \item $Q<0$: exothermic \\
    (heat is released)
    \item $Q>0$: endothermic \\
    (heat is absorbed)
\end{itemize}

\subsubsection*{Carnot's theorem}
Peak efficiency of a \underline{cyclic}
heat engine:
$$\eta=\frac{\dot{W}}{\dot{Q_H}}=1-\frac{Q_C}{Q_H}
=1-\frac{T_C}{T_H}$$
and is either in terms of rate or value, 
with units J or Js$^{-1}$.

\subsubsection*{State function entropy}
$$S=\frac{Q}{T}$$
$$\therefore\dd U=T\dd S-P\dd V$$

\subsubsection*{Entropy of mixing}
$$\Delta S=n_A R\ln\frac{V_A+V_B}{V_A}
+n_B R\ln\frac{V_A+V_B}{V_B}$$
$$\Delta s_{mix}=-R(x_A\ln x_A+x_B\ln x_B)$$
$$x_A=\frac{n_A}{n_A+n_B}
\hspace{0.05in}\text{and}\hspace{0.05in}
x_B=\frac{n_B}{n_A+n_B}$$

\subsubsection*{Second law}
$$\Delta S_{total}=\Delta_{system}
+\Delta_{reservoir}\geq0$$
$$\dd S\geq\frac{\dd Q}{T}$$

\subsubsection*{Helmholtz free energy}
$$F=U-TS$$
$$\dd F=-S\dd T-P\dd V$$

\subsubsection*{Gibbs free energy}
$$G=H-TS$$
$$\dd G=-S\dd T+V\dd P$$
Chemical reactions are spontaneous if:
$$\Delta G=\Delta H-T\Delta S<0.$$

\subsubsection*{Maxwell relations}
$$\left(\frac{\partial T}{\partial V}\right)_S=-\left(\frac{\partial P}{\partial S}\right)_V$$
$$\left(\frac{\partial T}{\partial P}\right)_S=\left(\frac{\partial V}{\partial S}\right)_P$$
$$\left(\frac{\partial S}{\partial V}\right)_T=\left(\frac{\partial P}{\partial T}\right)_V$$
$$-\left(\frac{\partial S}{\partial P}\right)_T=\left(\frac{\partial V}{\partial T}\right)_P$$
The isobaric expansivity is defined as:
$$\beta=\frac{1}{V}
\left(\frac{\partial V}{\partial T}\right)_P$$
and the isothermal compressibility:
$$\kappa_T=-\frac{1}{V}
\left(\frac{\partial V}{\partial P}\right)_T.$$

\newpage

\subsubsection*{Clausius-Clapeyron equation}
The slope of any phase boundaries is:
$$\dv{P}{T}=\frac{\Delta S}{\Delta V}.$$

\subsubsection*{Van der Waals state equation}
$$\left(P+\frac{an^2}{V^2}\right)
\bigl(V-nb\bigr)=nRT$$

\subsubsection*{Chemical potentials}
$$\mu=\frac{G}{N}$$

\subsubsection*{Third law}
$S=0$ at $T=0$K.

\end{multicols*}

\end{document}