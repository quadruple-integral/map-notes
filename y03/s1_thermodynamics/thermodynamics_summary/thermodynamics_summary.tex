\documentclass{article}
\usepackage[margin=0.3in]{geometry}
\usepackage{amsmath}
\usepackage{amsfonts}
\usepackage{amssymb}
\usepackage{amsthm}
\usepackage{parskip}
\usepackage{multicol}
\usepackage{xcolor}
\usepackage{fancyhdr}
\usepackage{physics}
\usepackage{graphicx} % Required for inserting images
\usepackage{hyperref}
\usepackage{enumitem}
\newcommand{\matr}[1]{\mathbf{#1}}
\def\dbar{{\mathchar'26\mkern-12mu d}}

\newtheorem{definition}{Definition}[section]
\newtheorem{theorem}{Theorem}[section]
\newtheorem{corollary}{Corollary}[theorem]
\newtheorem{lemma}[theorem]{Lemma}

\pagestyle{fancy}
\fancyhf{}
\renewcommand{\headrulewidth}{1pt}
\fancyhead[R]{\thepage}

\begin{document}

\begin{multicols*}{3}
\noindent

\subsubsection*{Definitions}

\textbf{Isolated system}: No exchanges

\textbf{Closed system}: Only energy exchange

\textbf{Open system}: Energy \& mass exchange

\textbf{Intensive state variables}: \\
Independent of mass

\textbf{Extensive state variables}: \\
Proportional to mass

\textbf{Reservoirs}: Infinite/very large system
that remains unchanged when in contact
with finite system.

\textbf{Mechanical equilibrium}: \\
No unbalanced forces

\textbf{Thermal equilibrium}: \\
No temperature differences

\textbf{Thermodynamic equilibrium}: \\
Intensive state variables of system are \underline{constant}.
Alternatively our system is in \underline{mechanical} and
\underline{thermal} equilibrium.

\textbf{Reversible processes}: \\
Every intermediate is an equilibrium state.

\textbf{Quasi-static processes}: \\
Process sufficiently slow such that only \\
infinitesimal temperature or pressure gradients exist.

Frictionless quasi-static processes are \underline{reversible}.

\textbf{Cyclic processes}:
$$\oint\dd X=0$$
where $X$ is a state variable.

\subsubsection*{Zeroth law}
If $A$ is in thermal equilibrium with $B$ and $C$
\underline{seperately} then $B$ and $C$
are also in thermal equilibrium.

\subsubsection*{Ideal gas state equation}
Given $n$ moles of gas at temperature $T$:
$$PV=nRT$$
where $R=8.314$JK$^{-1}$mol$^{-1}$.

\subsubsection*{First law}
$$\dd U=\dd Q-\dd W$$
Note $Q>0$ represents energy transferred
\underline{into} system.
When system does work on surroundings $W>0$.

Work done \underline{by} fluid
in reversible process:
$$\dd W=P\dd V.$$

\subsubsection*{Isochoric heat capacity}
$$C_V(T)=\left(\frac{\dd Q}{\dd T}\right)_V
=\left(\frac{\partial U}{\partial T}\right)_V$$

\subsubsection*{Isobaric heat capacity}
\begin{align*}
    C_P
    &=\left(\frac{\dd Q}{\dd T}\right)_P \\
    &=C_V+\left[P+
    \left(\frac{\partial U}{\partial V}\right)_T\right]
    \left(\frac{\partial V}{\partial T}\right)_P
\end{align*}
Heat capacity has units JK$^{-1}$.

For ideal gases:
$$C_P-C_V=nR$$
$$TV^{\gamma-1}=\text{constant}$$
$$\gamma=\frac{C_P}{C_V}.$$

\subsubsection*{State function enthalpy}
$$H=U+PV$$
\begin{align*}
    \dd H
    &=\dd U+V\dd P+P\dd V \\
    &=\dd Q+V\dd P
\end{align*}
$$\therefore C_P
=\left(\frac{\partial H}{\partial T}\right)_P$$

\subsubsection*{Chemical reactions}
$$Q=\Delta U+P_0\Delta V=\Delta H$$
Here $P_0$ is constant.
\begin{itemize}
    \item $Q<0$: exothermic \\
    (heat is released)
    \item $Q>0$: endothermic \\
    (heat is absorbed)
\end{itemize}

\subsubsection*{Carnot's theorem}
Peak efficiency of a \underline{cyclic}
heat engine:
$$\eta=1-\frac{Q_C}{Q_H}.$$

\subsubsection*{State function entropy}
$$S=\frac{Q}{T}$$
For reversible cyclic heat engine:
$$\dd S=\frac{\dd Q}{T}.$$
$$\therefore\dd U=T\dd S-P\dd V$$



\end{multicols*}

\end{document}