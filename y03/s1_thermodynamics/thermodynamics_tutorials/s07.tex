\pagestyle{fancy}
\fancyhead{}
\fancyhead[L]{Thermodynamics Tutorial 7}
\fancyhead[R]{Winter 2023}

\section{Set 7}
\begin{enumerate}
    \item Indexed as number $51$ in tutorial sheet 7. \\
    
    Consider the entropy change for an \underline{ideal} gas:
    \begin{align*}
        \Delta S(V,T)
        &=S(V,T)-S(V_0,T_0) \\
        &=C_V\ln\frac{T}{T_0}+nR\ln\frac{V}{V_0}.
    \end{align*}
    For part ($a$)($1$) write this expression as:
    \begin{align*}
        \Delta S(P,T)
        &=C_P\ln\frac{T}{T_0}-nR\ln\frac{P}{P_0}.
    \end{align*}
    It is important for us to recall the difference in heat capacity:
    $$C_P-C_V=nR.$$
    Using this and the ideal state equation $PV=nRT$:
    \begin{align*}
        \Delta S
        &=(C_P-nR)\ln\frac{T}{T_0}+nR\ln\frac{V}{V_0} \\
        &=C_P\ln\frac{T}{T_0}+nR\ln\left(
            \frac{V}{V_0}\cdot\frac{T_0}{T}
        \right) \\
        &=C_P\ln\frac{T}{T_0}-nR\ln\frac{P}{P_0}
    \end{align*}
    since the ideal gas equation holds for fixed physical quantities:
    $$P_0 V_0=nRT_0.$$

    For part ($a$)($2$) write this expression as:
    \begin{align*}
        \Delta S(P,V)
        &=C_P\ln\frac{V}{V_0}+C_V\ln\frac{P}{P_0}.
    \end{align*}
    Now we start again with our original equation and
    use $C_P-C_V=nR$.
    \begin{align*}
        \therefore\Delta S
        &=C_V\ln\frac{T}{T_0}+nR\ln\frac{V}{V_0} \\
        &=C_V\ln\frac{T}{T_0}+(C_P-C_V)\ln\frac{V}{V_0} \\
        &=C_P\ln\frac{V}{V_0}+C_V\ln\frac{P}{P_0}
    \end{align*}
    The last step we used the ideal gas equation.

    \newpage

    For part ($b$) we want to verify:
    $$-\left(\frac{\partial S}{\partial P}\right)_T
    =\left(\frac{\partial V}{\partial T}\right)_P$$
    given the following assumptions:
    \begin{itemize}
        \item $\Delta S(P,T)=C_P\ln\frac{T}{T_0}-nR\ln\frac{P}{P_0}$
        \item $PV=nRT$.
    \end{itemize}
    The right hand side of our equation is:
    $$\left(\frac{\partial V}{\partial T}\right)_P=\frac{nR}{P}.$$
    Now for the left hand side. Firstly we need to come up with
    an expression for entropy, and then take its partial derivatives to show equality.
    $$\because\Delta S(P,T)=S(P,T)-S_0$$
    $$\therefore S(P,T)
    =C_P\ln\frac{T}{T_0}-nR\ln\frac{P}{P_0}+S_0$$
    $$\therefore\frac{\partial}{\partial P}S(P,T)=-\frac{nR}{P}$$
    And clearly we have equality of both sides. \\

    For part ($c$) show that a reversible adiabatic process implies
    an isentropic process:
    $$PV^{\gamma}=\text{constant}\implies\Delta S=0$$
    where $\gamma=\displaystyle\frac{C_P}{C_V}$.

    Beginning with our derived expression $\Delta S(P,V)$:
    \begin{align*}
        \Delta S(P,V)
        &=C_P\ln\frac{V}{V_0}+C_V\ln\frac{P}{P_0} \\
        &=C_P\left(\ln\frac{V}{V_0}+\frac{1}{\gamma}\ln\frac{P}{P_0}\right).
    \end{align*}
    Now since we have that:
    $$\frac{P}{P_0}=\left(\frac{V_0}{V}\right)^{\gamma}$$
    it is clear that $\Delta S=0$.

    \newpage

    \item Indexed as number $52$ in tutorial sheet 7. \\
    
    For part ($a$) show that:
    $$\dd S=\frac{C_P}{T}\dd T -V\beta\dd P.$$ \\

    We take total differentials of the previously derived $S(P,T)$:
    $$\dd S=\left(\frac{\partial S}{\partial T}\right)_P\dd T
    +\left(\frac{\partial S}{\partial P}\right)_T\dd P.$$
    Since we have that:
    $$S(P,T)
    =C_P\ln\frac{T}{T_0}-nR\ln\frac{P}{P_0}+S_0$$
    therefore:
    $$\left(\frac{\partial S}{\partial T}\right)_P
    =\frac{C_P}{T}.$$
    The second partial derivative we recognise as a Maxwell relation:
    \begin{align*}
        \left(\frac{\partial S}{\partial P}\right)_T
        &=-\left(\frac{\partial V}{\partial T}\right)_P \\
        &=-V\beta
    \end{align*}
    where $\beta$ is the \underline{isobaric expansivity}.
    Therefore we have that:
    $$\dd S=\frac{C_P}{T}\dd T -V\beta\dd P.$$ \\

    For part ($b$), consider metal box subject to \underline{adiabatic} and \underline{reversible} increase in pressure. ($P_1\rightarrow P_2$)

    Show that its temperature change ($T_1\rightarrow T_2$)
    satisfies the following:
    $$\ln\frac{T_2}{T_1}=\frac{V\beta(P_2-P_1)}{C_P}.$$ \\

    Since $\Delta Q=0$ and our process is reversible the overall entropy change is \underline{zero}. Integrate result in previous part to obtain answer.

    \newpage

    Integrating from initial state to final state:
    \begin{align*}
        \Delta S
        &=\int_{(1)}^{(2)}\dd S \\
        &=\int_{(1)}^{(2)}
        \left(\frac{C_P}{T}\dd T -V\beta\dd P\right) \\
        &=0.
    \end{align*}
    We then have that:
    $$\int_{T_1}^{T_2}\frac{C_P}{T}\dd T
    -\int_{P_1}^{P_2}V\beta\dd P=0$$
    which gives:
    $$\ln\frac{T_2}{T_1}=\frac{V\beta(P_2-P_1)}{C_P}.$$

    \newpage

    \item Indexed as number $53$ in tutorial sheet 7. \\

    For part ($a$) show that:
    $$\mu_{JK}=\left(\frac{\partial T}{\partial P}\right)_H
    =\frac{V}{C_P}(\beta T-1).$$ \\

    We begin by using the cyclic relations of partial derivatives:
    $$\left(\frac{\partial T}{\partial P}\right)_H
    \left(\frac{\partial H}{\partial T}\right)_P
    \left(\frac{\partial P}{\partial H}\right)_T=-1$$
    and hence:
    \begin{align*}
        \left(\frac{\partial T}{\partial P}\right)_H
        &=-\left(\frac{\partial H}{\partial T}\right)_P^{-1}
        \left(\frac{\partial P}{\partial H}\right)_T^{-1} \\
        &=-\left(\frac{\partial H}{\partial T}\right)_P^{-1}
        \left(\frac{\partial H}{\partial P}\right)_T.
    \end{align*}
    The first one is by definition $C_P$:
    $$\left(\frac{\partial H}{\partial T}\right)_P
    =C_P.$$
    Since we have that:
    $$\dd H=T\dd S+V\dd P$$
    dividing through by $\dd P$ at fixed $T$ gives:
    $$\left(\frac{\partial H}{\partial P}\right)_T
    =T\left(\frac{\partial S}{\partial P}\right)_T+V$$
    and hence for the second partial derivative:
    \begin{align*}
        \left(\frac{\partial H}{\partial P}\right)_T
        &=-T\left(\frac{\partial V}{\partial T}\right)_P+V
    \end{align*}
    via a Maxwell relation. \\

    For part ($b$) since $V=\displaystyle\frac{nRT}{P}$ and using the definition of isobaric expansivity we get that $\mu_{JK}=0$.

    \newpage

    \item Indexed as number $54$ in tutorial sheet 7. \\
    
    For part ($a$) show that:
    $$\frac{\kappa_T}{\kappa_S}=\frac{C_P}{C_V}$$
    where we define:
    $$\kappa_T=-\frac{1}{V}
    \left(\frac{\partial V}{\partial P}\right)_T$$
    and
    $$\kappa_S=-\frac{1}{V}
    \left(\frac{\partial V}{\partial P}\right)_S.$$ \\

    Firstly using the definition we have that:
    $$\frac{\kappa_T}{\kappa_S}
    =\left(\frac{\partial V}{\partial P}\right)_T
    \left(\frac{\partial V}{\partial P}\right)_S^{-1}.$$
    Note the following cyclic relations:
    $$\left(\frac{\partial V}{\partial P}\right)_T
    \left(\frac{\partial T}{\partial V}\right)_P
    \left(\frac{\partial P}{\partial T}\right)_V=-1$$
    and
    $$\left(\frac{\partial V}{\partial P}\right)_S
    \left(\frac{\partial S}{\partial V}\right)_P
    \left(\frac{\partial P}{\partial S}\right)_V=-1.$$
    Then our first expression becomes:
    \begin{align*}
        \frac{\kappa_T}{\kappa_S}
        &=\left(\frac{\partial T}{\partial V}\right)_P^{-1}
        \left(\frac{\partial P}{\partial T}\right)_V^{-1}
        \left(\frac{\partial S}{\partial V}\right)_P
        \left(\frac{\partial P}{\partial S}\right)_V \\
        &=\left(\frac{\partial S}{\partial V}\right)_P
        \left(\frac{\partial V}{\partial T}\right)_P
        \left[\left(\frac{\partial S}{\partial P}\right)_V
        \left(\frac{\partial P}{\partial T}\right)_V
        \right]^{-1} \\
        &=T\left(\frac{\partial S}{\partial T}\right)_P
        \left[T\left(\frac{\partial S}{\partial T}\right)_V
        \right]^{-1} \\
        &=\frac{C_P}{C_V}.
    \end{align*}

    \newpage

    For part ($b$) show that:
    $$\kappa_T-\kappa_S=\frac{VT\beta^2}{C_P}.$$ \\

    Using previous results:
    $$C_P-C_V=\frac{VT\beta^2}{\kappa_T}$$
    and
    $$C_V=C_P\frac{\kappa_S}{\kappa_T}$$
    the result follows after substituting in $C_V$. \\

    For part ($c$) find $\kappa_S$ for an ideal gas
    and verify part ($b$). \\

    Let $\gamma=\displaystyle\frac{C_P}{C_V}$.
    We then have the following relation:
    $$\kappa_S=\frac{\kappa_T}{\gamma}.$$
    Now since $PV=nRT$:
    \begin{align*}
        \kappa_T
        &=-\frac{1}{V}
        \left(\frac{\partial V}{\partial P}\right)_T \\
        &=\frac{1}{P}
    \end{align*}
    and so:
    $$\kappa_S=\frac{1}{\gamma P}$$
    where $\gamma$ is our \underline{adiabatic exponent}. To verify part ($b$):
    \begin{align*}
        \kappa_T-\kappa_S
        &=\frac{1}{P}-\frac{1}{\gamma P} \\
        &=\frac{1}{P}\left(\frac{C_P-C_V}{C_P}\right) \\
        &=\frac{VT\beta^2}{C_P}
    \end{align*}
    via the difference in heat capacities.

    \newpage

    \item Consider the following:
    $$C_P-C_V=\frac{VT\beta^2}{\kappa_T}>0$$
    since $\beta>0$ and $\kappa_T>0$. Explain \underline{physically}
    why $\kappa_T>0$ must be true. \\

    Firstly recall the definition of $\kappa_T$:
    $$\kappa_T=-\frac{1}{V}
    \left(\frac{\partial V}{\partial P}\right)_T.$$
    Physically a \underline{compression} results in a reduction of volume:
    $$+\Delta P\implies-\Delta V$$
    and that vice versa:
    $$-\Delta P\implies+\Delta V.$$
    If we think of partial derivatives as infinitesimal fractions
    of change then it is clear that $\kappa_T$ is always positive.

    We can also contradict ourself by assuming:
    $$-\Delta P\implies-\Delta V$$
    or that a reduction in pressure results in compression
    of solid. This eventually forms a black hole which is impossible.
    (for 'normal' matter)

    \newpage

    \item For part ($a$) express $\mu_{JK}$
    using the Van der Waals state equation. \\

    So we have that:
    $$\mu_{JK}=\frac{V}{C_P}(\beta T-1)$$
    and the \underline{molar} Van der Waals state equation:
    $$\left(P+\frac{a}{v^2}\right)(v-b)=RT$$
    which states that at pressure $P$
    one mole of gas with volume $v$
    has temperature $T$.
    Now everything is all well and good except when finding
    $\beta$ which is defined as follows:
    $$\beta=\frac{1}{V}
    \left(\frac{\partial V}{\partial T}\right)_P
    =\frac{1}{v}
    \left(\frac{\partial v}{\partial T}\right)_P$$
    and is known as the isobaric expansivity constant.

    But we can use the inverse relations for
    partial derivatives:
    $$\left(\frac{\partial T}{\partial v}\right)_P
    =\frac{1}{Rv^3}\left(Pv^3-av+2ab\right)$$
    and so
    $$\beta=Rv^2\left(Pv^3-av+2ab\right)^{-1}.$$
    After some rearranging we get:
    $$\beta T=RTv^{-1}
    \Bigl[\frac{RT}{v-b}-\frac{2a(v-b)}{v^3}\Bigr]^{-1}$$
    and therefore our \underline{Joule-Kelvin} coefficient is:
    $$\mu_{JK}=\frac{v}{c_P}
    \left(RTv^{-1}\Bigl[\frac{RT}{v-b}
    -\frac{2a(v-b)}{v^3}\Bigr]^{-1}-1\right).$$

    \newpage

    For part ($b$) find temperature $T$
    such that $\mu_{JK}=0$. \\

    Setting the previous result to zero implies:
    $$RTv^{-1}\Bigl[\frac{RT}{v-b}
    -\frac{2a(v-b)}{v^3}\Bigr]^{-1}-1=0$$
    and solving for $T$ gives:
    $$T=\frac{2a}{bR}\left(\frac{v-b}{v}\right)^2.$$
    Now when $b=0$ our $T$ is maximised:
    $$T_{max}=\frac{2a}{bR}$$
    which is part ($c$).
\end{enumerate}