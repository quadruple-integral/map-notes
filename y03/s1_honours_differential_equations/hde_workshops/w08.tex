\pagestyle{fancy}
\fancyhead{}
\fancyhead[L]{Honours DE Workshop 8}
\fancyhead[R]{Winter 2023}

\begin{enumerate}
    \item For part ($a$) we want to use the method of separation of variables
    to find ODE solutions to:
    \begin{itemize}
        \item $\displaystyle{\frac{\partial^2 u}{\partial x^2}
        +\frac{\partial^2 u}{\partial x\partial t}+\frac{\partial u}{\partial t}=0}$

        \item $\displaystyle{\frac{\partial^2 u}{\partial x^2}
        +(x+y)\frac{\partial^2 u}{\partial y^2}=0}$
    \end{itemize}
    \hspace{0.1in} \\
    
    For the first PDE let
    $$u=X(x)T(t)$$
    and after differentiating we get:
    $$X''T+X'\dot{T}+X\dot{T}=0.$$
    Rearranging this equation and introducing a separation constant:
    $$\frac{X''}{X'+X}=-\frac{\dot{T}}{T}=\lambda.$$
    This is clearly two ODEs:
    $$X''-\lambda(X'+X)=0$$
    $$\dot{T}+\lambda T=0.$$ \\
    
    For the second equation let
    $$u=X(x)Y(y)$$
    and differentiating this gives
    $$X''Y+(x+y)XY''=0$$
    which is of non-separable form.

    \newpage

    For part ($b$)($i$) consider the heat equation:
    $$\frac{\partial u}{\partial t}=\alpha^2
    \left(\frac{\partial^2 u}{\partial x^2}+\frac{\partial^2 u}{\partial y^2}\right).$$
    We solve this problem via separation of variables:
    $$u=X(x)Y(y)T(t).$$
    Differentiating and substituting:
    $$XY\dot{T}=\alpha^2\left(X''YT+XY''T\right)$$
    and dividing both side by $XYT$ gives:
    $$\frac{1}{\alpha^2}\frac{\dot{T}}{T}
    =\frac{X''}{X}+\frac{Y''}{Y}=\lambda.$$
    We get our first ODE:
    $$\dot{T}-\lambda\alpha^2 T=0.$$
    Clearly we need another separation constant:
    $$\frac{X''}{X}=\lambda-\frac{Y''}{Y}=\mu$$
    and this yields the other two ODEs:
    $$X''-\mu X=0$$
    $$Y''+(\mu-\lambda)Y=0.$$

    \newpage

    For part ($b$)($ii$) reconsider the heat equation
    in polar coordinates:
    $$\frac{\partial u}{\partial t}=\alpha^2\left(\frac{\partial^2 u}{\partial r^2}
    +\frac{1}{r}\frac{\partial u}{\partial r}
    +\frac{1}{r^2}\frac{\partial^2 u}{\partial \phi^2}\right).$$
    We again use separation of variables:
    $$u=R(r)\Phi(\phi)T(t)$$
    and after differentiating we get:
    $$R\Phi\dot{T}=\alpha^2\left(R''\Phi T+\frac{1}{r}R'\Phi T
    +\frac{1}{r^2}R\Phi''T\right).$$
    Dividing through by $R\Phi T$ and setting a separation constant:
    $$\frac{1}{\alpha^2}\frac{\dot{T}}{T}=\frac{R''}{R}+\frac{1}{r}\frac{R'}{R}
    +\frac{1}{r^2}\frac{\Phi''}{\Phi}=\lambda$$
    and we obtain our first ODE:
    $$\dot{T}-\lambda\alpha^2 T=0.$$
    Multiplying both sides by $r^2$ gives:
    $$r^2\frac{R''}{R}+r\frac{R'}{R}
    +\frac{\Phi''}{\Phi}=r^2\lambda.$$
    Rearranging and introducing another separation constant:
    $$r^2\frac{R''}{R}+r\frac{R'}{R}-r^2\lambda
    =-\frac{\Phi''}{\Phi}=\mu$$
    and clearly we have two ODEs.
    $$\therefore\Phi''+\mu\Phi=0$$
    $$\therefore r^2 R''+rR'
    -(\mu+\lambda r^2)R=0$$

    \newpage

    \item The solution to the heat equation is:
    $$u(x,t)=\frac{c_0}{2}+\sum_{n=1}^{\infty}c_n
    \exp\left(-n^2\alpha^2\pi^2 t/L^2\right)\cos\frac{n\pi x}{L}$$
    where we have initial condition $u(x,0)=f(x)$ and boundary condition:
    $$\frac{\partial}{\partial x}u(0,t)=\frac{\partial}{\partial x}u(L,t)=0.$$ \\

    Firstly set $t=0$. We then have that:
    $$u(x,0)=\frac{c_0}{2}+\sum_{n=1}^{\infty}\cos\frac{n\pi x}{L}=f(x)$$
    for $\forall x\in[0,L]$. Taking the integral over this interval:
    $$\int_{0}^{L}f(x)\dd x=\int_{0}^{L}\frac{c_0}{2}\dd x=\frac{L}{2}c_0.$$
    $$\therefore c_0=\frac{2}{L}\int_{0}^{L}f(x)\dd x$$
    Similarly we have that:
    $$c_n=\frac{2}{L}\int_{0}^{L}\cos\frac{n\pi x}{L}f(x)\dd x.$$
    Now if
    $$f(x)=3\cos\frac{2\pi x}{L}$$
    then:
    \begin{align*}
        c_0
        &=\frac{2}{L}\int_{0}^{L}3\cos\frac{2\pi x}{L}\dd x \\
        &=\frac{6}{L}\left[\frac{L}{2\pi}
        \sin\frac{2\pi x}{L}\right]_{0}^{L} \\
        &=0.
    \end{align*}
    To find $c_n$ we need the following identity:
    $$\cos A\cos B=\frac{1}{2}\Bigl(\cos(A-B)+\cos(A+B)\Bigl).$$

    \newpage
    So \textbf{if} \underline{$n\neq2$}:
    \begin{align*}
        c_n
        &=\frac{2}{L}\int_{0}^{L}\cos\frac{n\pi x}{L}\cdot 3\cos\frac{2\pi x}{L}\dd x \\
        &=\frac{6}{L}\int_{0}^{L}\cos\frac{n\pi x}{L}\cos\frac{2\pi x}{L}\dd x \\
        &=\frac{6}{L}\int_{0}^{L}\frac{1}{2}\Bigl[\cos(2-n)\frac{\pi x}{L}
        +\cos(n+2)\frac{\pi x}{L}\Bigl]\dd x \\
        &=\frac{3}{\pi}\Bigl[\frac{1}{2-n}\sin\frac{(2-n)\pi x}{L}
        +\frac{1}{n+2}\sin\frac{(n+2)\pi x}{L}\Bigl]_{0}^{L} \\
        &=0.
    \end{align*}
    \begin{align*}
        \therefore c_2
        &=\frac{6}{L}
        \int_{0}^{L}\left(\cos\frac{2\pi x}{L}\right)^2\dd x \\
        &=\frac{6}{L}
        \int_{0}^{L}\frac{1}{2}
        \left(1+\cos\frac{4\pi x}{L}\right)\dd x \\
        &=\frac{3}{L}\left[x+\frac{L}{4\pi}\sin\frac{4\pi x}{L}\right]_{0}^{L} \\
        &=3
    \end{align*}
    Therefore our solution becomes:
    $$u(x,t)=3\exp\left(-n^2\alpha^2\pi^2 t/L^2\right)\cos\frac{2\pi x}{L}$$
    and tends to zero over time.

    \newpage

    \item For part ($a$) consider the solutions to the wave equation:
    $$u(x,t)=\sum_{n=1}^{\infty}c_n
    \sin\frac{n\pi x}{L}\cos\frac{n\pi at}{L}$$
    with \underline{no initial velocity} and
    $$u(x,0)=f(x).$$
    Integrating this over $[0,L]$ and using trigonometric identities gives:
    $$c_n=\frac{2}{L}\int_{0}^{L}
    \sin\frac{n\pi x}{L}f(x)\dd x$$
    Now we set:
    $$f(x)=
    \left\{
    \begin{array}{ll}
	   2x/L  & \mbox{} x\in[0,\frac{L}{2}] \\
	   2(L-x)/L & \mbox{} x\in[\frac{L}{2},L]
    \end{array}
    \right.$$
    and the coefficients are:
    \begin{align*}
        c_n
        &=\frac{2}{L}\int_{0}^{L}
        \sin\frac{n\pi x}{L}f(x)\dd x \\
        &=\frac{2}{L}\int_{0}^{L/2}
        \frac{2}{L}x\sin\frac{n\pi x}{L}\dd x
        +\frac{2}{L}\int_{L/2}^{L}
        (2-\frac{2}{L}x)\sin\frac{n\pi x}{L}\dd x \\
        &=\Bigl(-\frac{2}{n\pi}\cos\frac{n\pi}{2}
        +\frac{4}{n^2\pi^2}\sin\frac{n\pi}{2}\Bigl) \\
        &\quad+
        \Bigl(\frac{2}{n\pi}\cos\frac{n\pi}{2}
        -\frac{4}{n^2\pi^2}\sin n\pi
        +\frac{4}{n^2\pi^2}\sin\frac{n\pi}{2}\Bigl) \\
        &=\frac{8}{n^2\pi^2}\sin\frac{n\pi}{2}
        -\frac{4}{n^2\pi^2}\sin n\pi \\
        &=\frac{8}{n^2\pi^2}\sin\frac{n\pi}{2}
    \end{align*}
    since the last expression is always zero.

    \newpage

    For part ($b$) the solution to boundary condition:
    $$\frac{\partial}{\partial t}u(x,0)=g(x)$$
    is the following expression:
    $$u(x,t)=\sum_{n=1}^{\infty}k_n
    \sin\frac{n\pi x}{L}\sin\frac{n\pi at}{L}$$
    So we have that:
    \begin{align*}
        \frac{\partial}{\partial t}u(x,0)
        &=\sum_{n=1}^{\infty}\frac{\pi a}{L}
        nk_n
        \sin\frac{n\pi x}{L} \\
        &=g(x)
    \end{align*}
    and integrating this expression gives:
    $$k_m=\frac{L}{m\pi a}
    \frac{2}{L}\int_{0}^{L}
    \sin\frac{m\pi x}{L}g(x)\dd x.$$
    Now if $g(x)=f(x)$ we then have:
    \begin{align*}
        k_m
        &=\frac{L}{m\pi a}
        \frac{2}{L}\int_{0}^{L}
        \sin\frac{m\pi x}{L}f(x)\dd x \\
        &=\frac{L}{m\pi a}c_m \\
        &=\frac{8L}{m^3\pi^3 a}\sin\frac{m\pi}{2}
    \end{align*}
    and we are finished.

    \newpage

    \item Let's now consider Laplace's equation:
    $$\frac{\partial^2 u}{\partial x^2}+\frac{\partial^2 u}{\partial y^2}=0$$
    with the following boundary conditions:
    \begin{itemize}
        \item x-boundary: $u(0,y)=0$ and $u(a,y)=0$
        \item y-boundary: $u(x,0)=h(x)$ and $u(x,b)=0$
    \end{itemize}
    for $\forall x\in[0,a]$ and $\forall y\in(0,b)$.

    Note here that $u=u(x,y)$ maps a plane to a line.

    Begin by separation of variables:
    $$u(x,y)=X(x)Y(y).$$
    $$\therefore X''Y+XY''=0$$
    Now choosing a separation constant depends on our boundary conditions:
    $$X(0)=X(a)=Y(b)=0.$$
    $$\therefore\frac{X''}{X}=-\frac{Y''}{Y}=\textcolor{red}{-\lambda}$$
    We identify the two ODEs:
    $$X''+\lambda X=0\implies
    X(x)=a_1\cos\lambda^{1/2} x+a_2\sin\lambda^{1/2} x$$
    $$Y''-\lambda Y=0\implies
    Y(y)=b_1\cosh\lambda^{1/2}y+b_2\sinh\lambda^{1/2}y$$
    and using $X(0)=X(a)=0$ we eliminate $a_1$:
    $$X(a)=a_2\sin\lambda^{1/2}a=0.$$
    $$\therefore\lambda_n=\frac{n^2\pi^2}{a^2}$$
    $$\therefore X_n(x)=a_n\sin\frac{n\pi x}{a}$$
    Turning our attention to the other ODE:
    $$Y(b)=b_1\cosh\lambda_n^{1/2}b+b_2\sinh\lambda_n^{1/2}b=0$$
    and so
    $$b_2=-b_1\frac{\cosh\lambda_n^{1/2}b}{\sinh\lambda_n^{1/2}b}.$$
    $$\therefore Y_n(y)
    =b_n\sinh\left(\frac{n\pi b}{a}
    -\frac{n\pi y}{a}\right)$$

    \newpage

    Putting all these together we obtain the general solution:
    $$u(x,y)=\sum_{n=1}^{\infty}
    c_n
    \sin\frac{n\pi}{a}x
    \hspace{0.03in}
    \sinh\frac{n\pi}{a}(b-y)$$
    and finally since $u(x,0)=h(x)$ for $\forall x\in[0,a]$:
    $$\sum_{n=1}^{\infty}\sin\frac{n\pi}{a}x\hspace{0.03in}
    \sinh\frac{n\pi b}{a}
    =h(x).$$
    Integrating this over $[0,a]$ gives:
    $$c_n=\frac{2}{a}\frac{1}{\sinh\frac{n\pi b}{a}}
    \int_{0}^{a}\sinh\frac{n\pi x}{a}h(x)\dd x.$$
\end{enumerate}