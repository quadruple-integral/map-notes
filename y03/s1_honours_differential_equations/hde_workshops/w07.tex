\pagestyle{fancy}
\fancyhead{}
\fancyhead[L]{Honours DE Workshops}
\fancyhead[R]{Winter 2023}

\section{Workshop 7}
\begin{enumerate}
    \item Find solutions to \textbf{BVPs}.

    For part ($i$) we have that:
    $$y''+y=x\hspace{0.1in}\text{for}\hspace{0.1in}y(0)=y(\pi)=0.$$
    
    We first find the homogeneous solution:
    $$y''+y=0,$$
    and this gives:
    $y_H=\alpha\cos x+\beta\sin x$. The particular solution for the differential equation is $y_p=x$, and therefore our general solution is:
    $$y=x+\alpha\cos x+\beta\sin x.$$
    However if we substitute our \textbf{boundary conditions} we find that:
    $$y(0)=\alpha=0$$
    yet
    $$y(\pi)=\pi-\alpha=0.$$
    This is clearly a contradiction and there are \textbf{no solutions} to this problem. \\

    For part ($ii$) we are asked:
    $$y''+4y=\cos x\hspace{0.1in}\text{for}\hspace{0.1in}y'(0)=y'(\pi)=0.$$

    The homogeneous equation is:
    $$y''+4y=0$$
    and has eigenvalues $\lambda=\pm 2i$, which corresponds to solution:
    $$y_H=\alpha\cos 2x+\beta\sin 2x.$$
    We try for a particular solution of form $y_p=\gamma\cos x+\eta\sin x$ and after substituting our general solution takes the form:
    $$y=\alpha\cos 2x+\beta\sin 2x+\frac{1}{3}\cos x.$$
    Taking derivatives and substituting boundary conditions we find $\beta=0$,
    and so for this \textbf{BVP} there are infinitely many solutions of form:
    $$y=\alpha\cos 2x+\frac{1}{3}\cos x,$$
    where $\alpha\in\mathbb{R}$.

    \newpage

    \item Consider the following function:
    $$g(x)=
    \left\{
    \begin{array}{ll}
        1+x  & \mbox{} x\in[-1,0) \\
        1-x & \mbox{} x\in[0,1).
    \end{array}
    \right.$$
    Find its Fourier series with period $2L$, where $L=1$. \\

    Firstly our function $g$ is an \underline{even} function
    and so its Fourier series is purely in cosine form:
    $$g_{FS}(x)=\frac{c_0}{2}+\sum_{n=1}^{\infty}c_n\cos n\pi x$$
    since $L=1$. Its coefficients are:
    \begin{align*}
        c_0
        &=\frac{1}{L}\int_{-L}^{L}g(x)\dd x \\
        &=\int_{-1}^{1}g(x)\dd x \\
        &=\int_{-1}^{0}(1+x)\dd x
        +\int_{0}^{1}(1-x)\dd x \\
        &=1
    \end{align*}
    and
    \begin{align*}
        c_n
        &=\frac{1}{L}\int_{-L}^{L}\cos\frac{n\pi x}{L}g(x)\dd x \\
        &=\int_{-1}^{1}\cos(n\pi x)g(x)\dd x \\
        &=\frac{2}{(n\pi)^2}\Bigl(1-(-1)^n\Bigr)
    \end{align*}
    for $n\in\{1,2,\dots\}$ and we note that $c_n$ is nonzero in only \underline{odd} terms.
    $$\therefore g_{FS}(x)=\frac{1}{2}+
    \frac{4}{\pi^2}\sum_{m=0}^{\infty}
    \frac{\cos(2m+1)\pi x}{(2m+1)^2}$$

    \newpage

    \item For part ($i$) find the Fourier series of the following:
    $$f(x)=
    \left\{
    \begin{array}{ll}
        -1  & \mbox{} x\in[-L,0) \\
        1 & \mbox{} x\in(0,L]
    \end{array}
    \right.$$
    with period $2L$ and $L=\pi$. \\

    Firstly our function is \underline{odd} as hence:
    $$f_{FS}(x)=\frac{a_0}{2}+\sum_{n=}^{\infty}
    a_n\sin\frac{n\pi x}{L}$$
    where its coefficients are:
    \begin{align*}
        a_0
        &=\frac{1}{L}\int_{-L}^{L}f(x)\dd x \\
        &=\frac{1}{\pi}\int_{-\pi}^{\pi}f(x)\dd x \\
        &=\frac{1}{\pi}\int_{-\pi}^{0}(-1)\dd x
        +\frac{1}{\pi}\int_{0}^{\pi}\dd x \\
        &=0
    \end{align*}
    and
    \begin{align*}
        a_n
        &=\frac{1}{L}\int_{-L}^{L}\sin\frac{n\pi x}{L}f(x)\dd x \\
        &=\frac{1}{\pi}\int_{-\pi}^{\pi}\sin(nx)f(x)\dd x \\
        &=\frac{1}{\pi}\int_{-\pi}^{0}-\sin(nx)\dd x
        +\frac{1}{\pi}\int_{0}^{\pi}\sin(nx)\dd x \\
        &=\frac{2}{n\pi}\bigl[1-(-1)^n\bigr].
    \end{align*}
    Then our Fourier series takes the following form:
    $$f_{FS}(x)=\frac{4}{\pi}\sum_{m=0}^{\infty}
    \frac{\sin(2m+1)x}{2m+1}.$$

    \newpage

    For part ($ii$) consider the following partial sums:
    $$f_N(x)=\frac{4}{\pi}\sum_{m=0}^{N-1}\frac{\sin(2m+1)x}
    {2m+1}.$$
    Show that its derivative is the following:
    $$f'_N(x)=\frac{2}{\pi}\frac{\sin2Nx}{\sin x}.$$ \\

    Starting from our partial sums we take derivatives:
    $$\therefore f'_N(x)=\frac{4}{\pi}\sum_{m=0}^{N-1}
    \Bigl[\cos(2m+1)x\Bigr].$$
    I had the idea of taking the real component of its complex
    exponential, but the algebra proved too tedious. Following the solutions:
    $$\cos x=\frac{1}{2}\bigl(e^{ix}+e^{-ix}\bigr)$$
    and therefore
    $$\cos(2m+1)x=\frac{1}{2}\bigl(e^{i(2m+1)x}+e^{-i(2m+1)x}\bigr).$$
    Substituting this into our partial sums:
    \begin{align*}
        f'_N(x)
        &=\frac{4}{\pi}\sum_{m=0}^{N-1}
        \Bigl[\cos(2m+1)x\Bigr] \\
        &=\frac{2}{\pi}\sum_{m=0}^{N-1}
        \Bigl[e^{i(2m+1)x}+e^{-i(2m+1)x}\Bigr] \\
        &=\frac{2}{\pi}\sum_{m=0}^{N-1}
        \Bigl[e^{i(2m+1)x}+e^{-i(2m+1)x}\Bigr] \\
        &=\frac{2}{\pi}\Bigl(
            e^{ix}\sum_{m=0}^{N-1}\left(e^{i\cdot 2x}\right)^m
            +e^{-ix}\sum_{m=0}^{N-1}\left(e^{-i\cdot 2x}\right)^m
        \Bigr).
    \end{align*}
    Recall the geometric series formula:
    $$\sum_{m=0}^{N-1} r^m=\frac{1-r^N}{1-r}.$$

    \newpage
    $$\therefore\sum_{m=0}^{N-1}\left(e^{i\cdot 2x}\right)^m
    =\frac{1-e^{iN2x}}{1-e^{i2x}}$$
    $$\therefore\sum_{m=0}^{N-1}\left(e^{-i\cdot 2x}\right)^m
    =\frac{1-e^{-iN2x}}{1-e^{-i2x}}$$
    Then:
    \begin{align*}
        f'_N(x)
        &=\frac{2}{\pi}\Bigl(
            e^{ix}\sum_{m=0}^{N-1}\left(e^{i\cdot 2x}\right)^m
            +e^{-ix}\sum_{m=0}^{N-1}\left(e^{-i\cdot 2x}\right)^m
        \Bigr) \\
        &=\frac{2}{\pi}\Bigl(
            e^{ix}\frac{1-e^{iN2x}}{1-e^{i2x}}
            +e^{-ix}\frac{1-e^{-iN2x}}{1-e^{-i2x}}
        \Bigr) \\
        &=\frac{2}{\pi}\Bigl(
            \frac{1-e^{iN2x}}{e^{-ix}-e^{ix}}
            +\frac{1-e^{-iN2x}}{e^{ix}-e^{-ix}}
        \Bigr) \\
        &=\frac{2}{\pi}\Bigl(
            \frac{e^{iN2x}-e^{-iN2x}}{e^{ix}-e^{-ix}}
        \Bigr)
    \end{align*}
    and because:
    $$\sin x=\frac{1}{2i}\bigl(e^{ix}-e^{-ix}\bigr)$$
    we finally have that:
    $$f'_N(x)=\frac{2}{\pi}\frac{\sin2Nx}{\sin x}.$$ \\

    For part ($iii$) set $f'_N(x)=0$ and it is evident that:
    $$2Nx=m\pi$$
    solves the equation.
    $$\therefore x=\frac{m\pi}{2N}$$
    where $m\in\mathbb{Z}$. \\

    For part ($iv$) to show that $x=\displaystyle\frac{\pi}{2N}$
    is a maximum we take second derivatives:
    $$f''(x)=\frac{4N}{\pi}\cos2Nx(\sin x)^{-1}
    -\frac{2}{\pi}\sin 2Nx\cos x (\sin x)^{-2}$$
    and since $f''(\frac{\pi}{2N})<0$ this is a maximum.

    \newpage

    For part ($v$) show that for \underline{large} $N$:
    $$f_N(\frac{\pi}{2N})\approx\frac{2}{\pi}\int_{0}^{\pi}
    \frac{\sin\beta}{\beta}\dd\beta.$$ \\

    So begin by integrating the following expression:
    $$\int_{0}^{\frac{\pi}{2N}}
    \frac{2}{\pi}\frac{\sin2Nx}{\sin x}\dd x.$$
    By substituting $\beta=2Nx$ we get:
    \begin{align*}
        \int_{x=0}^{x=\frac{\pi}{2N}}
        \frac{2}{\pi}\frac{\sin2Nx}{\sin x}\dd x
        &=\int_{\beta=0}^{\beta=\pi}\frac{2}{\pi}
        \frac{\sin\beta}{\sin\frac{\beta}{2N}}
        \frac{\dd\beta}{2N} \\
        &=\frac{2}{\pi}\int_{0}^{\pi}
        \frac{1}{2N}\frac{\sin\beta}{\sin\frac{\beta}{2N}}
        \dd\beta \\
        &\approx\frac{2}{\pi}\int_{0}^{\pi}
        \frac{\sin\beta}{\beta}
        \dd\beta
    \end{align*}
    since when $N\gg\beta$ by small angle approximations
    we have that:
    $$\sin\frac{\beta}{2N}\approx\frac{\beta}{2N}.$$ \\

    Finally for part ($vi$) consider the power series expansion for:
    $$\frac{\sin\beta}{\beta}=\sum_{m=0}^{\infty}
    (-1)^m\frac{\beta^{2m}}{(2m+1)!}.$$
    Using this series in our integral:
    \begin{align*}
        \frac{2}{\pi}\int_{0}^{\pi}
        \frac{\sin\beta}{\beta}
        \dd\beta
        &=\frac{2}{\pi}\sum_{m=0}^{\infty}
        (-1)^m\int_{0}^{\pi}
        \frac{\beta^{2m}}{(2m+1)!}\dd\beta \\
        &=2\sum_{m=0}^{\infty}
        (-1)^m\frac{\pi^{2m}}{(2m)!(2m+1)^2} \\
        &\approx1.18.
    \end{align*}

\end{enumerate}