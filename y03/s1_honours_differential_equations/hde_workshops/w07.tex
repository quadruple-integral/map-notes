\pagestyle{fancy}
\fancyhead{}
\fancyhead[L]{Honours DE Workshop 7}
\fancyhead[R]{Winter 2023}

\begin{enumerate}
    \item Find solutions to \textbf{BVPs}.

    For part ($i$) we have that:
    $$y''+y=x\hspace{0.1in}\text{for}\hspace{0.1in}y(0)=y(\pi)=0.$$
    
    We first find the homogeneous solution:
    $$y''+y=0,$$
    and this gives:
    $y_H=\alpha\cos x+\beta\sin x$. The particular solution for the differential equation is $y_p=x$, and therefore our general solution is:
    $$y=x+\alpha\cos x+\beta\sin x.$$
    However if we substitute our \textbf{boundary conditions} we find that:
    $$y(0)=\alpha=0$$
    yet
    $$y(\pi)=\pi-\alpha=0.$$
    This is clearly a contradiction and there are \textbf{no solutions} to this problem. \\

    For part ($ii$) we are asked:
    $$y''+4y=\cos x\hspace{0.1in}\text{for}\hspace{0.1in}y'(0)=y'(\pi)=0.$$

    The homogeneous equation is:
    $$y''+4y=0$$
    and has eigenvalues $\lambda=\pm 2i$, which corresponds to solution:
    $$y_H=\alpha\cos 2x+\beta\sin 2x.$$
    We try for a particular solution of form $y_p=\gamma\cos x+\eta\sin x$ and after substituting our general solution takes the form:
    $$y=\alpha\cos 2x+\beta\sin 2x+\frac{1}{3}\cos x.$$
    Taking derivatives and substituting boundary conditions we find $\beta=0$,
    and so for this \textbf{BVP} there are infinitely many solutions of form:
    $$y=\alpha\cos 2x+\frac{1}{3}\cos x,$$
    where $\alpha\in\mathbb{R}$.

    \newpage

    \item
\end{enumerate}