\pagestyle{fancy}
\fancyhead{}
\fancyhead[L]{Honours DE Assignment 3}
\fancyhead[R]{Winter 2023}

\begin{enumerate}
    \item Consider $f(x)=\frac{x^2}{2}$ where $0\leq x<L$.

    Part ($a$) wants us to find its Fourier series with period $2L$. \\

    Define an extension of our function $f$ as $g$:
    $$g(x) =
    \left\{
    \begin{array}{ll}
	   f(x)  & \mbox{} x\in[0,L) \\
	   f(-x) & \mbox{} x\in[-L,0)
    \end{array}
    \right.$$
    and therefore $g$ has period $2L$. Since $g$ is also an even function its Fourier series can only consist of cosine terms.
    $$\therefore g_{FS}(x)=\frac{a_0}{2}+\sum_{n=1}^{\infty}a_n\cos\frac{n\pi x}{L}$$
    Its coefficients are:
    \begin{align*}
        a_0
        &=\frac{1}{L}\int_{-L}^{L}g(x)dx \\
        &=\frac{1}{L}\int_{0}^{L}f(x)dx+\frac{1}{L}\int_{-L}^{0}f(-x)dx \\
        &=\frac{1}{L}\int_{0}^{L}f(x)dx-\frac{1}{L}\int_{0}^{-L}f(-x)dx \\
        &=\frac{1}{L}\int_{0}^{L}f(x)dx+\frac{1}{L}\int_{0}^{L}f(x^*)dx^* \\
        &=\frac{2}{L}\int_{0}^{L}f(x)dx
    \end{align*}
    where we define $x=-x^*$ and similarly
    \begin{align*}
        a_n
        &=\frac{1}{L}\int_{-L}^{L}\cos\frac{n\pi x}{L}g(x)dx \\
        &=\frac{1}{L}\int_{0}^{L}\cos\frac{n\pi x}{L}f(x)dx+\frac{1}{L}\int_{-L}^{0}\cos\frac{n\pi x}{L}f(-x)dx \\
        &=\frac{2}{L}\int_{0}^{L}\cos\frac{n\pi x}{L}f(x)dx.
    \end{align*}

    \newpage

    Calculating this explicitly:
    \begin{align*}
        a_0
        &=\frac{2}{L}\int_{0}^{L}f(x)dx \\
        &=\frac{2}{L}\int_{0}^{L}\frac{1}{2}x^2dx \\
        &=\frac{1}{L}\left[\frac{x^3}{3}\right]_{0}^{L} \\
        &=\frac{1}{3}L^2
    \end{align*}
    and
    \begin{align*}
        a_n
        &=\frac{2}{L}\int_{0}^{L}\cos\frac{n\pi x}{L}f(x)dx \\
        &=\frac{2}{L}\int_{0}^{L}\cos\frac{n\pi x}{L}\frac{1}{2}x^2dx \\
        &=\frac{1}{L}\int_{0}^{L}x^2\cos\frac{n\pi x}{L}dx \\
        &=\frac{1}{n\pi}\left[x^2\sin\frac{n\pi x}{L}\right]_{0}^{L}
        -\frac{2}{n\pi}\int_{0}^{L}x\sin\frac{n\pi x}{L}dx \\
        &=\frac{L^2}{n\pi}\sin n\pi
        -\frac{2}{n\pi}\int_{0}^{L}x\sin\frac{n\pi x}{L}dx \\
        &=\frac{L^2}{n\pi}\sin n\pi
        -\frac{2}{n\pi}\left(-\frac{L^2}{n\pi}\cos n\pi+
        \left(\frac{L}{n\pi}\right)^2\sin n\pi\right) \\
        &=\left(\frac{L^2}{n\pi}-\frac{2L^2}{(n\pi)^3}\right)\sin n\pi
        +\left(\frac{2L^2}{(n\pi)^2}\right)\cos n\pi \\
        &=(-1)^n\frac{2L^2}{(n\pi)^2}
    \end{align*}
    where we have integrated by parts, and $\sin n\pi=0$ for $\forall n\in\mathbb{N}$.

    Putting all this together we get our Fourier series for $f$:
    $$\therefore f_{FS}(x)=\frac{L^2}{6}+
    \sum_{n=1}^{\infty}\left((-1)^n\frac{2L^2}{(n\pi)^2}\cos\frac{n\pi x}{L}\right)$$
    and is valid for $\forall x\in[0,L)$.

    \newpage

    For part ($b$) we want $f_{FS}(0)$.
    By Fourier's convergence theorem:
    $$f_{FS}(0)=\frac{L^2}{6}+
    \sum_{n=1}^{\infty}\left((-1)^n\frac{2L^2}{(n\pi)^2}\right).$$

    For part ($c$):
    \begin{align*}
        f(0)
        &=\frac{L^2}{6}+
        \sum_{n=1}^{\infty}\left((-1)^n\frac{2L^2}{(n\pi)^2}\cos\frac{n\pi x}{L}\right) \\
        &=0
    \end{align*}
    $$\therefore\frac{L^2}{6}+\frac{2L^2}{\pi^2}
    \sum_{n=1}^{\infty}\frac{(-1)^n}{n^2}=0$$
    $$\therefore\sum_{n=1}^{\infty}\frac{(-1)^n}{n^2}
    =-\frac{\pi^2}{12}$$

    \newpage

    \item For part ($a$) we given the solution to Laplace's equation:
    $$u(r,\theta)=\frac{a_0}{2}+\sum_{n=1}^{\infty}r^n
    (a_n\cos n\theta+b_n\sin n\theta)$$
    find expressions for coefficients $a_0$, $a_n$ and $b_n$. \\

    Because our solution has period $2\pi$, using the Euler-Fourier formulae:
    $$a_0=\frac{1}{\pi}\int_{-\pi}^{\pi}f(\theta)d\theta$$
    $$a_n=\frac{1}{\pi}\int_{-\pi}^{\pi}
    \cos(n\theta) f(\theta)d\theta$$
    and
    $$b_n=\frac{1}{\pi}\int_{-\pi}^{\pi}
    \sin(n\theta) f(\theta)d\theta.$$

    For part ($b$) we begin with the following expression:
    \begin{align*}
        u(r,\theta)
        &=\frac{1}{2\pi}\int_{-\pi}^{\pi}f(\psi)d\psi \\
        &+\frac{1}{\pi}\sum_{n=1}^{\infty}r^n\left(\cos(n\theta)\int_{-\pi}^{\pi}\cos(n\psi)f(\psi)d\psi
        +\sin(n\theta)\int_{-\pi}^{\pi}\sin(n\psi)f(\psi)d\psi
        \right)
    \end{align*}
    Now since $e^{in(\theta-\psi)}=\cos n(\theta-\psi)+i\sin n(\theta-\psi)$:
    \begin{align*}
        Re(e^{in(\theta-\psi)})
        &=\cos n(\theta-\psi) \\
        &=\cos(n\theta)\cos(n\psi)+\sin(n\theta)\sin(n\psi)
    \end{align*}
    \begin{align*}
        \therefore\int_{-\pi}^{\pi}Re(e^{in(\theta-\psi)})f(\psi)d\psi
        &=\int_{-\pi}^{\pi}\left(\cos(n\theta)\cos(n\psi)+\sin
        (n\theta)\sin(n\psi)\right)f(\psi) d\psi \\
        &=\cos(n\theta)\int_{-\pi}^{\pi}\cos(n\psi)f(\psi)d\psi
        \\ &\quad+\sin(n\theta)\int_{-\pi}^{\pi}\sin(n\psi)f(\psi)d\psi
    \end{align*}
    Therefore our original expression becomes:
    $$u(r,\theta)
    =\frac{1}{2\pi}\int_{-\pi}^{\pi}f(\psi)d\psi
    +\frac{1}{\pi}\sum_{n=1}^{\infty}
    \int_{-\pi}^{\pi}r^nRe(e^{in(\theta-\psi)})f(\psi)d\psi$$

    \newpage

    Taking the real component of each element in a sum is equivalent to taking the real component of the overall sum:
    $$\therefore u(r,\theta)=
    \frac{1}{\pi}Re\left(\frac{1}{2}\int_{-\pi}^{\pi}f(\psi)d\psi
    +\sum_{n=1}^{\infty}
    \int_{-\pi}^{\pi}r^ne^{in(\theta-\psi)}f(\psi)d\psi\right)$$
    Then by the linearity of integrals:
    $$\therefore u(r,\theta)=
    \frac{1}{\pi}Re\left(\int_{-\pi}^{\pi}
    \left[\frac{1}{2}+
    \sum_{n=1}^{\infty}r^ne^{in(\theta-\psi)}\right]f(\psi)d\psi\right)$$

    Finally for part ($c$) we have that
    $$\sum_{n=0}^{\infty}q^n=\frac{1}{1-q}$$
    and let:
    \begin{align*}
        q
        &=re^{i(\theta-\psi)} \\
        &=r\left(\cos(\theta-\psi)+i\sin(\theta-\psi)\right).
    \end{align*}
    Now $|q|<1$ since we defined $r<1$ and $e^{i2\pi}=1$.
    Returning to our sum:
    \begin{align*}
        \sum_{n=1}^{\infty}q^n
        &=\frac{1}{1-q}-1 \\
        &=\frac{q}{1-q}.
    \end{align*}
    Furthermore:
    \begin{align*}
        \frac{1}{2}+\sum_{n=1}^{\infty}q^n
        &=\frac{1}{2}\frac{1+q}{1-q} \\
        &=\frac{1}{2}\frac{1+r(\cos(\theta-\psi)+i\sin(\theta-\psi))}
        {1-r(\cos(\theta-\psi)+i\sin(\theta-\psi))} \\
        &=\frac{1}{2}\left[\frac{(1+rc)+i(rs)}{(1-rc)+i(-rs)}\right]
        \times\frac{1-rc+i(rs)}{1-rc+i(rs)} \\
        &=\frac{1}{2}\frac{1-r^2+i(2rs)}{1+r^2-2rc}
    \end{align*}
    $$\therefore Re(\frac{1}{2}+\sum_{n=1}^{\infty}q^n)
    =\frac{1-r^2}{1+r^2-2rc}$$

    \newpage

    Finally we have that:
    \begin{align*}
        u(r,\theta)
        &=\frac{1}{\pi}Re\left(\int_{-\pi}^{\pi}
        \left[\frac{1}{2}+\sum_{n=1}^{\infty}r^ne^{in(\theta-\psi)}\right]f(\psi)d\psi\right) \\
        &=\frac{1}{\pi}\left(\int_{-\pi}^{\pi}
        Re\left[\frac{1}{2}+\sum_{n=1}^{\infty}r^ne^{in(\theta-\psi)}\right]f(\psi)d\psi\right) \\
        &=\frac{1}{2\pi}\left(\int_{-\pi}^{\pi}
        \frac{1-r^2}{1+r^2-2r\cos(\theta-\psi)}f(\psi)d\psi\right).
    \end{align*}
\end{enumerate}