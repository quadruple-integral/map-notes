\pagestyle{fancy}
\fancyhead{}
\fancyhead[L]{Honours DE Assignment 4}
\fancyhead[R]{Winter 2023}

\section{Assignment 4}
\begin{enumerate}
    \item For part ($a$), $\phi_n(x)$ are the orthonormal
    eigenfunctions and $\lambda_n$ the real eigenvalues of the
    corresponding homogeneous regular S-L problem:
    $$-\dv{x}\left(p(x)\dv{y}{x}\right)+q(x)y=\lambda r(x)y$$
    with initial conditions:
    \begin{itemize}
        \item $\alpha_1 y(0)+\alpha_2 y'(0)=0$
    
        \item $\beta_1 y(1)+\beta_2 y'(1)=0.$
    \end{itemize}
    for $x\in[0,1]$. The solution to the following:
    $$-\dv{x}\left(p(x)\dv{y}{x}\right)+q(x)y=\mu r(x)y+f(x)$$
    is then:
    $$y(x)=\sum_{n=1}^{\infty}b_n\phi_n(x)$$
    where:
    $$b_n=\frac{c_n}{\lambda_n-\mu}$$
    and
    $$c_n=\int_{0}^{1}\phi_n(x)f(x)\dd x.$$ \\

    For part ($b$) solve the following:
    $$\dv[2]{x}y(x)+7y(x)=2\sin 5x+3\sin 7x$$
    with boundary conditions $y(0)=y(\pi)=0$
    for $\forall x\in[0,\pi]$. \\

    First define change of variables $x=\pi t$.
    $$\therefore y(x)\iff y(t)$$
    $$\therefore \dv{x}y(t=\frac{x}{\pi})=\frac{1}{\pi}\dv{y}{t}$$
    $$\therefore \dv{x}\left(\dv{x}y(t=\frac{x}{\pi})\right)
    =\frac{1}{\pi^2}\dv[2]{y}{t}$$

    \newpage

    Then our ODE becomes:
    $$\frac{1}{\pi^2}\dv[2]{y}{t}+7y(t)=2\sin 5\pi t+3\sin 7\pi t$$
    with boundary conditions $y(0)=y(1)=0$
    for $\forall t\in[0,1]$. This is now of S-L form, and its
    corresponding homogeneous S-L system is:
    
\end{enumerate}