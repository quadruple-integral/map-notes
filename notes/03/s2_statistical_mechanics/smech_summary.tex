\documentclass{article}
\usepackage{geometry}
\usepackage{amsmath}
\usepackage{amsfonts}
\usepackage{amssymb}
\usepackage{amsthm}
\usepackage{parskip}
\usepackage{multicol}
\usepackage{xcolor}
\usepackage{fancyhdr}
\usepackage{physics}
\usepackage{graphicx} % Required for inserting images
\usepackage{hyperref}
\usepackage{enumitem}
\usepackage{mathtools}

% commands
\newcommand{\deq}{\vcentcolon=}
\newcommand{\idd}{\text{đ}}
\newcommand{\nimplies}{\centernot\implies}
\newcommand{\vc}[1]{\boldsymbol{#1}}
\newcommand{\pr}{\mathbb{P}}

% margin settings
\geometry{
    a4paper,
    left=7mm,
    right=7mm,
    top=2cm,
    bottom=7mm
}

% testing
\usepackage{blindtext}

% proof environments
\newtheorem{definition}{Definition}[section]
\newtheorem{theorem}{Theorem}[section]
\newtheorem{corollary}{Corollary}[theorem]
\newtheorem{lemma}[theorem]{Lemma}
\newtheorem*{remark}{Remark} % unnumbered remarks

% header and footer
\pagestyle{fancy}
\fancyfoot{} % removes footer
\fancyhf{}
\renewcommand{\headrulewidth}{0.5pt}
\fancyhead[L]{Statistical mechanics}
\fancyhead[R]{\thepage}

\begin{document}

\begin{multicols*}{3}
% starred environment ensures text remains in same column
\noindent

\subsubsection*{Probability distributions}
The probablity of an event in a trial is:
$$\pr(\text{event})\deq\lim_{N\rightarrow\infty}\frac{n}{N}$$
given $n$ occurrences in $N$ trials. \\
For discrete probabilities:
$$\sum_{i=1}^{q}\pr(i)=1$$
$$\pr(\text{$i$ or $j$})=\pr(i)+\pr(j)$$
$$\pr(\text{$i$ and $j$})=\pr(i)\pr(j).$$
Given continuous random variables:
$$\pr([x,x+\dd x])=P(x)\dd x$$
for $P$ is the probability density function:
$$\int_{-\infty}^{\infty}P(x)\dd x=1.$$
We define the \textbf{mean} and \textbf{variance} as:
$$\overline{x}=\sum_{i=1}^{q}x_i P_i
\hspace{0.05in}\text{or}\hspace{0.05in}
\int_{-\infty}^{\infty}xP(x)\dd x$$
\begin{align*}
    \overline{\Delta x^2}
    &=\sum_{i=1}^{q}(x_i-\overline{x})^2 P_i \\
    &=\int_{-\infty}^{\infty}
    (x-\overline{x})^2P(x)\dd x \\
    &=\overline{x^2}-(\overline{x})^2.
\end{align*}
The \textbf{standard deviation} is the square root
of the variance $\bigl(\overline{\Delta x^2}\bigr)^{1/2}$ and:
$$\overline{f(x)}=
\int_{-\infty}^{\infty}f(x)P(x)\dd x.$$

\subsubsection*{Binomial distribution}
The probability of observing $n$ events 
each with probability $p$ in $N$ trials is:
$$P_n=\begin{pmatrix}
N \\ n\end{pmatrix}p^n(1-p)^{N-n}$$
where
$\displaystyle\begin{pmatrix}N \\ n\end{pmatrix}
=\frac{N!}{n!(N-n)!}$ with:
$$\overline{n}=Np
\hspace{0.07in}\text{and}\hspace{0.07in}
\overline{\Delta n^2}=Np(1-p)$$
since we have that:
$$(a+b)^N=\sum_{n=0}^{N}
\begin{pmatrix}N \\ n\end{pmatrix}a^n b^{N-n}$$
\begin{align*}
    f(\alpha)
    &=\sum_{n=0}^{N}\begin{pmatrix}N \\ n\end{pmatrix}
    (p\alpha)^n(1-p)^{N-n} \\
    &=(p\alpha+1-p)^N.
\end{align*}
Note that $\begin{pmatrix}N \\ n\end{pmatrix}$
denotes ways to pick $n$ items 
from $N$ items.
For large $N$:
$$\ln(N!)\approx N\ln(N)-N$$
known as \textbf{Stirling's approximation}.

\newcolumn

We also define the \textbf{fractional deviation}
as the deviation on the scale of the mean:
$$\frac{\bigl(\overline{\Delta x^2}\bigr)^{1/2}}{\overline{n}}
=\frac{1}{N^{1/2}}.$$

\subsubsection*{Taylor expansions}
Let $s(n)$ be expanded at $n=a$:
\begin{align*}
    s(n)&=s(a)+s'(a)(n-a) \\
    &\quad+\frac{1}{2}s''(a)(n-a)^2+\mathcal{O}[(n-a)^3].
\end{align*}

\subsubsection*{Poisson distribution}
Let $N\gg n$ and let $p$ be the probability of 
an event in a trial. Assume that as $N\rightarrow\infty$,
$p\rightarrow0$. Under such conditions
the binomial probability of observing $n$ events 
in $N$ trials is:
\begin{align*}
    P_n
    &\approx(\overline{n})^n\frac{\exp(-\overline{n})}{n!}
\end{align*}
with mean and variance $Np$.

\subsubsection*{Gaussian distribution}
Let $N$ be very large. Then the binomial distribution
becomes Gaussian:
$$P_n\approx\frac{1}{\sqrt{2\pi\sigma^2}}
\exp\left(-\frac{(n-Np)^2}{2\sigma^2}\right)$$
via Stirling's approximation and Taylor expansions
with variance $\sigma^2=Np(1-p)$ and mean $\mu=Np$.

\subsubsection*{Microstates and macrostates}

\end{multicols*}

\end{document}