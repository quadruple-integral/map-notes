\documentclass{article}
\usepackage{geometry}
\usepackage{amsmath}
\usepackage{amsfonts}
\usepackage{amssymb}
\usepackage{amsthm}
\usepackage{parskip}
\usepackage{multicol}
\usepackage{xcolor}
\usepackage{fancyhdr}
\usepackage{physics}
\usepackage{graphicx} % Required for inserting images
\usepackage{hyperref}
\usepackage{enumitem}
\usepackage{mathtools}
\usepackage[utf8]{inputenc}
\usepackage[T1]{fontenc}

% commands
\newcommand{\deq}{\vcentcolon=}
\newcommand{\idd}{\text{đ}}

% margin settings
\geometry{
    a4paper,
    left=7mm,
    right=7mm,
    top=2cm,
    bottom=7mm
}

% testing
\usepackage{blindtext}

% proof environments
\newtheorem{definition}{Definition}[section]
\newtheorem{theorem}{Theorem}[section]
\newtheorem{corollary}{Corollary}[theorem]
\newtheorem{lemma}[theorem]{Lemma}
\newtheorem*{remark}{Remark} % unnumbered remarks

% header and footer
\pagestyle{fancy}
\fancyfoot{} % removes footer
\fancyhf{}
\renewcommand{\headrulewidth}{0.5pt}
\fancyhead[L]{Thermodynamics}
\fancyhead[R]{\thepage}

\begin{document}

\begin{multicols*}{3}
\noindent

\subsubsection*{Definitions}

\textbf{Isolated system}: No exchanges

\textbf{Closed system}: Only energy exchange

\textbf{Open system}: Energy \& mass exchange

\textbf{Intensive state variables}: \\
Independent of mass

\textbf{Extensive state variables}: \\
Proportional to mass

\textbf{Reservoirs}: Infinite/very large system
that remains unchanged when in contact
with finite system.

\textbf{Mechanical equilibrium}: \\
No unbalanced forces

\textbf{Thermal equilibrium}: \\
No temperature differences

\textbf{Thermodynamic equilibrium}: \\
Intensive state variables of system are \underline{constant}.
Alternatively our system is in \underline{mechanical} and
\underline{thermal} equilibrium.

\textbf{Reversible processes}: \\
Every intermediate is an equilibrium state.

\textbf{Quasi-static processes}: \\
Process sufficiently slow such that only \\
infinitesimal temperature or pressure gradients exist.

Frictionless quasi-static processes are \underline{reversible}.

\textbf{Cyclic processes}:
$$\Delta U=0
\hspace{0.05in}\text{and}\hspace{0.05in}
W=Q$$
For conservative forces:
$$\oint\dd X=0$$
where $X$ is a state variable.

\textbf{Adiabatic processes}: $\Delta Q=0$

\textbf{Isothermal processes}: $\Delta T=0$

\textbf{Isobaric processes}: $\Delta P=0$

\subsubsection*{Density}
We define the density of a material as:
$$\rho=\frac{m}{V}.$$
If mass $m$ is constant:
$$\Delta V=m\left(\frac{1}{\rho_f}
-\frac{1}{\rho_i}\right)$$
assuming homogeneous material.

\subsubsection*{Zeroth law}
If $A$ is in thermal equilibrium with $B$ and $C$
\underline{seperately} then $B$ and $C$
are also in thermal equilibrium.

\subsubsection*{Ideal gas state equation}
Given $n$ moles of gas at temperature $T$:
\begin{align*}
    PV&=nRT \\
    &=Nk_B T
\end{align*}
where $R=N_A k_B=8.314$JK$^{-1}$mol$^{-1}$ \\
and $N$ the number of molecules.

\subsubsection*{Calculus identities}
\begin{enumerate}
    \item $\displaystyle\dd f(x,y)
    =\left(\frac{\partial f}{\partial x}\right)_y \dd x
    +\left(\frac{\partial f}{\partial y}\right)_x \dd x
    $ \\
    if $f=f(x,y)$.

    \item Differential $\idd f$ is \textbf{inexact} if:
    $$\int_C \idd f
    \hspace{0.05in}\text{is \underline{dependent} of path}.$$

    \item $\displaystyle
    \left(\frac{\partial Z}{\partial Y}\right)_X
    =\left[\left(\frac{\partial Y}{\partial Z}\right)_X
    \right]^{-1}$

    \item $\displaystyle
    \left(\frac{\partial X}{\partial Y}\right)_Z
    \left(\frac{\partial Y}{\partial Z}\right)_X
    \left(\frac{\partial Z}{\partial X}\right)_Y
    =-1$
\end{enumerate}

\subsubsection*{First law}
Total energy $E$ is conserved and:
$$\Delta U=Q-W$$
$$\dd U=\idd Q-\idd W$$
$$\dot{U}=\dot{Q}-\dot{W}$$
where $U$ is internal energy and $E\geq U$.
\begin{itemize}
    \item Heat exchange $Q$: \\
    The energy transfer of two systems at
    different temperatures in thermal contact.
    $Q>0$ represents energy transfer
    \underline{into} system.

    \item Work exchange $W$: \\
    The work done \underline{on} the surroundings by system
    is represented by $W>0$.

    Work is generally \underline{path dependent}.
\end{itemize}

The work done \underline{by} a fluid
in \textbf{reversible} processes is:
$$\idd W=P\dd V$$
and has units Joules (J).

\subsubsection*{Isothermal expansion}
Let $P_1>P_2$ where $P_1$ and $P_2$ 
denote system and external
pressure respectively.
Only mechanical work is exchanged via
a piston.
By applying a force
such that there exists pressure difference $\dd P$, 
our expansion becomes
reversible and hence:
$$W_{1\rightarrow2}=nRT\int_{V_1}^{V_2}
\frac{\dd V}{V}.$$
Note that for isothermal processes
under \underline{ideal} gas assumption, $\Delta U=0$.

\subsubsection*{Heat capacity}
Heat capacity (JK$^{-1}$) is defined as:
$$C(P,T)=\lim_{\Delta T\rightarrow0}
\frac{\Delta Q}{\Delta T}$$
and is the heat needed to produce
unit change in sample temperature.

Specific heat capacity
(Jkg$^{-1}$K$^{-1}$):
$$Q=mc\Delta T.$$

We define the \textbf{isochoric} 
heat capacity as:
$$C_V(T)\deq\left(\frac{\idd Q}{\dd T}\right)_V
=\left(\frac{\partial U}{\partial T}\right)_V$$
and the \textbf{isobaric} heat capacity as:
\begin{align*}
    C_P
    &\deq\left(\frac{\idd Q}{\dd T}\right)_P \\
    &=C_V+\left[P+
    \left(\frac{\partial U}{\partial V}\right)_T\right]
    \left(\frac{\partial V}{\partial T}\right)_P.
\end{align*}

For ideal gases we have that:
$$C_P-C_V=nR.$$

\subsubsection*{Adiabatic expansion}
The \underline{reversible} \underline{adiabatic}
expansion of an \textbf{ideal} gas is given by:
\begin{align*}
    &\dd U=-P\dd V
    \hspace{0.05in}\text{and}\hspace{0.05in}
    \dd U=C_V\dd T \\
    &\implies
    \frac{\dd T}{T}
    +\frac{C_P-C_V}{C_V}\frac{\dd V}{V}=0
\end{align*}
since $U=U(T)$.
Integrating this yields:
$$TV^{\gamma-1}=\text{constant}$$
$$PV^{\gamma}=\text{constant}$$
$$PT^{\frac{\gamma}{\gamma-1}}=\text{constant}$$
where $\gamma$ is the \underline{adiabatic
exponent}:
$$\gamma=\frac{C_P}{C_V}
=\frac{f+2}{f}$$
$$U=\frac{f}{2}nRT$$
and $f$ is degrees of freedom.
The practical computation of work done
for adiabats is given by:
$$W_{1\rightarrow2}
=-\int_{T_1}^{T_2}C_V\dd T.$$

\subsubsection*{General form for first law}
Given system with $m$ conjugate pairs
$(x_i,X_i)$ that represent various modes
of work exchange:
$$\dd U=\idd Q+
\sum_{i=1}^{m}x_i\dd X_i$$
for each $\{x_i\}$ 
drives $\{X_i\}$.

\columnbreak

\subsubsection*{Enthalpy}
The state function enthalpy simplies the
description of heat transfer. 

Enthalpy has units J
and is defined as:
$$H=U+PV.$$
Under \underline{reversible} conditions:
\begin{align*}
    \dd H
    &=\dd U+P\dd V+V\dd P \\
    &=\idd Q+V\dd P
\end{align*}
$$\therefore\dd H=\idd Q_P
\implies C_P
=\left(\frac{\partial H}{\partial T}\right)_P.$$
Latent heat (J) is heat needed for sample
to undergo a phase transition:
$$\Delta U=Q_{\ell}-P\Delta V
\implies Q_{\ell}=\Delta H.$$

\subsubsection*{Carnot's theorem}
Peak efficiency of a \underline{cyclic}
heat engine:
\begin{align*}
    \eta&\deq\frac{W}{Q_H}
    =\frac{\dot{W}}{\dot{Q_H}} \\
    &=1-\frac{Q_C}{Q_H}=1-\frac{T_C}{T_H}
\end{align*}
since $\Delta U=0
\implies W=Q=Q_H-Q_C$.
This is the Carnot cycle:
\begin{center}
    \includegraphics*[scale=0.13]{f0.png}
\end{center}
where AB,CD are isothermal processes
and BC, DA are adiabatic processes.

\subsubsection*{Entropy}
The state function entropy is a measure
of disorder defined as:
$$S=\frac{Q}{T}$$
where $Q$ is heat received from a reservoir
at temperature $T$
and units JK$^{-1}$.

For \underline{reversible} processes:
$$\Delta S_{1\rightarrow2}
=\int_{T_1}^{T_2}\frac{\idd Q}{T}$$
$$\dd U=T\dd S-P\dd V$$
$$\dd H=T\dd S+V\dd P.$$

\subsubsection*{Entropy of mixing}
For ideal gases A and B:
$$\Delta S=n_A R\ln\frac{V_A+V_B}{V_A}
+n_B R\ln\frac{V_A+V_B}{V_B}.$$
The \underline{molar} specific entropy of mixing is:
$$\Delta s_{mix}=-R(x_A\ln x_A+x_B\ln x_B)$$
$$x_A=\frac{n_A}{n_A+n_B}
\hspace{0.05in}\text{and}\hspace{0.05in}
x_B=\frac{n_B}{n_A+n_B}.$$

\subsubsection*{Second law}
Total entropy cannot decrease:
$$\Delta S_{total}=\Delta S_{system}
+\Delta S_{reservoir}\geq0$$
due to the Clausius inequality:
$$\dd S\geq\frac{\idd Q}{T}.$$
$$\therefore W_{rev}-W_{irr}
=T\Delta S_{irr}>0$$

\subsubsection*{Helmholtz free energy}
$$F=U-TS$$
For \underline{reversible} processes:
$$\dd F=-S\dd T-P\dd V.$$

\subsubsection*{Gibbs free energy}
$$G=H-TS$$
For \underline{reversible} processes:
$$\dd G=-S\dd T+V\dd P.$$

\subsubsection*{Chemical reactions}
Since
$Q=\Delta U+P_0\Delta V=\Delta H$:
\begin{itemize}
    \item $Q<0$: exothermic \\
    (heat is released)
    \item $Q>0$: endothermic \\
    (heat is absorbed)
\end{itemize}
at constant pressure $P_0$.

Chemical reactions are spontaneous if:
$$\Delta G=\Delta H-T\Delta S<0.$$

\subsubsection*{Maxwell relations}
$$\left(\frac{\partial T}{\partial V}\right)_S
=-\left(\frac{\partial P}{\partial S}\right)_V$$
$$\left(\frac{\partial T}{\partial P}\right)_S
=\left(\frac{\partial V}{\partial S}\right)_P$$
$$\left(\frac{\partial S}{\partial V}\right)_T
=\left(\frac{\partial P}{\partial T}\right)_V$$
$$-\left(\frac{\partial S}{\partial P}\right)_T
=\left(\frac{\partial V}{\partial T}\right)_P$$
The isobaric expansivity is defined as:
$$\beta=\frac{1}{V}
\left(\frac{\partial V}{\partial T}\right)_P$$
and the isothermal compressibility:
$$\kappa_T=-\frac{1}{V}
\left(\frac{\partial V}{\partial P}\right)_T.$$

\subsubsection*{Throttling}
Throttling is the \underline{adiabatic} 
reduction in gas pressure
and is an isenthalpic process.
We define
the slope of a P-T plot:
\begin{align*}
    \mu_{JK}&=\left(
    \frac{\partial T}{\partial P}\right)_H \\
    &=\frac{V(T,P)}{C_P}(\beta T-1)
\end{align*}
as the
Joule-Kelvin coefficient.

\subsubsection*{Clausius-Clapeyron equation}
The slope of any phase boundary is:
$$\dv{P}{T}=\frac{\Delta S}{\Delta V}
=\frac{\Delta H}{T\Delta V}$$
since constant pressure at boundaries.

\subsubsection*{Van der Waals state equation}
$$\left(P+\frac{an^2}{V^2}\right)
\bigl(V-nb\bigr)=nRT$$

\subsubsection*{Chemical potentials}
The Euler equation for a $1$-component
\textbf{open} system with
$N$ particles is:
$$U=TS-PV+\mu N$$
with modified first law statement: 
$$\dd U
=T\dd S-P\dd V+\mu\dd N.$$
This gives the Gibbs-Duhem relation:
$$S\dd T-V\dd P+N\dd\mu=0$$
where $\mu$ is the chemical potential:
$$\mu=\frac{G}{N}$$
since $G$ is extensive. At constant $T$
with ideal gas assumptions:
$$\mu(P,T)=RT\ln\frac{P}{P_0}+\mu_0(P,T).$$
Chemical potential $\mu$ has units J.

\subsubsection*{Third law}
$S=0$ at $T=0$K.

\subsubsection*{Questions}
\begin{enumerate}
    \item What is temperature?
\end{enumerate}

\end{multicols*}

\end{document}