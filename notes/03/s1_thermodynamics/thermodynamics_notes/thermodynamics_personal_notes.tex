\documentclass{article}
\usepackage{amsmath}
\usepackage{amsfonts}
\usepackage{amssymb}
\usepackage{amsthm}
\usepackage{parskip}
\usepackage{multicol}
\usepackage{xcolor}
\usepackage{fancyhdr}
\usepackage{physics}
\usepackage{graphicx} % Required for inserting images
\usepackage{hyperref}
\newcommand{\matr}[1]{\mathbf{#1}}
\def\dbar{{\mathchar'26\mkern-12mu d}}

\title{Thermodynamics}
\author{Notes by Christopher Shen}
\date{Winter 2023}

\begin{document}

\maketitle
\newpage

\tableofcontents
\newpage

\pagestyle{fancy}
\fancyhead{}
\fancyhead[L]{Thermodynamics}
\fancyhead[R]{Winter 2023}

\section{Zeroth law}

Zeroth law

temperature scales

\newpage

\section{Definitions}

\subsection{Ideal gas state equation}

\subsection{Systems and processes}

\newpage

\section{Isothermal expansion}


\section{First law}

\newpage

\section{Heat capacities}
include derivations

\newpage

\section{Enthalpy}

\subsection{State function enthalpy}
We define \underline{enthalpy} as:
$$H=U+PV.$$
This state function greatly simplifies our $C_P$
definition:
$$C_P=\left(\dv{Q}{T}\right)_P=\left(
\frac{\partial H}{\partial T}\right)_P$$
and its differential is:
$$\dd H=\dd U+V\dd P+P\dd V.$$

\subsection{Latent heats}

\subsection{Constant pressure expansion}

\subsection{Filling empty tank}

\subsection{Chemical reactions}

\newpage

\section{Heat engines}
include hurricanes application

\newpage

\section{Entropy}

\subsection{State function entropy}
\textbf{Entropy} is defined as:
$$S=\frac{Q}{T},$$
where $Q$ is \underline{heat from reservoir} and $T$ the \underline{reservoir temperature}.

Entropy has units $JK^{-1}$, and if we have a \textbf{reversible cyclic} heat engine:
$$\dd S=\frac{\dd Q}{T}.$$
This is the differential form of entropy. Applying the first law under again \textbf{reversible} conditions gives:
$$\dd U=T\dd S-P \dd V$$
The entropy change of a system is:
\begin{align*}
    \Delta S
    &=\int_{(1)}^{(2)}\frac{\dd U}{T}+\int_{(1)}^{(2)}P\dd V \\
    &=\int_{(1)}^{(2)}\frac{\dd Q}{T},
\end{align*}
for a \textbf{reversible path}.

\subsection{Entropy of ideal gases}
Let's now assume an \underline{ideal}, \textbf{monoatomic} gas. Then its heat capacity becomes:
$$C_V=\frac{\dd U}{\dd T}.$$
So $C_V \dd T=T\dd S-P\dd V$ and $C_V=\frac{3}{2}nR$. Combining these:
$$\dd S=nR\left(\frac{3}{2}\frac{\dd T}{T}+\frac{\dd V}{V}\right)$$
which can be discretised to find $\Delta S$. Rearranging we get $T\dd S=C_V \dd T+P \dd V$. Integrating and using identity $C_P-C_V=nR$ we get:
$$\Delta S=nc_V\ln\left(\frac{P_2}{P_1}\right)+nc_P\ln\left(\frac{V_2}{V_1}\right)$$
for here $n$ is moles and the heat capacities are in molar quantities.

\newpage

\subsection{Entropy of mixing}
Consider a box with separator, containing $n_A$ moles at $V_A$ of gas A on one side and $n_B$ moles at $V_B$ of gas B on the other. We set $\Delta T=0$, $\Delta P=0$ and proceed to \underline{mix} gases A and B.

The \textbf{entropy of mixing} is:
\begin{align*}
    \Delta S
    &=\Delta S_A+\Delta S_B \\
    &=n_i R\ln\left(\frac{V_f}{V_i}\right) \\
    &=n_A R\ln\left(\frac{V_A+V_B}{V_A}\right)
    +n_B R\ln\left(\frac{V_A+V_B}{V_B}\right),
\end{align*}
Since $P$ and $T$ are fixed, the ideal state equation $PV=nRT$ implies that:
$$\frac{V_A+V_B}{V_A}=\frac{n_A+n_B}{n_A}$$
and so we define \textbf{inverse mole fractions} as:
$$x_A=\frac{n_A}{n_A+n_B}$$
and
$$x_B=\frac{n_B}{n_A+n_B}.$$
After substituting and dividing through by $n_A+n_B$ we find the \textbf{molar specific entropy of mixing}:
$$\Delta s_{mix}=-R(x_A\ln x_A+x_B\ln x_B).$$

\subsection{Measuring entropy}
Practically, $C_P$ is much easier to measure than $C_V$. Recall definition:
$$C_P=\frac{\dd Q_p}{\dd T}=\left(\frac{\dd H}{\dd T}\right)_p$$
where state function $H$ is the enthalpy. Since $H=U+PV$:
$$\dd H=T\dd S+V\dd P$$
and dividing by $\dd T$:
$$\frac{\dd H}{\dd T}=T\frac{\dd S}{\dd T}+V\frac{\dd P}{\dd T}.$$
If $\Delta P=0$ then:
$$C_P(T)=T\left(\frac{\dd S}{\dd T}\right)_P.$$
Rearranging and integrating gives us the change in entropy:
$$\Delta S=\int_{(1)}^{(2)}\frac{C_P(T)}{T}\dd T.$$

\newpage

\section{Second law}

\subsection{Reversibility}

\subsection{Second law}
include spontaneous process

sections 12 and 13

\newpage

\section{Helmholtz and Gibbs free energy}
include chemical reactions and colloidal particles

sections 14 and 15


\newpage
    

\section{Maxwell relations}

\subsection{Derivation}
So from previous sections we have the following four differential relations\footnote{Here $U$ is the internal energy, $H$ is enthalpy, $F$ is the Helmholtz free energy and $G$ is the Gibbs free energy.}:
$$\dd U=T\dd S-P\dd V,$$
$$\dd H=T\dd S+V\dd P,$$
$$\dd F=-S\dd T-P\dd V,$$
$$\dd G=-S\dd T+V\dd P.$$
For equation of state $f(P,V,T)=0$, the \textbf{Maxwell relations} are the following:
$$\left(\frac{\partial T}{\partial V}\right)_S=-\left(\frac{\partial P}{\partial S}\right)_V,$$
$$\left(\frac{\partial T}{\partial P}\right)_S=\left(\frac{\partial V}{\partial S}\right)_P,$$
$$\left(\frac{\partial S}{\partial V}\right)_T=\left(\frac{\partial P}{\partial T}\right)_V,$$
$$-\left(\frac{\partial S}{\partial P}\right)_T=\left(\frac{\partial V}{\partial T}\right)_P.$$
Notice how each equation relates entropy $S$ to \underline{measurable quantities}.

These relations are derived by taking \textbf{total differentials} and then applying \textbf{Clairaut's theorem}. We are only going to do the first relation:    
\begin{align*}
    \dd U
    &= T\dd S-P\dd V\\
    &= \left(\frac{\partial U}{\partial S}\right)_V \dd S
    + \left(\frac{\partial U}{\partial V}\right)_S \dd V
\end{align*}
Therefore we have that:
$$T=\left(\frac{\partial U}{\partial S}\right)_V$$
and 
$$P=-\left(\frac{\partial U}{\partial V}\right)_S.$$
Taking partials again and equating gives us the first relation.

\newpage

\subsection{Applications}
We can define \textbf{isobaric expansivity} $\beta$ with units $K^{-1}$ as:
$$\beta=\frac{1}{V}
\left(\frac{\partial V}{\partial T}\right)_P$$
and \textbf{isothermal compressibility} $\kappa_T$ with units $Pa^{-1}$ as:
$$\kappa_T=-\frac{1}{V}
\left(\frac{\partial V}{\partial P}\right)_T.$$
These may be thought of as volume change per physical unit change.

The Maxwell relations can be used to find entropy. For example:
\begin{align*}
    \Delta S
    &=\int_{(1)}^{(2)}\dd S\\
    &=-\int_{(1)}^{(2)}\left(\frac{\partial V}{\partial T}\right)_P \dd P.
\end{align*}

\subsubsection{Ideal gas internal energy}
Now we can show why $U=U(T)$ if we have an ideal gas. Beginning with:
$$\dd U=T\dd S-P\dd V$$
and dividing through by $\dd V$ gives:
\begin{align*}
    \left(\frac{\partial U}{\partial V}\right)_T
    &=T\left(\frac{\partial S}{\partial V}\right)_T-P \\
    &=T\left(\frac{\partial P}{\partial T}\right)_V-P
\end{align*}
via a Maxwell relation.
Finally our ideal gas assumption implies that:
$$P=\frac{nRT}{V}$$
and substituting this into the previous expression results in:
$$\left(\frac{\partial U}{\partial V}\right)_T=0$$
which means that $U=U(T)$.

\newpage

\subsubsection{Difference in heat capacities}
We begin with the following expression:
\begin{align*}
    C_P
    &=C_V+\bigl(P+
    \left(\frac{\partial U}{\partial V}\right)_T\bigl)
    \left(\frac{\partial V}{\partial T}\right)_P \\
    &=C_V+T\left(\frac{\partial P}{\partial T}\right)_V
    \left(\frac{\partial V}{\partial T}\right)_P \\
    &=C_V+T\left(\frac{\partial P}{\partial T}\right)_V
    \beta V.
\end{align*}
$$\therefore C_P-C_V=\beta TV
\left(\frac{\partial P}{\partial T}\right)_V$$
Then using the following identities:
$$\left(\frac{\partial P}{\partial T}\right)_V
\left(\frac{\partial V}{\partial P}\right)_T
\left(\frac{\partial T}{\partial V}\right)_P=-1$$
$$\frac{1}{\left(\frac{\partial V}{\partial P}\right)_T}
=\left(\frac{\partial P}{\partial V}\right)_T$$

we get that the difference in heat capacities is:
$$C_P-C_V=\frac{\beta^2}{\kappa_T} TV$$
where $\beta$ is the \textbf{isobaric expansivity}
and $\kappa_T$ is the \textbf{isothermal compressibility}.
Furthermore it is clear that $C_P-C_V>0$ for all cases.

\subsubsection{Liquefying gases}

\newpage

\section{Phase transitions}

\subsection{PT diagrams}
phase transition diagrams (PT)

supercritical points

phase boundaries

triple point

clausius clapeyron equation

\newpage

\subsection{Van der Waals state equation}
The Van der Waals equation is a modification of the 
ideal gas equation:
$$\left(P+\frac{an^2}{V^2}\right)
\bigl(V-nb\bigr)=nRT$$
and defines the volume $V$ occupied by $n$ moles of gas
at pressure $P$ and temperature $T$.
The \underline{molar} version of this is the following:
$$\left(P+\frac{a}{v^2}\right)(v-b)=RT$$
where we define the \underline{molar volume}
as $v=\frac{V}{n}$.

\subsubsection{Helmholtz and Gibbs free energies}
Since we have that
$\dd F=-S\dd T-P\dd V$ at fixed temperatures:
$$\left(\frac{\partial F}{\partial V}\right)_T=-P$$
and integrating the molar version of this:
\begin{align*}
    f
    &=-\int P\hspace{0.03in}\dd v \\
    &=\tau(T)-RT\ln(v-b)-\frac{a}{v}
\end{align*}
where we have substituted the Van der Waals equation.

For the Gibbs free energy because
$f=u-TS$ we then have that:
\begin{align*}
    g
    &=u+Pv-TS \\
    &=f+Pv \\
    &=\tau(T)-RT\ln(v-b)-\frac{a}{v}+Pv
\end{align*}


\newpage

\subsection{PV diagrams}

\newpage

\section{Chemical potentials}

\newpage

\section{Third law}

\end{document}
