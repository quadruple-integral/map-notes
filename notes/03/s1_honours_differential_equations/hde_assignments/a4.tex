\pagestyle{fancy}
\fancyhead{}
\fancyhead[L]{Honours DE Assignment 4}
\fancyhead[R]{Winter 2023}

\section{Assignment 4}
\begin{enumerate}
    \item For part ($a$), $\phi_n(x)$ are the orthonormal
    eigenfunctions and $\lambda_n$ the real eigenvalues of the
    corresponding homogeneous regular S-L problem:
    $$-\dv{x}\left(p(x)\dv{y}{x}\right)+q(x)y=\lambda r(x)y$$
    with initial conditions:
    \begin{itemize}
        \item $\alpha_1 y(0)+\alpha_2 y'(0)=0$
    
        \item $\beta_1 y(1)+\beta_2 y'(1)=0.$
    \end{itemize}
    for $x\in[0,1]$. The solution to the following:
    $$-\dv{x}\left(p(x)\dv{y}{x}\right)+q(x)y=\mu r(x)y+f(x)$$
    is then:
    $$y(x)=\sum_{n=1}^{\infty}b_n\phi_n(x)$$
    where:
    $$b_n=\frac{c_n}{\lambda_n-\mu}$$
    and
    $$c_n=\int_{0}^{1}\phi_n(x)f(x)\dd x.$$ \\

    For part ($b$) solve the following:
    $$\dv[2]{x}y(x)+7y(x)=2\sin 5x+3\sin 7x$$
    with boundary conditions $y(0)=y(\pi)=0$
    for $\forall x\in[0,\pi]$. \\

    First define change of variables $x=\pi t$.
    $$\therefore y(x)\iff y(t)$$
    $$\therefore \dv{x}y(t=\frac{x}{\pi})=\frac{1}{\pi}\dv{y}{t}$$
    $$\therefore \dv{x}\left(\dv{x}y(t=\frac{x}{\pi})\right)
    =\frac{1}{\pi^2}\dv[2]{y}{t}$$

    \newpage

    Then our ODE becomes:
    $$\frac{1}{\pi^2}\dv[2]{y}{t}+7y(t)=2\sin 5\pi t+3\sin 7\pi t$$
    with boundary conditions $y(0)=y(1)=0$
    for $\forall t\in[0,1]$. This is now of S-L form, and its
    corresponding homogeneous S-L system is:
    $$y''+\pi^2\lambda y=0.$$
    We know that S-L problems have real valued eigenvalues, and so
    we consider the sign of $\lambda$ separately.
    Now if $\lambda=0$ we have linear solutions:
    $$y=a_1 t+a_2$$
    but since $y(0)=a_2=0$ and $y=a_1=0$,
    only trivial solutions remain.

    If $\lambda<0$ we have solutions:
    $$y=b_1\cosh\pi\sqrt{\lambda}t
    +b_2\sinh\pi\sqrt{\lambda}t$$
    and using boundary conditions, $y(0)=b_1=0$
    and $y(1)=b_2\sinh\pi\sqrt{\lambda}=0$.
    This also yields only trivial solutions since:
    $$\sinh\pi\sqrt{\lambda}=\frac{1}{2}
    \bigl(e^{\pi\sqrt{\lambda}}
    -e^{-\pi\sqrt{\lambda}}\bigr)=0$$
    implies $e^{2\pi\sqrt{\lambda}}=1$ and the only
    eigenvalue satifying this is $\lambda=0$.

    Finally if $\lambda>0$ we have solutions of form:
    $$y=c_1\sin\pi\sqrt{\lambda}t
    +c_2\cos\pi\sqrt{\lambda}t$$
    and our boundary conditions yields $y(0)=c_2=0$ and:
    $$y(1)=c_1\sin\pi\sqrt{\lambda}=0$$
    implies $\lambda_n=n^2$ where $n\in\mathbb{N}$.
    So our eigenfunctions are:
    \begin{align*}
        \phi_n(t)
        &=k_n \sin\pi\sqrt{\lambda_n}t \\
        &=k_n \sin n\pi t.
    \end{align*}
    Since $\langle\phi_n,\phi_n\rangle=1$
    and $r(x)=\pi^2$ we then have that:
    $$\int_{0}^{1}\pi^2 k_n^2(\sin^2n\pi t)\dd t=1.$$
    $$\therefore k_n=\frac{1}{\pi}
    \Bigl[1-\frac{1}{2\pi n}\sin2\pi n\Bigr]^{-1/2}$$

    \newpage

    Returning to our original ODE
    and rearranging it into S-L form:
    $$-y''=7\pi^2y-\pi^2
    \bigl(2\sin5\pi t+3\sin7\pi t\bigr)$$
    where we have $\mu=7$, $r(t)=\pi^2$ and
    $$f(t)=-\pi^2\bigl(2\sin5\pi t
    +3\sin7\pi t\bigr).$$
    Let solutions be of the following form:
    $$y(t)=\sum_{n=1}^{\infty}
    b_n\phi_n(t)$$
    where
    $$b_n=\frac{c_n}{\lambda_n-\mu}$$
    and
    \begin{align*}
        c_n
        &=\int_{0}^{1}\phi_n(x)f(x)\dd x \\
        &=-\pi^2k_n
        \Bigl[
        \int_{0}^{1}\sin n\pi t
        \bigl(2\sin5\pi t+3\sin7\pi t\bigr)\dd t
        \Bigr] \\
        &=-\pi^2k_n\Bigl[2\int_{0}^{1}
        \sin(n\pi t)\sin(5\pi t)\dd t
        +3\int_{0}^{1}
        \sin(n\pi t)\sin(7\pi t)\dd t\Bigr] \\
        &=-\pi^2k_n\Bigl[\frac{\sin(n-5)\pi}{(n-5)\pi}
        -\frac{\sin(n+5)\pi}{(n+5)\pi}\Bigr] \\
        &\quad-\frac{3}{2}\pi^2k_n
        \Bigl[\frac{\sin(n-7)\pi}{(n-7)\pi}
        -\frac{\sin(n+7)\pi}{(n+7)\pi}\Bigr]
    \end{align*}
    for $n\neq5,7$. We then have that:
    $$b_n=\frac{c_n}{n^2-7}$$
    and
    $$k_n=\frac{1}{\pi}
    \Bigl[1-\frac{1}{2\pi n}\sin2\pi n\Bigr]^{-1/2}.$$
    These coefficients form the solutions to our ODE:
    $$y(t)=\sum_{n=1}^{\infty}
    b_n\phi_n(t)$$
    where
    $$\phi_n(t)=k_n \sin n\pi t.$$

    \newpage

    \item For part ($a$) find the Laplace transform of:
    $$f(t)=\sinh at.$$ \\

    Firstly we have that:
    \begin{align*}
        e^{-st}\sinh at
        &=\frac{1}{2}e^{-st}
        \Bigl[e^{at}+e^{-at}\Bigr] \\
        &=\frac{1}{2}
        \Bigl[e^{-(s-a)t}
        -e^{-(s+a)t}\Bigr].
    \end{align*}
    Therefore:
    \begin{align*}
        \mathcal{L}[f(t)]
        &=\int_{0}^{\infty}e^{-st}
        \sinh(at)\dd t \\
        &=\int_{0}^{\infty}
        \frac{1}{2}
        \Bigl[e^{-(s-a)t}
        -e^{-(s+a)t}\Bigr]\dd t \\
        &=\lim_{T\rightarrow\infty}
        \Bigl[-\frac{1}{s-a}e^{-(s-a)t}
        +\frac{1}{s+a}e^{-(s+a)t}\Bigr]_{0}^{T} \\
        &=\frac{a}{s^2-a^2}. \\
    \end{align*}

    For part ($b$) show that:
    $$\mathcal{L}[t^n]=\frac{n!}{s^{n+1}}$$
    for all non-negative $n$. \\

    We proceed via induction. Let $n=1$:
    \begin{align*}
        \mathcal{L}[t]
        &=\int_{0}^{\infty}t e^{-st}\dd t \\
        &=\Bigl[-\frac{1}{s}te^{-st}
        -\frac{1}{s^2}e^{-st}\Bigr]_{0}^{\infty} \\
        &=\frac{1}{s^2}
    \end{align*}
    since an exponential grows faster than a linear one.

    \newpage

    We then assume that the following is true:
    $$\mathcal{L}[t^k]=\frac{k!}{s^{k+1}}$$
    and we want to show:
    $$\mathcal{L}[t^{k+1}]=\frac{(k+1)!}{s^{k+2}}.$$
    So integrating this by parts we get:
    \begin{align*}
        \mathcal{L}[t^{k+1}]
        &=\int_{0}^{\infty}t^{k+1}e^{-st}\dd t \\
        &=\Bigl[-\frac{1}{s}t^{k+1}e^{-st}\Bigr]_{0}^{\infty}
        +\frac{k+1}{s}\int_{0}^{\infty}t^k e^{-st}\dd t \\
        &=\frac{k+1}{s}\mathcal{L}[t^k] \\
        &=\frac{(k+1)!}{s^{k+2}}.
    \end{align*}
    Now verifying this for $k=0$:
    \begin{align*}
        \mathcal{L}[t^0]
        &=\int_{0}^{\infty}e^{-st}\dd t \\
        &=\Bigl[-\frac{1}{s}e^{-st}\Bigr]_{0}^{\infty} \\
        &=\frac{1}{s}. \\
    \end{align*}

    For part ($c$) solve the following ODE:
    $$y^{(4)}(t)=3\sinh 2t$$
    with initial conditions:
    $$y(0)=y^{(1)}(0)=y^{(2)}(0)=y^{(3)}(0)=1.$$ \\

    First begin by taking the Laplace transforms of both sides:
    $$\mathcal{L}[y^{(4)}]=\mathcal{L}[3\sinh 2t]
    =\frac{6}{s^2-4}.$$
    And our left hand side becomes:
    \begin{align*}
        \mathcal{L}[y^{(4)}]
        &=s^4\mathcal{L}[y(t)]-s^3 y^{(0)}(0)
        -s^2 y^{(1)}(0)-s y^{(2)}(0)-y^{(3)}(0) \\
        &=s^4\mathcal{L}[y(t)]-s^3-s^2-s-1.
    \end{align*}
    Equating and rearranging:
    \begin{align*}
        \mathcal{L}[y(t)]
        &=\frac{6}{s^4(s^2-4)}+\frac{1}{s}
        +\frac{1}{s^2}+\frac{1}{s^3}+\frac{1}{s^4} \\
        &=\frac{1}{s}+\frac{5}{8}\frac{1}{s^2}
        +\frac{1}{s^3}+-\frac{1}{2}\frac{1}{s^4}
        +\frac{3}{16}\frac{2}{s^2-4}.
    \end{align*}
    Since we have the following standard transforms:
    $$\frac{1}{s}=\mathcal{L}[1],\hspace{0.1in}
    \frac{1}{s^2}=\mathcal{L}[t],\hspace{0.1in}
    \frac{1}{s^3}=\frac{1}{2}\mathcal{L}[t^2],\hspace{0.1in}
    \frac{1}{s^4}=\frac{1}{6}\mathcal{L}[t^3]$$
    and
    $$\frac{2}{s^2-4}=\mathcal{L}[\sinh 2t]$$
    we have that by inspection:
    $$y(t)=1+\frac{5}{8}t+\frac{1}{2}t^2
    -\frac{1}{12}t^3+\frac{3}{16}\sinh 2t.$$

\end{enumerate}