\pagestyle{fancy}
\fancyhead{}
\fancyhead[L]{EM S1 Handin 2}
\fancyhead[R]{Winter 2023}

\section{Handin 2}
\begin{enumerate}
    \item For part ($i$) consider \underline{uniformly} charged
    region bounded by two spheres, of radius $b$ and $a$
    where $b>a$. Let the charge density of this region be $\rho$.
    Find its potential $\phi(\boldsymbol{r})$. \\

    Let $\boldsymbol{r}=r\boldsymbol{e}_r
    +\theta\boldsymbol{e}_{\theta}
    +\phi\boldsymbol{e}_{\phi}$
    where $r\in(a,b)$, $\theta\in[0,\pi]$
    and $\phi\in[0,2\pi]$
    be our parametrisation of the region.
    We have Gauss's law:
    $$\int_S\boldsymbol{E}\cdot\dd\boldsymbol{S}
    =\frac{Q_{enc}}{\epsilon_0}.$$
    Because of spherical symmetry,
    our electric field is of only radial form:
    $$\boldsymbol{E}(\boldsymbol{r})
    =E_r(r)\boldsymbol{e}_r.$$
    Evaluating our surface integral:
    \begin{align*}
        \int_S\boldsymbol{E}\cdot\dd\boldsymbol{S}
        &=\iint\boldsymbol{E}\cdot
        \left(\frac{\partial\boldsymbol{r}}{\partial\theta}\times
        \frac{\partial\boldsymbol{r}}{\partial\phi}\right)
        \dd\theta\dd\phi \\
        &=\iint E_r(r)\boldsymbol{e}_r\cdot
        (r^2\sin\theta\boldsymbol{e}_r)\dd\theta\dd\phi \\
        &=r^2E_r(r)\int_{\phi=0}^{\phi=2\pi}
        \int_{\theta=0}^{\theta=\pi}
        \sin\theta\dd\theta\dd\phi \\
        &=4\pi r^2E_r(r).
    \end{align*}
    Now the total charge enclosed is $Q_{enc}=\rho V$
    where $V$ is enclosed volume.
    $$\therefore 4\pi r^2E_r(r)=
    \left\{
	\begin{array}{ll}
        \displaystyle\frac{\rho}{\epsilon_0}\frac{4}{3}\pi
        (b^3-a^3) 
        & \mbox{$r>b$} \\\\
		\displaystyle\frac{\rho}{\epsilon_0}
        \frac{4}{3}\pi(r^3-a^3) 
        & \mbox{$r\in[a,b]$} \\\\
		0 
        & \mbox{$r<a$}
	\end{array}
    \right.$$
    Then rearranging:
    $$E_r(r)=
    \left\{
	\begin{array}{ll}
        \displaystyle\frac{1}{3}
        \frac{\rho}{\epsilon_0}(b^3-a^3)
        \frac{1}{r^2}
        & \mbox{$r>b$} \\\\
		\displaystyle\frac{1}{3}
        \frac{\rho}{\epsilon_0}
        \left(r-\frac{a^3}{r^2}\right)
        & \mbox{$r\in[a,b]$} \\\\
		0 
        & \mbox{$r<a$.}
	\end{array}
    \right.$$

    \newpage

    Since $\boldsymbol{E}=
    -\boldsymbol{\nabla}\phi
    (\boldsymbol{r})$ we then have that
    $E_r(r)=-\displaystyle
    \frac{\partial\phi}{\partial r}$
    and:
    $$\phi(r)=-\int E_r(r)\dd r.$$
    So when $r\in[a,b]$:
    \begin{align*}
        \phi(r)
        &=-\int\frac{\rho}{\epsilon_0}
        \frac{1}{3}(r-a^3 r^{-2})\dd r \\
        &=-\frac{1}{3}\frac{\rho}{\epsilon_0}
        \Bigl[\frac{r^2}{2}+\frac{a^3}{r}\Bigr]+C_1
    \end{align*}
    and when $r>b$:
    \begin{align*}
        \phi(r)
        &=-\frac{1}{3}
        \frac{\rho}{\epsilon_0}
        (b^3-a^3)
        \int r^{-2}\dd r \\
        &=\frac{1}{3}\frac{\rho}
        {\epsilon_0}(b^3-a^3)
        \frac{1}{r}+C_2
    \end{align*}
    where $C_2=0$ so that $\phi\rightarrow0$.
    For continuity, $C_1=\displaystyle
    \frac{1}{2}\frac{\rho}{\epsilon_0}b^2$.
    $$\therefore\phi(r)=
    \left\{
	\begin{array}{ll}
        \displaystyle\frac{1}{3}\frac{\rho}
        {\epsilon_0}(b^3-a^3)
        \frac{1}{r}
        & \mbox{$r>b$} \\\\
		\displaystyle-\frac{1}{3}\frac{\rho}{\epsilon_0}
        \Bigl[\frac{r^2}{2}+\frac{a^3}{r}\Bigr]
        +\frac{1}{2}\frac{\rho}{\epsilon_0}b^2
        & \mbox{$r\in[a,b]$} \\\\
		0 
        & \mbox{$r<a$.}
	\end{array}
    \right.$$


    For part ($ii$):
    \begin{align*}
        E_r(r)
        &=\frac{1}{3}
        \frac{\rho}{\epsilon_0}
        \left(r-\frac{a^3}{r^2}\right) \\
        &=\frac{1}{3}
        \frac{\rho}{\epsilon_0}\frac{1}{r^2}
        (r^3-a^3) \\
        &=\frac{\rho}{\epsilon_0}(r-a)
        \frac{1}{3r^2}(r^2+a^2+ra) \\
        &=\frac{\rho}{\epsilon_0}(b-a)
        \frac{1}{3b^2}(b^2+b^2+rb)
    \end{align*}
    but since $b\rightarrow a$ we then have that:
    $$E_r(r)=\frac{\rho}{\epsilon_0}(b-a)
    =\frac{\sigma}{\epsilon_0}.$$
    
    \newpage

    \item For part ($i$)
    a dipole at $\boldsymbol{r}_1$
    with moment $\boldsymbol{p}_1$
    generates an electric field:
    $$\boldsymbol{E}_1(\boldsymbol{r})=\frac{1}{4\pi\epsilon_0}
    \frac{3(\boldsymbol{p}_1\cdot(\boldsymbol{r}
    -\boldsymbol{r}_1))(\boldsymbol{r}-\boldsymbol{r}_1)
    -|\boldsymbol{r}-\boldsymbol{r}_1|^2\boldsymbol{p}_1}
    {|\boldsymbol{r}-\boldsymbol{r}_1|^5}$$
    and so at point $\boldsymbol{r}_2$ we have:
    $$\boldsymbol{E}_1(\boldsymbol{r}_2)=\frac{1}{4\pi\epsilon_0}
    \frac{3(\boldsymbol{p}_1\cdot(\boldsymbol{r}_2
    -\boldsymbol{r}_1))(\boldsymbol{r}_2-\boldsymbol{r}_1)
    -|\boldsymbol{r}_2-\boldsymbol{r}_1|^2\boldsymbol{p}_1}
    {|\boldsymbol{r}_2-\boldsymbol{r}_1|^5}.$$ \\

    For part ($ii$) the force induced by our dipole at $\boldsymbol{r}_1$
    with moment $\boldsymbol{p}_1$ on \underline{another} dipole 
    at $\boldsymbol{r}_2$ with moment $\boldsymbol{p}_2$ is:
    \begin{align*}
        \boldsymbol{F}_2(\boldsymbol{r}_2)
        &=\frac{1}{4\pi\epsilon_0}
        \Bigl[-\frac{15}{|\boldsymbol{r}_2-\boldsymbol{r}_1|^7}
        \bigl(\boldsymbol{p}_1\cdot(\boldsymbol{r}_2-\boldsymbol{r}_1)\bigr)
        \bigl(\boldsymbol{p}_2\cdot(\boldsymbol{r}_2-\boldsymbol{r}_1)\bigr)
        (\boldsymbol{r}_2-\boldsymbol{r}_1) \\
        &\quad+\frac{3}{|\boldsymbol{r}_2-\boldsymbol{r}_1|^5}
        \Bigl((\boldsymbol{p}_1\cdot\boldsymbol{p}_2)
        (\boldsymbol{r}_2-\boldsymbol{r}_1)
        +\Bigl(\boldsymbol{p}_1\cdot
        (\boldsymbol{r}_2-\boldsymbol{r}_1)\Bigr)
        \boldsymbol{p}_2 \\
        &\quad+\Bigl(\boldsymbol{p}_2\cdot
        (\boldsymbol{r}_2-\boldsymbol{r}_1)\Bigr)
        \boldsymbol{p}_1
        \Bigr)\Bigr]. \\
    \end{align*}

    For part ($iii$) find the torque $\boldsymbol{G}_1$
    on dipole $\boldsymbol{p}_1$ at $\boldsymbol{r}_1$
    due to the electric field $\boldsymbol{E}_2$ generated by
    dipole $\boldsymbol{p}_2$ at $\boldsymbol{r}_2$. \\

    So we have that:
    $$\boldsymbol{G}_1(\boldsymbol{r}_1)
    =\boldsymbol{p}_1\times
    \boldsymbol{E}_2(\boldsymbol{r}_1)$$
    where
    $$\boldsymbol{E}_2(\boldsymbol{r}_1)=\frac{1}{4\pi\epsilon_0}
    \frac{3(\boldsymbol{p}_2\cdot(\boldsymbol{r}_1
    -\boldsymbol{r}_2))(\boldsymbol{r}_1-\boldsymbol{r}_2)
    -|\boldsymbol{r}_1-\boldsymbol{r}_2|^2\boldsymbol{p}_2}
    {|\boldsymbol{r}_1-\boldsymbol{r}_2|^5}$$
    and evaluating this expression:
    \begin{align*}
        \boldsymbol{G}_1(\boldsymbol{r}_1)
        &=\frac{1}{4\pi\epsilon_0}
        \frac{3\bigl(\boldsymbol{p}_2
        \cdot(\boldsymbol{r}_1-\boldsymbol{r}_2)\bigr)
        \boldsymbol{p}_1\times
        (\boldsymbol{r}_1-\boldsymbol{r}_2)
        -|\boldsymbol{r}_1-\boldsymbol{r}_2|^2
        \hspace{0.05in}\boldsymbol{p}_1\times\boldsymbol{p}_2}
        {|\boldsymbol{r}_1-\boldsymbol{r}_2|^5}.
    \end{align*}

    \newpage

    For part ($iv$) find the torque $\boldsymbol{G}_2$
    on dipole $\boldsymbol{p}_2$ at $\boldsymbol{r}_1$
    from the electric field $\boldsymbol{E}_1$
    generated by $\boldsymbol{p}_1$ at $\boldsymbol{r}_1$. \\
    
    Now the torque on $\boldsymbol{p}_2$ at $\boldsymbol{r}_2$ is then:
    \begin{align*}
        \boldsymbol{G}_2(\boldsymbol{r}_2)
        &=\tau_{+q}+\tau_{-q} \\
        &=q(\boldsymbol{0}+\boldsymbol{d})\times
        \boldsymbol{E}_1(\boldsymbol{r}_2+\boldsymbol{d})
        -q\boldsymbol{0}\times\boldsymbol{E}_1(\boldsymbol{r}_2) \\
        &=q\boldsymbol{d}\times
        \Bigl[\boldsymbol{E}_1(\boldsymbol{r}_2)
        +(\boldsymbol{d}\cdot\boldsymbol{\nabla})
        \boldsymbol{E}_1(\boldsymbol{r}_2)
        +\dots\Bigr] \\
        &\approx q\boldsymbol{d}\times\boldsymbol{E}_1(\boldsymbol{r}_2)
        +\boldsymbol{d}\times(q\boldsymbol{d}\cdot\boldsymbol{\nabla})
        \boldsymbol{E}_1(\boldsymbol{r}_2) \\
        &=\boldsymbol{p}_2\times\boldsymbol{E}_1(\boldsymbol{r}_2)
        +\boldsymbol{d}\times(\boldsymbol{p}_2\cdot\boldsymbol{\nabla})
        \boldsymbol{E}_1(\boldsymbol{r}_2) \\
        &=\boldsymbol{p}_2\times\boldsymbol{E}_1(\boldsymbol{r}_2)
        +\boldsymbol{d}\times\boldsymbol{F}_2(\boldsymbol{r}_2)
    \end{align*}
    where dipole $\boldsymbol{p}_2=q\boldsymbol{d}$ has $-q$ charge at $\boldsymbol{r}_2$
    and $+q$ charge at $\boldsymbol{r}_2+\boldsymbol{d}$ \\
    with the limit $\boldsymbol{d}\rightarrow\boldsymbol{0}$.

    In the dipole limit
    our two dipoles $\boldsymbol{p}_1$ and $\boldsymbol{p}_2$
    are infinitely close. \\
    There probably exists dipole interactions between these two dipoles. \\
    So let $\boldsymbol{r}_1=\boldsymbol{r}_2-\boldsymbol{d}$,
    and in the limit $\boldsymbol{d}\rightarrow\boldsymbol{0}$ we get:
    $$\boldsymbol{G}_2(\boldsymbol{r}_1)
    =\boldsymbol{p}_2\times\boldsymbol{E}_1(\boldsymbol{r}_2)
    +(\boldsymbol{r}_2-\boldsymbol{r}_1)
    \times\boldsymbol{F}_2(\boldsymbol{r}_2)$$
    where $\boldsymbol{r}_2\rightarrow\boldsymbol{r}_1$. \\

    Finally for part ($v$) verify that:
    $$\boldsymbol{G}_2(\boldsymbol{r}_1)
    =-\boldsymbol{G}_1(\boldsymbol{r}_1).$$ \\

    Using previous parts we have that:
    $$\boldsymbol{p}_2\times\boldsymbol{E}_1(\boldsymbol{r}_2)
    =\frac{1}{4\pi\epsilon_0}
    \frac{3\bigl(\boldsymbol{p}_1
    \cdot(\boldsymbol{r}_2-\boldsymbol{r}_1)\bigr)
    \boldsymbol{p}_2\times
    (\boldsymbol{r}_2-\boldsymbol{r}_1)
    -|\boldsymbol{r}_2-\boldsymbol{r}_1|^2
    \hspace{0.05in}\boldsymbol{p}_2\times\boldsymbol{p}_1}
    {|\boldsymbol{r}_2-\boldsymbol{r}_1|^5}$$
    \begin{align*}
        (\boldsymbol{r}_2-\boldsymbol{r}_1)
        \times\boldsymbol{F}_2(\boldsymbol{r}_2)
        &=\frac{1}{4\pi\epsilon_0}
        \frac{3}{|\boldsymbol{r}_2-\boldsymbol{r}_1|^5}
        \Bigl[-\bigl(
        \boldsymbol{p}_1\cdot
        (\boldsymbol{r}_2-\boldsymbol{r}_1)\bigr)
        \boldsymbol{p}_2\times
        (\boldsymbol{r}_2-\boldsymbol{r}_1) \\
        &\quad+-
        \bigl(\boldsymbol{p}_2\cdot
        (\boldsymbol{r}_2-\boldsymbol{r}_1)\bigr)
        \boldsymbol{p}_1\times
        (\boldsymbol{r}_2-\boldsymbol{r}_1)
        \Bigr]
    \end{align*}
    Adding these two up:
    \begin{align*}
        \boldsymbol{G}_2(\boldsymbol{r}_1)
        &=-\frac{1}{4\pi\epsilon_0}
        \frac{3\bigl(\boldsymbol{p}_2
        \cdot(\boldsymbol{r}_2-\boldsymbol{r}_1)\bigr)
        \boldsymbol{p}_1\times
        (\boldsymbol{r}_2-\boldsymbol{r}_1)
        +|\boldsymbol{r}_2-\boldsymbol{r}_1|^2
        \hspace{0.05in}\boldsymbol{p}_2\times\boldsymbol{p}_1}
        {|\boldsymbol{r}_2-\boldsymbol{r}_1|^5} \\
        &=-\boldsymbol{G}_1(\boldsymbol{r}_1).
    \end{align*}

    \newpage

    \item For part ($i$) using Gauss's law:
    \begin{align*}
        \int_S\boldsymbol{E}\cdot
        \dd\boldsymbol{S}
        &=\iint E_{\rho}(\rho)
        \cdot
        \Bigl(\frac{\partial\boldsymbol{r}}
        {\partial\phi}
        \times
        \frac{\partial\boldsymbol{r}}
        {\partial z}\Bigr)
        \dd\phi\dd z \\
        &=E_{\rho}(\rho)\int_z
        \int_{0}^{2\pi}\rho\dd\phi\dd z \\
        &=z E_{\rho}(\rho)\rho 2\pi
    \end{align*}
    and when $\rho\in(a,b)$ this is equivalent to:
    $$z E_{\rho}(\rho)\rho 2\pi
    =-\frac{\lambda}{\epsilon_0}z.$$
    When $\rho\notin(a,b)$ choose spherical shell
    with radius $\rho$. By definition it has no charge
    and hence:
    $$E_{\rho}(\rho)=0.$$
    So we have that:
    $$E_{\rho}(\rho)=
    \left\{
	\begin{array}{ll}
		\displaystyle-\frac{\lambda}
        {2\pi\epsilon_0}\frac{1}{\rho}
        & \mbox{$\rho\in(a,b)$} \\\\
		0 
        & \mbox{$\rho\notin(a,b)$.}
	\end{array}
    \right.$$ \\

    For part ($ii$) since $\boldsymbol{E}=
    -\boldsymbol{\nabla}\phi
    (\boldsymbol{r})$ integrating gives:
    $$\phi(\rho)=
    \left\{
	\begin{array}{ll}
		\displaystyle\frac{\lambda}
        {2\pi\epsilon_0}\ln{\rho}
        & \mbox{$\rho\in(a,b)$} \\\\
		0 
        & \mbox{$\rho\notin(a,b)$}
	\end{array}
    \right.$$
    The potential difference between the
    two plates is
    \begin{align*}
        V
        &=\phi(b)-\phi(a) \\
        &=\frac{\lambda}
        {2\pi\epsilon_0}\ln{\frac{b}{a}}
    \end{align*}
    and since $Q=\lambda$:
    \begin{align*}
        C
        &=\frac{Q}{V} \\
        &=\frac{2\pi\epsilon_0}{\ln b/a}.
    \end{align*}

    \newpage

    For part ($iii$) the energy per unit length is:
    \begin{align*}
        W
        &=\frac{Q}{2}(\phi_b-\phi_a) \\
        &=\frac{\lambda^2}
        {4\pi\epsilon_0}\ln\frac{b}{a}. \\
    \end{align*}

    For part ($iv$):
    \begin{align*}
        W
        &=\frac{\epsilon_0}{2}
        \int\dd V |\boldsymbol{E}(\boldsymbol{r})|^2 \\
        &=\frac{\epsilon_0}{2}
        \frac{\lambda^2}{4\pi^2\epsilon_0}
        \int_z\int_{0}^{2\pi}\int_{a}^{b}
        \frac{1}{\rho^2}|\rho|
        \dd\rho\dd\phi\dd z \\
        &=\frac{\lambda^2}
        {4\pi\epsilon_0}\ln\frac{b}{a}. \\
    \end{align*}

    For part ($v$) show that when
    $b$ is slightly larger than $a$ we have that:
    $$C=\frac{2\pi\epsilon_0}{\ln b/a}
    \approx \frac{2\pi a\epsilon_0}{d}$$
    where $a$ is the radius of a cylinder
    and $d$ is plate separation distance.

    Here capacitance in per unit length.
    Let $b-a=\epsilon$ where $\epsilon$
    is a small quantity. Then consider the following:
    \begin{align*}
        \ln\frac{b}{a}
        &=\ln\left(1+\frac{\epsilon}{a}\right) \\
        &\approx\frac{\epsilon}{a}
    \end{align*}
    and
    \begin{align*}
        \frac{2\pi\epsilon_0}{\ln b/a}
        &\approx\frac{2\pi\epsilon_0}{\frac{\epsilon}{a}} \\
        &=\frac{2\pi a\epsilon_0}{b-a}
    \end{align*}
    which is an expression for the capacitance of
    a thinly separated parallel-plate capacitor.
\end{enumerate}