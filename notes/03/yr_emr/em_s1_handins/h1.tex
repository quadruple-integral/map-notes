\pagestyle{fancy}
\fancyhead{}
\fancyhead[L]{EM S1 Handin 1}
\fancyhead[R]{Winter 2023}

\section{Handin 1}
\begin{enumerate}
    \item Consider set $T_{ijk}$ with $3^3$ elements, satisfying:
    $$v_i=T_{ijk}R_{jk}$$
    where $v_i$ is a vector and $R_{jk}$ a rank 2 tensor. 
    
    Show that $T_{ijk}$ is a rank 3 tensor. \\

    \begin{proof}
    Direct proof.
    
    Firstly define $T'_{ijk}$ in frame $S'$ and $T_{ijk}$ in frame $S$, where our frames are related by $\boldsymbol{e}'_i=\ell_{ij}\boldsymbol{e}_j$.
    We then have that:
    \begin{align*}
        v'_i
        &=T'_{ijk}R'_{jk} \\
        &=T'_{ijk}\ell_{jl}\ell_{km}R_{lm}
    \end{align*}
    and
    \begin{align*}
        v'_i
        &=\ell_{ij}v_j \\
        &=\ell_{ij}T_{jkl}R_{kl}.
    \end{align*}
    $$\therefore T'_{ijk}\ell_{jl}\ell_{km}R_{lm}
    =\ell_{ij}T_{jkl}R_{kl}$$
    Using the fact that $R_{lm}$ is a tensor, we multiply both sides by vector $a_m$:
    $$T'_{ijk}\ell_{jl}\ell_{km}R_{lm}a_m
    =\ell_{ij}T_{jkl}R_{kl}a_m.$$
    The left hand side:
    \begin{align*}
        T'_{ijk}\ell_{jl}\ell_{km}R_{lm}a_m
        &=T'_{ijk}\ell_{jl}\ell_{km}a_l \\
        &=T'_{ijk}\ell_{jl}\ell_{km}\delta_{kl}a_k.
    \end{align*}
    The right hand side:
    \begin{align*}
        \ell_{ij}T_{jkl}R_{kl}a_m
        &=\ell_{ij}T_{jkl}R_{kl}a_l\delta_{lm} \\
        &=\ell_{ij}T_{jkl}\delta_{lm}a_k.
    \end{align*}
    Since equality still holds:
    $$T'_{ijk}\ell_{jl}\ell_{km}\delta_{kl}a_k
    =\ell_{ij}T_{jkl}\delta_{lm}a_k.$$
    $$\therefore\Bigl(T'_{ijk}\ell_{jl}\ell_{km}\delta_{kl}
    -\ell_{ij}T_{jkl}\delta_{lm}\Bigl)a_k=0$$
    Because $a_k$ is a vector that is not always zero:
    $$T'_{ijk}\ell_{jl}\ell_{km}\delta_{kl}=
    \ell_{ij}T_{jkl}\delta_{lm}.$$
    $$\therefore T'_{ijk}\ell_{jk}\ell_{km}=
    \ell_{ij}T_{jkm}$$

    \newpage

    Then multiply both sides by $\ell_{in}$:
    $$T'_{ijk}\ell_{jk}\ell_{km}\ell_{in}=
    \ell_{ij}\ell_{in}T_{jkm}.$$
    $$\therefore T'_{ijk}\ell_{jk}\ell_{km}\ell_{in}=
    \delta_{jn}T_{jkm}$$
    $$\therefore T'_{ijk}\ell_{jk}\ell_{km}\ell_{in}=T_{nkm}$$
    $$\therefore T'_{ijk}\ell_{in}\ell_{jk}\ell_{km}=T_{nkm}$$
    And this is by definition a third rank tensor.
    \end{proof}
    
    \newpage

    \item Define
    $$\dd\boldsymbol{r}_i=\frac{\partial\boldsymbol{r}}
    {\partial u_i}\dd u_i$$
    and the volume of the infinitesimal parallelepiped with $\dd\boldsymbol{r}_1$,
    $\dd\boldsymbol{r}_2$ and $\dd\boldsymbol{r}_3$:
    $$\dd V=|\dd\boldsymbol{r}_1\cdot
    (\dd\boldsymbol{r}_2\times\dd\boldsymbol{r}_3)|.$$

    For part ($i$) show that:
    $$\dd V=|J|\hspace{0.03in}\dd u_1\hspace{0.03in}\dd u_2\hspace{0.03in}\dd u_3$$
    for $J=\det M$ where:
    $$M_{ij}=\frac{\partial x_i}{\partial u_j}.$$ \\

    Using our definition of $\dd\boldsymbol{r}_i$:
    \begin{align*}
    \dd V
    &=|\dd\boldsymbol{r}_1\cdot(\dd\boldsymbol{r}_2\times\dd\boldsymbol{r}_3)| \\
    &=|\frac{\partial\boldsymbol{r}}{\partial u_1}\dd u_1\cdot
    (\frac{\partial\boldsymbol{r}}{\partial u_2}\dd u_2\times
    \frac{\partial\boldsymbol{r}}{\partial u_3}\dd u_3)| \\
    &=|\frac{\partial\boldsymbol{r}}{\partial u_1}\cdot
    (\frac{\partial\boldsymbol{r}}{\partial u_2}\times\frac{\partial\boldsymbol{r}}{\partial u_3})|
    \hspace{0.03in}
    \dd u_1\hspace{0.03in}\dd u_2\hspace{0.03in}\dd u_3
    \end{align*}
    Since $\boldsymbol{r}=x_i\boldsymbol{e}_i$:
    $$\frac{\partial\boldsymbol{r}}{\partial u_j}
    =\frac{\partial x_i}{\partial u_j}\boldsymbol{e}_i.$$
    Now the triple scalar product of three vectors is equivalent
    to the determinant of a matrix consisting of these three vectors,
    as either rows or columns. Therefore:
    \begin{align*}
        |\frac{\partial\boldsymbol{r}}{\partial u_1}\cdot
        (\frac{\partial\boldsymbol{r}}{\partial u_2}\times
        \frac{\partial\boldsymbol{r}}{\partial u_3})|
        &=\det
        \begin{bmatrix}
            \displaystyle\frac{\partial x_1}{\partial u_1} & \displaystyle\frac{\partial x_1}{\partial u_2}
            & \displaystyle\frac{\partial x_1}{\partial u_3} \\
            \displaystyle\frac{\partial x_2}{\partial u_1} & \displaystyle\frac{\partial x_2}{\partial u_2}
            & \displaystyle\frac{\partial x_2}{\partial u_3} \\
            \displaystyle\frac{\partial x_3}{\partial u_1} & \displaystyle\frac{\partial x_3}{\partial u_2}
            & \displaystyle\frac{\partial x_3}{\partial u_3}
        \end{bmatrix} \\
        &=\det M \\
        &=J.
    \end{align*}
    Since volume is nonnegative:
    $$\therefore\dd V=|J|\hspace{0.03in}\dd u_1\hspace{0.03in}\dd u_2\hspace{0.03in}\dd u_3.$$
    
    \newpage

    For part ($ii$) show:
    \begin{itemize}
        \item $(M^T M)_{ij}=g_{ij}$ for $g_{ij}$ is the metric tensor.

        \item $\dd V=\sqrt{g}\hspace{0.03in}\dd u_1\hspace{0.03in}\dd u_2\hspace{0.03in}\dd u_3$
        for $g_{ij}=(G)_{ij}$ and $g=\det G$. \\
    \end{itemize}

    By the definition of the metric tensor:
    \begin{align*}
        g_{ij}
        &=\frac{\partial x_k}{\partial u_i}
        \frac{\partial x_k}{\partial u_j} \\
        &=M_{ki}M_{kj} \\
        &=(M^T)_{ik}(M)_{kj} \\
        &=(M^T M)_{ij}.
    \end{align*}
    Since $G=M^T M$ taking the determinants gives:
    \begin{align*}
        \det G
        &=\det (M^T M) \\
        &=\det M^T\det M \\
        &=(\det M)^2 \\
        &=J^2.
    \end{align*}
    Then from part ($i$):
    $$\therefore J=\pm\sqrt{g}$$
    \begin{align*}
        \therefore\dd V
        &=|J|\hspace{0.03in}\dd u_1\hspace{0.03in}\dd u_2\hspace{0.03in}\dd u_3 \\
        &=\sqrt{g}\hspace{0.03in}\dd u_1\hspace{0.03in}\dd u_2\hspace{0.03in}\dd u_3
    \end{align*} \\

    \newpage
    
    For part ($iii$) show that given \underline{orthogonal} curvilinear coordinates we have:
    $$\dd V=h_1\hspace{0.03in}h_2\hspace{0.03in}h_3
    \hspace{0.03in}\dd u_1\hspace{0.03in}\dd u_2\hspace{0.03in}\dd u_3$$ \\

    For OCCs, $\boldsymbol{e}_3=\boldsymbol{e}_1\times\boldsymbol{e}_2$ and therefore:
    \begin{align*}
    \dd V
    &=|\dd\boldsymbol{r}_1\cdot(\dd\boldsymbol{r}_2\times\dd\boldsymbol{r}_3)| \\
    &=|\frac{\partial\boldsymbol{r}}{\partial u_1}\dd u_1\cdot
    (\frac{\partial\boldsymbol{r}}{\partial u_2}\dd u_2\times
    \frac{\partial\boldsymbol{r}}{\partial u_3}\dd u_3)| \\
    &=|\boldsymbol{e}_1\cdot(\boldsymbol{e}_2\times\boldsymbol{e}_3)|
    \hspace{0.03in}h_1\hspace{0.03in}h_2\hspace{0.03in}h_3\hspace{0.03in}
    \dd u_1\hspace{0.03in}\dd u_2\hspace{0.03in}\dd u_3 \\
    &=|\boldsymbol{e}_1\cdot\boldsymbol{e}_1|
    \hspace{0.03in}h_1\hspace{0.03in}h_2\hspace{0.03in}h_3\hspace{0.03in}
    \dd u_1\hspace{0.03in}\dd u_2\hspace{0.03in}\dd u_3 \\
    &=h_1\hspace{0.03in}h_2\hspace{0.03in}h_3\hspace{0.03in}
    \dd u_1\hspace{0.03in}\dd u_2\hspace{0.03in}\dd u_3
    \end{align*}
    where $h_i=|\displaystyle\frac{\partial\boldsymbol{r}}{\partial u_i}|$
    and $\dd u_i=\displaystyle\frac{1}{h_i}
    \displaystyle\frac{\partial\boldsymbol{r}}{\partial u_i}$.
    \newpage

    \item For the first part of this problem we need to
    find the electric field $E(x)$ generated our rod of length $2a$
    centered at $x=0$ with total charge $Q$.
    
    The thin rod is also assumed to have \underline{uniform} charge density of $\rho$.
    
    Then we can use the formula $F(x)=qE(x)$
    to find force on our point charge $q$ at $x=R$. \\

    By Coulomb's law:
    \begin{align*}
        E(x)
        &=\int_{-a}^{a}\frac{\rho}{4\pi\epsilon_0}
        \frac{x-x'}{(x-x')^3}\dd x' \\
        &=\frac{\rho}{4\pi\epsilon_0}
        \left[\frac{1}{x-x'}\right]_{x'=-a}^{x'=a} \\
        &=\frac{\rho}{4\pi\epsilon_0}\frac{2a}{x^2-a^2}.
    \end{align*}
    Because we have a uniform charge density across $2a$:
    $$\rho=\frac{Q}{2a}$$
    then the electric field generated by the thin rod becomes:
    \begin{align*}
        E(x)
        &=\frac{1}{4\pi\epsilon_0}\frac{Q}{2a}\frac{2a}{x^2-a^2} \\
        &=\frac{Q}{4\pi\epsilon_0}\frac{1}{x^2-a^2}.
    \end{align*}
    The force on our point charge $q$ at $x=R$ is then:
    \begin{align*}
        F(R)
        &=qE(R) \\
        &=\frac{qQ}{4\pi\epsilon_0}\frac{1}{R^2-a^2}.
    \end{align*}
    
    \newpage

    The force on charge $q$ at $x=R$ by charge $Q$ at $x_1=0$ is given by:
    \begin{align*}
        F(x)
        &=F(R) \\
        &=\frac{qQ}{4\pi\epsilon_0}
        \frac{x-x_1}{|x-x_1|^3} \\
        &=\frac{qQ}{4\pi\epsilon_0}
        \frac{R}{|R|^3} \\
        &=\frac{qQ}{4\pi\epsilon_0}
        \frac{1}{R^2}.
    \end{align*}
    Comparing these two forces:
    $$F_{rod}=\frac{qQ}{4\pi\epsilon_0}\frac{1}{R^2-a^2}$$
    $$F_{point}=\frac{qQ}{4\pi\epsilon_0}\frac{1}{R^2},$$
    since
    $$\frac{1}{R^2-a^2}>\frac{1}{R^2}$$
    therefore $F_{rod}>F_{point}$.
    
    \newpage

    \item For part ($i$) show that:
    $$[T_i,T_j]=i\epsilon_{ijk}T_k$$ \\

    The Lie brackets are defined:
    $$[x,y]=xy-yx$$
    and so
    $$[T_i,T_j]
    =T_iT_j-T_jT_i.$$
    Because
    $$(T_k)_{ij}=-i\epsilon_{ijk}$$
    then we have:
    $$(T_i)_{lk}=-i\epsilon_{lki}$$
    $$(T_j)_{km}=-i\epsilon_{kmj}$$
    and swapping order gives:
    $$(T_j)_{lk}=-i\epsilon_{lkj}$$
    $$(T_i)_{km}=-i\epsilon_{kmi}.$$
    It is important to use more indices here:
    \begin{align*}
        \bigl(T_iT_j-T_jT_i\bigl)_{lm}
        &=(T_i)_{lk}(T_j)_{km}
        -(T_j)_{lk}(T_i)_{km} \\
        &=-\epsilon_{lki}\epsilon_{kmj}
        +\epsilon_{lkj}\epsilon_{kmi} \\
        &=-\epsilon_{ilk}\epsilon_{kmj}
        +\epsilon_{jlk}\epsilon_{kmi}.
    \end{align*}
    We first consider $-\epsilon_{ilk}\epsilon_{kmj}$. Because we have
    \begin{align*}
        \epsilon_{ijk}\epsilon_{klm}
        &=\delta_{il}\delta_{jm}-\delta_{im}\delta_{jl} \\
        &=\delta_{im}\delta_{lj}-\delta_{ij}\delta_{lm} \\
        &=\epsilon_{ilk}\epsilon_{kmj}
    \end{align*}
    where we swap $j\rightarrow l$, $l\rightarrow m$ and $m\rightarrow j$:
    $$\therefore-\epsilon_{ilk}\epsilon_{kmj}
    =-\bigl(\delta_{im}\delta_{lj}-\delta_{ij}\delta_{lm}\bigl)$$
    $$\therefore\epsilon_{jlk}\epsilon_{kmi}
    =\delta_{jm}\delta_{li}-\delta_{ij}\delta_{lm}$$

    \newpage

    Then:
    \begin{align*}
        \bigl(T_iT_j-T_jT_i\bigl)_{lm}
        &=-\epsilon_{ilk}\epsilon_{kmj}
        +\epsilon_{jlk}\epsilon_{kmi} \\
        &=-\bigl(\delta_{im}\delta_{lj}-\delta_{ij}\delta_{lm}\bigl)\hspace{0.05in}
        +\hspace{0.05in}\delta_{jm}\delta_{li}-\delta_{ij}\delta_{lm} \\
        &=-\delta_{im}\delta_{lj}+\delta_{jm}\delta_{li}.
    \end{align*}
    Due to the symmetry of the Kronecker delta:
    \begin{align*}
        \bigl(T_iT_j-T_jT_i\bigl)_{lm}
        &=-\delta_{im}\delta_{lj}+\delta_{jm}\delta_{li} \\
        &=\delta_{il}\delta_{jm}-\delta_{im}\delta_{jl} \\
        &=\epsilon_{ijk}\epsilon_{klm} \\
        &=\epsilon_{ijk}\epsilon_{lmk} \\
        &=-i\epsilon_{ijk}\cdot -i\epsilon_{lmk} \\
        &=-i\epsilon_{ijk}\cdot(T_k)_{lm}.
    \end{align*}
    $$\therefore [T_i,T_j]_{lm}=-i\epsilon_{ijk}\cdot(T_k)_{lm}$$
    $$\therefore [T_i,T_j]=-i\epsilon_{ijk}T_k$$
    
    \newpage

    For part ($ii$) we want to show:
    \begin{itemize}
        \item $T_i$ is hermitian

        \item $\Tr(T_iT_j)=2\delta_{ij}$ \\
    \end{itemize}

    The definition of a hermitian matrix is as such:
    $$H=(H^T)^*$$
    $$H_{ij}=(H^T)^*_{ij}$$
    where $*$ denotes the complex conjugate. So:
    $$(T_i)_{jk}=-i\epsilon_{ijk}.$$
    \begin{align*}
        \therefore (T^T_i)_{jk}
        &=(T_i)_{kj} \\
        &=-i\epsilon_{kji} \\
        &=-i\epsilon_{ikj} \\
        &=i\epsilon_{ijk}
    \end{align*}
    Then by the definition of the complex conjugate:
    $$\therefore (T^T_i)^*_{jk}=-i\epsilon_{ijk}.$$
    Since $(T_i)_{jk}=(T^T_i)^*_{jk}$ our matrix $T_i$ is hermitian.

    For the second part we firstly define:
    $$(T_i)_{lk}=-i\epsilon_{lki}$$
    and
    $$(T_j)_{km}=-i\epsilon_{kmj}.$$
    \begin{align*}
        \therefore(T_i)_{lk}(T_j)_{km}
        &=(T_iT_j)_{lm} \\
        &=-\epsilon_{lki}\epsilon_{kmj} \\
        &=-\epsilon_{ilk}\epsilon_{kmj}
    \end{align*}
    Now:
    \begin{align*}
        \epsilon_{ijk}\epsilon_{klm}
        &=\delta_{il}\delta_{jm}-\delta_{im}\delta_{jl} \\
        &=\delta_{im}\delta_{lj}-\delta_{ij}\delta_{lm} \\
        &=\epsilon_{ilk}\epsilon_{kmj}
    \end{align*}
    if we swap $j\rightarrow l$, $l\rightarrow m$ and $m\rightarrow j$.

    \newpage

    \begin{align*}
        \therefore(T_iT_j)_{lm}
        &=-\epsilon_{ilk}\epsilon_{kmj} \\
        &=-\bigl(\delta_{im}\delta_{lj}-\delta_{ij}\delta_{lm}\bigl)
    \end{align*}
    Taking the trace of a matrix is summing up its diagonals:
    \begin{align*}
        \Tr(T_iT_j)
        &=(T_iT_j)_{ll} \\
        &=-\bigl(\delta_{il}\delta_{lj}-\delta_{ij}\delta_{ll}\bigl) \\
        &=-\bigl(\delta_{ij}-3\delta_{ij}\bigl) \\
        &=2\delta_{ij}.
    \end{align*}

    \newpage

    Finally for part ($iii$) consider:
    $$R(\alpha,\boldsymbol{n})=\exp\bigl(-i\alpha\boldsymbol{n}
    \cdot\boldsymbol{T}\bigl)$$
    where $\alpha$ is our rotation angle about \underline{unit} axis vector $\boldsymbol{n}$.

    $\boldsymbol{T}$ is a vector of generator matrices.

    Our aims are:
    \begin{itemize}
        \item Show $\bigl(\boldsymbol{n}\cdot\boldsymbol{T}\bigl)^2_{ij}
        =\delta_{ij}-n_i n_j$.

        \item Show $\bigl(\boldsymbol{n}\cdot\boldsymbol{T}\bigl)^3_{ij}
        =\bigl(\boldsymbol{n}\cdot\boldsymbol{T}\bigl)_{ij}$.

        \item General formula for
        $\bigl(\boldsymbol{n}\cdot\boldsymbol{T}\bigl)^m_{ij}$
        where $m>3$.

        \item Expand $\exp\bigl(-i\alpha\boldsymbol{n}
        \cdot\boldsymbol{T}\bigl)$ as a power series.

        \item Recover standard rotation tensor of form:
        $$R_{ij}(\alpha,\boldsymbol{n})
        =\delta_{ij}\cos\alpha+n_i n_j\bigl(1-\cos\alpha\bigl)
        -\epsilon_{ijk}n_k\sin\alpha.$$ \\
    \end{itemize}

    Firstly define:
    $$(\boldsymbol{n}\cdot\boldsymbol{T})_{ik}
    =n_{\alpha}(T_{\alpha})_{ik}=n_{\alpha}\cdot -i\epsilon_{ik\alpha}$$
    $$(\boldsymbol{n}\cdot\boldsymbol{T})_{kj}
    =n_{\beta}(T_{\beta})_{kj}=n_{\beta}\cdot -i\epsilon_{kj\beta}$$
    then we have that
    \begin{align*}
        (\boldsymbol{n}\cdot\boldsymbol{T})^2_{ij}
        &=n_{\alpha}(T_{\alpha})_{ik}\cdot n_{\beta}(T_{\beta})_{kj} \\
        &=n_{\alpha}n_{\beta}\cdot -1\cdot\epsilon_{ik\alpha}\epsilon_{kj\beta}.
    \end{align*}
    Using the standard identity we get:
    \begin{align*}
        \epsilon_{ijk}\epsilon_{klm}
        &=\epsilon_{jki}\epsilon_{klm} \\
        &=\delta_{il}\delta_{jm}-\delta_{im}\delta_{jl} \\
        &=\delta_{\alpha j}\delta_{i\beta}-\delta_{\alpha\beta}\delta_{ij} \\
        &=\epsilon_{ik\alpha}\epsilon_{kj\beta}
    \end{align*}
    where $j\rightarrow i$, $i\rightarrow\alpha$, $l\rightarrow j$
    and $m\rightarrow\beta$.

    Then substituting back into our equation:
    \begin{align*}
        (\boldsymbol{n}\cdot\boldsymbol{T})^2_{ij}
        &=n_{\alpha}n_{\beta}\cdot -1\cdot\epsilon_{ik\alpha}\epsilon_{kj\beta} \\
        &=n_{\alpha}n_{\beta}\bigl(\delta_{\alpha\beta}\delta_{ij}
        -\delta_{\alpha j}\delta_{i\beta}\bigl) \\
        &=\delta_{ij}-n_i n_j
    \end{align*}

    \newpage

    Now we show that$\bigl(\boldsymbol{n}\cdot\boldsymbol{T}\bigl)^3_{ij}
    =\bigl(\boldsymbol{n}\cdot\boldsymbol{T}\bigl)_{ij}$. Define:
    $$(\boldsymbol{n}\cdot\boldsymbol{T})^2_{ik}
    =\delta_{ik}-n_i n_k$$
    $$(\boldsymbol{n}\cdot\boldsymbol{T})_{kj}
    =n_l(T_l)_{kj}=n_l\cdot -i\epsilon_{kjl}$$
    and so multiplying them together gives:
    \begin{align*}
        (\boldsymbol{n}\cdot\boldsymbol{T})^3_{ij}
        &=(\boldsymbol{n}\cdot\boldsymbol{T})^2_{ik}
        (\boldsymbol{n}\cdot\boldsymbol{T})_{kj} \\
        &=(\delta_{ik}-n_i n_k) n_l\cdot -i\epsilon_{kjl} \\
        &=-i n_l\bigl(\epsilon_{ijl}-n_i n_k\epsilon_{kjl}\bigl) \\
        &=-i n_l\epsilon_{ijl}+i n_l n_i n_k\epsilon_{kjl} \\
        &=-i n_l\epsilon_{ijl}+i n_i\delta_{lk}\epsilon_{kjl} \\
        &=-i n_l\epsilon_{ijl} \\
        &=(\boldsymbol{n}\cdot\boldsymbol{T})_{ij}.
    \end{align*}
    
    The general formula for $\bigl(\boldsymbol{n}\cdot\boldsymbol{T}\bigl)^m_{ij}$
    takes the form:
    $$\bigl(\boldsymbol{n}\cdot\boldsymbol{T}\bigl)^m_{ij} =
    \left\{
	\begin{array}{ll}
		\delta_{ij}-n_i n_j  & m\hspace{0.05in}\text{even} \\
		\bigl(\boldsymbol{n}\cdot\boldsymbol{T}\bigl)_{ij} & m\hspace{0.05in}\text{odd}.
	\end{array}
    \right.$$
    The power series for an exponential is:
    $$e^x=\sum_{k=0}^{\infty}\frac{x^k}{k!}$$
    and so:
    \begin{align*}
        \Bigl(\exp\bigl(-i\alpha\boldsymbol{n}\cdot\boldsymbol{T}\bigl)\Bigl)_{ij}
        &=\sum_{k=0}^{\infty}\Bigl(\frac{1}{k!}[-i\alpha]^k
        (\boldsymbol{n}\cdot\boldsymbol{T})^k_{ij}\Bigl) \\
        &=1+[-i\alpha](\boldsymbol{n}\cdot\boldsymbol{T})^k_{ij} \\
        &\quad+\sum_{k=2,4,\dots}\frac{1}{k!}[-i\alpha]^k
        (\delta_{ij}-n_i n_j) \\
        &\quad+\sum_{k=3,5,\dots}\frac{1}{k!}[-i\alpha]^k
        (\boldsymbol{n}\cdot\boldsymbol{T})_{ij}.
    \end{align*}
    Set $k=2n$ for the first sum and $k=2m+1$ for the second. Here $n,m\in\mathbb{N}$.

    \newpage

    \begin{align*}
        \therefore\Bigl(\exp\bigl(-i\alpha\boldsymbol{n}\cdot\boldsymbol{T}\bigl)\Bigl)_{ij}
        &=1+[-i\alpha](\boldsymbol{n}\cdot\boldsymbol{T})^k_{ij} \\
        &\quad+(\delta_{ij}-n_i n_j)\Bigl(\bigl(\sum_{n=0}^{\infty}(-1)^n
        \frac{\alpha^{2n}}{(2n)!}\bigl)-1\Bigl) \\
        &\quad+(\boldsymbol{n}\cdot\boldsymbol{T})_{ij}\cdot
        -i\Bigl(\bigl(\sum_{m=0}^{\infty}(-1)^m\frac{\alpha^{2m+1}}{(2m+1)!}
        \bigl)-\alpha\Bigl)
    \end{align*}
    Recognising this as the series expansion for cosine and sine:
    \begin{align*}
        \therefore\Bigl(\exp\bigl(-i\alpha\boldsymbol{n}\cdot\boldsymbol{T}\bigl)\Bigl)_{ij}
        &=1+[-i\alpha](\boldsymbol{n}\cdot\boldsymbol{T})^k_{ij} \\
        &\quad+(\delta_{ij}-n_i n_j)\Bigl(\cos\alpha-1\Bigl) \\
        &\quad+(\boldsymbol{n}\cdot\boldsymbol{T})_{ij}\cdot
        -i\Bigl(\sin\alpha-\alpha\Bigl)
    \end{align*}
    Since $(\boldsymbol{n}\cdot\boldsymbol{T})_{ij}=n_k(T_k)_{ij}=-i n_k\epsilon_{ijk}$:
    \begin{align*}
        \Bigl(\exp\bigl(-i\alpha\boldsymbol{n}\cdot\boldsymbol{T}\bigl)\Bigl)_{ij}
        &=1-\alpha n_k\epsilon_{ijk}
        +(\delta_{ij}-n_i n_j)\Bigl(\cos\alpha-1\Bigl)
        -n_k\epsilon_{ijk}\Bigl(\sin\alpha-\alpha\Bigl) \\
        &=\delta_{ij}\cos\alpha+n_i n_j\Bigl(1-\cos\alpha\Bigl)
        -\epsilon_{ijk}n_k\sin\alpha
    \end{align*}
    where we use the bogus argument:
    \begin{align*}
        1-\delta_{ij}
        &=\frac{1}{3}\delta_{ii}-\frac{1}{3}\delta_{jj}\delta_{ij} \\
        &=\frac{1}{3}\delta_{ii}-\frac{1}{3}\delta_{ii} \\
        &=0.
    \end{align*}
\end{enumerate}