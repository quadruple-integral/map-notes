\pagestyle{fancy}
\fancyhead{}
\fancyhead[L]{EM S1 Tutorial 8}
\fancyhead[R]{Winter 2023}

\section{Tutorial 8}
\begin{enumerate}
    \item Firstly consider a
    \underline{point dipole} with moment $\boldsymbol{p}$
    at position $\boldsymbol{r}$:
    $$\phi(\boldsymbol{r})=\frac{1}{4\pi\epsilon_0}
    \frac{\boldsymbol{p}\cdot\boldsymbol{r}}{r^3}.$$
    Show that it generates the following electric field:
    $$\boldsymbol{E}(\boldsymbol{r})
    =\frac{1}{4\pi\epsilon_0}
    \frac{3(\boldsymbol{p}\cdot\boldsymbol{r})
    \boldsymbol{r}-r^2\boldsymbol{p}}{r^5}.$$ \\

    Since $\boldsymbol{E}=-\boldsymbol{\nabla}
    \phi(\boldsymbol{r})$:
    \begin{align*}
        \boldsymbol{E}(\boldsymbol{r})
        &=-\frac{1}{4\pi\epsilon_0}\boldsymbol{\nabla}
        \left(\frac{\boldsymbol{p}\cdot\boldsymbol{r}}{r^3}\right) \\
        &=-\frac{1}{4\pi\epsilon_0}
        \left(\boldsymbol{\nabla}(\boldsymbol{p}\cdot\boldsymbol{r})\frac{1}{r^3}
        +(\boldsymbol{p}\cdot\boldsymbol{r})
        \boldsymbol{\nabla}\left(\frac{1}{r^3}\right)\right).
    \end{align*}
    Let's separately deal with the dels:
    \begin{align*}
        \boldsymbol{\nabla}(\boldsymbol{p}\cdot\boldsymbol{r})
        &=\boldsymbol{e}_i\frac{\partial}{\partial x_i}
        (p_j x_j) \\
        &=\boldsymbol{e}_i p_j\delta_{ij} \\
        &=\boldsymbol{p}
    \end{align*}
    and
    \begin{align*}
        \boldsymbol{\nabla}\left(\frac{1}{r^3}\right)
        &=\boldsymbol{e}_i\frac{\partial}{\partial x_i}
        \Bigl[(x_j^2)^{-3/2}\Bigr] \\
        &=\boldsymbol{e}_i\left(
        -\frac{3}{2}(x_j^2)^{-5/2}
        \cdot 2x_j\frac{\partial x_j}{\partial x_i}
        \right) \\
        &=-3r^{-5}\boldsymbol{r}.
    \end{align*}
    Combining these yields us the desired electric field.
\end{enumerate}