\pagestyle{fancy}
\fancyhead{}
\fancyhead[L]{EM S1 Tutorials}
\fancyhead[R]{Winter 2023}

\section{Tutorial 7}
\begin{enumerate}
    \item For part ($i$), we are asked to find $\boldsymbol{\nabla}(\boldsymbol{c}\cdot\boldsymbol{r})^n$.

    This is the \textbf{gradient} operator and acts \underline{only} on scalar fields $\phi(\boldsymbol{r})$. Note:
    $$\boldsymbol{\nabla}\phi(\boldsymbol{r})=\frac{\partial\phi}{\partial x_i}\boldsymbol{e}_i,$$
    where $\boldsymbol{r}=x_i\boldsymbol{e}_i$. Using the chain rule we get:
    \begin{align*}
        \boldsymbol{\nabla}(\boldsymbol{c}\cdot\boldsymbol{r})^n
        &= n(\boldsymbol{c}\cdot\boldsymbol{r})^{n-1}
        \boldsymbol{\nabla}(\boldsymbol{c}\cdot\boldsymbol{r}) \\
        &= n\boldsymbol{c}(\boldsymbol{c}\cdot\boldsymbol{r})^{n-1}.
    \end{align*}
    Our solution makes sense since we obtain a vector quantity. Now however our question asks us to
    show this using suffix notation:
    \begin{align*}
        \boldsymbol{\nabla}(\boldsymbol{c}\cdot\boldsymbol{r})^n
        &=\boldsymbol{e}_i\frac{\partial}{\partial x_i}(\boldsymbol{c}\cdot\boldsymbol{r})^n \\
        &=n(\boldsymbol{c}\cdot\boldsymbol{r})^{n-1}
        \boldsymbol{e}_i\frac{\partial}{\partial x_i}(c_j x_j) \\
        &=n(\boldsymbol{c}\cdot\boldsymbol{r})^{n-1}
        \boldsymbol{e}_i c_j\delta_{ij} \\
        &=n(\boldsymbol{c}\cdot\boldsymbol{r})^{n-1}
        \boldsymbol{c}.
    \end{align*}

    For part ($ii$) we want $\boldsymbol{\nabla}r^n$.

    Firstly we need to prove an important result, namely that:
    $$\boldsymbol{\nabla}r=\frac{1}{r}\boldsymbol{r},$$
    where $r=|\boldsymbol{r}|^2=\sqrt{x_j^2}$ for
    $\boldsymbol{r}=\begin{bmatrix} x_1 \\ x_2 \\ x_3\end{bmatrix}$.

    So consider the following:
    \begin{align*}
        \boldsymbol{\nabla} r
        &= \frac{\partial r}{\partial x_i}\boldsymbol{e}_i \\
        &= \boldsymbol{e}_i\frac{\partial}{\partial x_i}\sqrt{x_j^2} \\
        &= \boldsymbol{e}_i\cdot\frac{1}{2}\frac{1}{\sqrt{x_j^2}}
        \cdot 2x_j\frac{\partial x_j}{\partial x_i} \\
        &= \boldsymbol{e}_i\cdot\frac{1}{r}\cdot x_j\delta_{ij} \\
        &\implies \frac{1}{r}\boldsymbol{r}.
    \end{align*}
    Now that this is established we may use it:
    \begin{align*}
        \boldsymbol{\nabla}r^n
        &= \frac{\partial}{\partial r}(r^n)\cdot
        \boldsymbol{\nabla}r \\
        &= n\cdot r^{n-2}\boldsymbol{r},
    \end{align*}
    and we are finished. We can also do this with suffix notation:
    \begin{align*}
        \boldsymbol{\nabla}r^n
        &=\boldsymbol{e}_i\frac{\partial}{\partial x_i}[x_j^2]^{n/2} \\
        &=\frac{n}{2}[x_j^2]^{(n-2)/2}\boldsymbol{e}_i\frac{\partial}{\partial x_i}[x_k^2] \\
        &=\frac{n}{2}[x_j^2]^{(n-2)/2}\boldsymbol{e}_i\cdot2x_k\delta_{ik} \\
        &=n r^{n-2}\boldsymbol{r}.
    \end{align*}

    For part ($iii$), we need to find
    $\boldsymbol{\nabla}\cdot(r^n\boldsymbol{r})$.

    This is the \textbf{divergence} operator and is defined:
    $$\boldsymbol{\nabla}\cdot\boldsymbol{E}
    =\frac{\partial E_i}{\partial x_i},$$
    where $\boldsymbol{E}=\boldsymbol{E}(\boldsymbol{r})$ and is a \underline{vector field}. Therefore:
    $$\boldsymbol{\nabla}\cdot\boldsymbol{r}=3.$$
    We also need the \underline{chain rule for divergence}:
    \begin{align*}
        \boldsymbol{\nabla}\cdot(\phi\boldsymbol{a})
        &= \frac{\partial}{\partial x_i}(\phi\boldsymbol{a}) \\
        &\implies (\boldsymbol{\nabla}\phi)\cdot\boldsymbol{a}
        +\phi(\boldsymbol{\nabla}\cdot\boldsymbol{a}).
    \end{align*}
    Putting all this together we get:
    \begin{align*}
        \boldsymbol{\nabla}\cdot(r^n\boldsymbol{r})
        &= (\boldsymbol{\nabla}r^n)\cdot\boldsymbol{r}
        +r^n(\boldsymbol{\nabla}\cdot\boldsymbol{r}) \\
        &= (n+3)r^n.
    \end{align*}
    We can also do this in suffix notation:
    \begin{align*}
        \boldsymbol{\nabla}\cdot(r^n\boldsymbol{r})
        &=\frac{\partial}{\partial x_i}\bigl[r^n x_i\bigr] \\
        &=\left(\frac{\partial}{\partial x_i} r^n\right)x_i
        +r^n\left(\frac{\partial}{\partial x_i}x_i\right) \\
        &=\left(nr^{n-1}\frac{\partial}{\partial x_i}[x_k^2]^{1/2}\right)x_i
        +r^n\delta_{ii} \\
        &=(n+3)\boldsymbol{r}^n.
    \end{align*}
    
    \newpage

    For part ($iv$) we want $\boldsymbol{\nabla}\times
    (r^n\boldsymbol{r})$.

    So note the \underline{chain rule for curl}:
    $$\boldsymbol{\nabla}\times(\phi\boldsymbol{a})=
    \boldsymbol{\nabla}\phi\times\boldsymbol{a}
    +\phi\boldsymbol{\nabla}\times\boldsymbol{a}.$$
    Therefore:
    \begin{align*}
        \boldsymbol{\nabla}\times(r^n\boldsymbol{r})
        &=\boldsymbol{\nabla}r^n\times\boldsymbol{r}
        +r^n\boldsymbol{\nabla}\times\boldsymbol{r} \\
        &=n\cdot r^{n-2}\cdot\boldsymbol{r}\times\boldsymbol{r}
        +r^n\cdot\boldsymbol{0} \\
        &=\boldsymbol{0}.
    \end{align*}
    We can also do this with suffix notation:
    \begin{align*}
        \boldsymbol{\nabla}\times(r^n\boldsymbol{r})
        &=\boldsymbol{e}_i\epsilon_{ijk}\frac{\partial}{\partial x_j}
        [r^n\boldsymbol{r}]_k \\
        &=\boldsymbol{e}_i\epsilon_{ijk}\frac{\partial}{\partial x_j}
        [r^n x_k] \\
        &=\boldsymbol{e}_i\epsilon_{ijk}\left(
            \frac{\partial}{\partial x_j}
        [r^n] x_k+r^n\frac{\partial}{\partial x_j}x_k
        \right) \\
        &=\boldsymbol{e}_i\epsilon_{ijk}\left(
            nr^{n-2}x_i x_k+r^n\delta_{jk}
        \right) \\
        &=0.
    \end{align*}

    For part ($v$) we want $\boldsymbol{\nabla}\cdot
    (\boldsymbol{c}\times\boldsymbol{r})$.
    \begin{align*}
        \boldsymbol{\nabla}\cdot(\boldsymbol{c}\times\boldsymbol{r})
        &=\frac{\partial}{\partial x_i}[\epsilon_{ijk}c_jx_k] \\
        &=\epsilon_{ijk}c_j\cdot\frac{\partial}{\partial x_i}x_k \\
        &=\epsilon_{ijk}c_j\delta_{ik}
        =0.
    \end{align*}

    For part ($vi$) we want $\boldsymbol{\nabla}\times
    (\boldsymbol{c}\times\boldsymbol{r})$.

    \begin{align*}
        \boldsymbol{\nabla}\times(\boldsymbol{c}\times\boldsymbol{r})
        &=\epsilon_{ijk}\cdot\frac{\partial}{\partial x_j}[\epsilon_{klm}c_lx_m] \\
        &=\epsilon_{ijk}\epsilon_{klm}\cdot c_l\delta_{mj} \\
        &=(\delta_{il}\delta_{jm}-\delta_{im}\delta_{jl})c_l\delta_{mj} \\
        &=2c_i \\
        &=\boldsymbol{c}.
    \end{align*}

    For part ($vii$) we want $(\boldsymbol{c}\cdot\boldsymbol{\nabla})
    \boldsymbol{r}$.
    \begin{align*}
        (\boldsymbol{c}\cdot\boldsymbol{\nabla})\boldsymbol{r}
        &=c_i\frac{\partial}{\partial x_i}x_j\boldsymbol{e}_j\\
        &=c_i\delta_{ij}\boldsymbol{e}_j \\
        &=\boldsymbol{c}.
    \end{align*}

    \newpage
    
    \item


    \newpage


    \item 


    \newpage


    \item For ($i$) we are asked to show the following:
    \begin{itemize}
        \item $x\delta(x)=0$
        \item $\delta(cx)=\frac{1}{|c|}\delta(x)$
    \end{itemize}
    So we have the \textbf{sift} property:
    $$\int_{\mathbb{R}}f(x)\delta(x-a)dx=f(a).$$
    Taking $f(x)=x$ and $a=0$ gives:
    $$\int_{\mathbb{R}}x\delta(x)dx=0.$$
    Recall the \textbf{fundamental theorem of calculus}:
    $$\frac{d}{dx}\int_{x_0}^{x}g(t)dt=g(x).$$
    Let $g(t)=t\delta(t)$ and integrate over $\mathbb{R}$.
    $\therefore x\delta(x)=0.$
    
    Now consider the following identity:
    $$\int_{\mathbb{R}}\delta(x)dx=1.$$
    We aim to write $\delta(cx)$ into this form. First assume that $c>0$:
    \begin{align*}
        \int_{-\infty}^{\infty}\delta(cx)dx
        &= \int_{-\infty}^{\infty}\frac{1}{c}\delta(cx)d(cx) \\
        &= \frac{1}{c}.
    \end{align*}
    If $c=0$ then our equality holds trivially. Now assume that $c<0$.
    We can write $c$ as $c=-|c|$.
    or that:
    $$\int_{\mathbb{R}}|c|\delta(cx)dx=1.$$
    It must then be that:
    $$|c|\delta(cx)=\delta(x).$$
\end{enumerate}