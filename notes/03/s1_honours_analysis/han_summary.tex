\documentclass{article}
\usepackage{geometry}
\usepackage{amsmath}
\usepackage{amsfonts}
\usepackage{amssymb}
\usepackage{amsthm}
\usepackage{parskip}
\usepackage{multicol}
\usepackage{xcolor}
\usepackage{fancyhdr}
\usepackage{physics}
\usepackage{graphicx} % Required for inserting images
\usepackage{hyperref}
\usepackage{enumitem}
\usepackage{mathtools}

% commands
\newcommand{\deq}{\vcentcolon=}
\newcommand{\idd}{\text{đ}}
\newcommand{\nimplies}{\centernot\implies}
\newcommand{\vc}[1]{\boldsymbol{#1}}

% margin settings
\geometry{
    a4paper,
    left=7mm,
    right=7mm,
    top=2cm,
    bottom=7mm
}

% testing
\usepackage{blindtext}
    
% proof environments
\newtheorem{definition}{Definition}[section]
\newtheorem{theorem}{Theorem}[section]
\newtheorem{corollary}{Corollary}[theorem]
\newtheorem{lemma}[theorem]{Lemma}
\newtheorem*{remark}{Remark} % unnumbered remarks

% header and footer
\pagestyle{fancy}
\fancyfoot{} % removes footer
\fancyhf{}
\renewcommand{\headrulewidth}{0.5pt}
\fancyhead[L]{Honours Analysis}
\fancyhead[R]{\thepage}

\begin{document}

\begin{multicols*}{3}
% starred environment ensures text remains in same column
\noindent

\subsubsection*{D: Functions}
A function $f:X\rightarrow Y$ is an assignment
of an element of $Y$ to \underline{each} element of $X$.
\begin{enumerate}
    \item $f$ is \textbf{injective} if:
    \begin{align*}
        &\forall x_1,x_2\in X;
        f(x_1)=f(x_2) \\ &\implies x_1=x_2.
    \end{align*}

    \item $f$ is \textbf{surjective} if:
    $$\forall y\in Y;\exists x\in X: y=f(x).$$

    \item $f$ is \textbf{bijective}
    if it is injective and surjective.
\end{enumerate}

\subsubsection*{T: Triangle inequalities}
Let $\alpha,\beta\in\mathbb{R}$.
We then have that:
\begin{enumerate}
    \item $|\alpha|+|\beta|
    \geq|\alpha+\beta|$

    \item $\bigl||\alpha|-|\beta|\bigr|
    \leq|\alpha-\beta|$.
\end{enumerate}

\subsubsection*{D: Supremum and infimum}
Let $\alpha=\sup S$. Then:
\begin{enumerate}
    \item $\forall s\in S; \alpha\geq s$
    \item $\forall a\in\mathbb{R}:
    \forall s\in S; a\geq s;$ \\
    $\textcolor{red}{a\geq\alpha}$
\end{enumerate}
and similarly for infimum.

\subsubsection*{T: Approximation property}
Consider bounded $E\subset\mathbb{R}$. Then:
$$\forall\epsilon>0;\exists a\in E:
\sup E-\epsilon<a\leq\sup E.$$

\subsubsection*{D: Completeness of $\mathbb{R}$}
Every nonempty \underline{bounded} subset
of $\mathbb{R}$ has an infimum and supremum.

\subsubsection*{T: Archimedean property}
$\forall a,b\in\mathbb{R}; a>0;\exists n\in\mathbb{N}
: na>b$

\subsubsection*{D1.1: Nested intervals}
A sequence of sets $(I_n)_{n\in\mathbb{N}}$
is nested \\ if $I_1\supset I_2\supset I_3\dots$.

\subsubsection*{T1.1: Nested interval property}
Let $(I_n)_{n\in\mathbb{N}}$ be a sequence of
\underline{nonempty}, \underline{closed}
and \underline{bounded} nested intervals.
Then:
$$E=\bigcap_{n\in\mathbb{N}}I_n
\neq\emptyset.$$
If $\lambda(I_n)\rightarrow0$
then $E$ contains one number,
where $\lambda$ denotes length.

\subsubsection*{T1.2}
Let $E=[a,b]$ and that there exists an \underline{open}
collection of nested intervals
$(I_{\alpha})_{\alpha\in A}$ such that:
$$E\subset\bigcup_{\alpha\in A}I_{\alpha}.$$
Then $\exists\{\alpha_1,\alpha_2,\dots,\alpha_n\}
\subset A$ such that:
$$E\subset I_{\alpha_1}\cup I_{\alpha_2}\cup
\dots\cup I_{\alpha_n}.$$

\subsubsection*{D1.2: $\epsilon$-$N$ convergence}
Let $\displaystyle\lim_{n\rightarrow\infty}x_n=a$. Then:
\begin{align*}
    &\forall\epsilon>0; \exists N\in\mathbb{N}:
    \forall n\geq N \\
    &\implies |x_n-a|<\epsilon.
\end{align*}

\subsubsection*{D1.3: Cauchy sequences}
The sequence $(x_n)$ is Cauchy if:
\begin{align*}
    &\forall\epsilon>0;\exists N\in\mathbb{N}:
    \forall n,m\geq N \\ &\implies |x_n-x_m|<\epsilon.
\end{align*}

\subsubsection*{T1.3 and T1.4}
Cauchy $\iff$ $\epsilon$-$N$ convergent.

\subsubsection*{T: Monotone convergence}
Let $(x_n)_{n\in\mathbb{N}}$ be increasing and
bounded above. Then:
$$\lim_{n\rightarrow\infty}x_n=\sup_{n\in\mathbb{N}}
\{x_n\}$$
and similarly for sequences that are
decreasing and bounded below.

\subsubsection*{D1.4: Subsequences}
The subsequence of $(x_n)_{n\in\mathbb{N}}$ is
a sequence of form $(x_{n_k})_{k\in\mathbb{N}}$
and is a selection of the original sequence 
\textbf{taken in order}.

\subsubsection*{T1.5: Bolzano-Weierstrass}
Every \underline{bounded} real sequence has \textbf{a}
convergent subsequence.

\subsubsection*{D1.5: Limit inferior and superior}
Let $(x_n)$ be a bounded real sequence. Then:
$$\limsup_{n\rightarrow\infty}x_n
=\lim_{n\rightarrow\infty}\left(\sup_{k\geq n}x_k\right)$$
$$\liminf_{n\rightarrow\infty}x_n
=\lim_{n\rightarrow\infty}\left(\inf_{k\geq n}x_k\right).$$

\subsubsection*{T1.6}
The real sequence $(x_n)$ is convergent
\underline{if and only if}:
$$\limsup_{n\rightarrow\infty}x_n
=\liminf_{n\rightarrow\infty}x_n.$$

\subsubsection*{D1.6: Convergence of infinite series}
Series $\displaystyle S=\sum_{k=1}^{\infty}a_k$ is
convergent if:
$$\lim_{n\rightarrow\infty}\sum_{k=1}^{n}a_k<\infty.$$
Series $S$ is \textbf{absolutely convergent} if
$\displaystyle\sum_{k=1}^{\infty}|a_k|$ is also convergent.

Otherwise $S$ is conditionally convergent.

\subsubsection*{T1.7: Cauchy criterion for series}
$\displaystyle S=\sum_{k=1}^{\infty}a_k$ is convergent \textbf{if{}f}:
\begin{align*}
    &\forall\epsilon>0; \exists N\in\mathbb{N}:
    \forall m\geq n\geq N \\
    &\implies \left|\sum_{k=n+1}^{m}a_k\right|<\epsilon.
\end{align*}

\subsubsection*{T1.8}
Let $\displaystyle S=\sum_{k=1}^{\infty}a_k$
be absolutely convergent. Let $z:\mathbb{N}
\rightarrow\mathbb{N}$ be a bijection. Then:
$$\sum_{k=1}^{\infty}a_k
=\sum_{k=1}^{\infty}a_{z(k)}.$$

\subsubsection*{T1.9: Riemann rearrangement}
Let $\displaystyle S=\sum_{k=1}^{\infty}a_k$
be conditionally convergent. Then there exists rearrangements
such that $S$ can take on any value.

\subsubsection*{T: Geometric series}
Let $a\in\mathbb{R}$
and $\textcolor{red}{|r|<1}$. Then:
$$\sum_{k=1}^{\infty}ar^{k-1}
=\frac{a}{1-r}$$
$$\sum_{k=m}^{n}ar^{k-1}
=\left\{\begin{array}{ll}
    \frac{a(r^{m-1}-r^n)}{1-r} & r\neq1 \\
    a(n-m+1) & r=1
\end{array}\right.$$
where $m,n\in\mathbb{N}$.

\subsubsection*{D1.7: Sequential continuity}
Let $f:\text{dom}(f)\rightarrow\mathbb{R}$
where $\text{dom}(f)\subset\mathbb{R}$.
$f$ is \underline{continuous} at $\alpha\in\text{dom}(f)$ if:
\begin{align*}
    &\forall(x_n)_{n\in\mathbb{N}}\subset\text{dom}(f):
    \lim_{n\rightarrow\infty}x_n=\alpha \\
    &\implies\lim_{n\rightarrow\infty}f(x_n)=f(\alpha).
\end{align*}

\subsubsection*{T1.10}
Let $\alpha\in\mathbb{R}$ and $f,g$ continuous on $D$. \\
Then $\alpha f$, $f+g$, $fg$ are continuous on $D$.

\subsubsection*{T1.11}
Let $f$ be continuous at $\alpha\in\mathbb{R}$
and $g$ at $f(\alpha)$.
Then $g\circ f$ is continuous at $\alpha$.

\subsubsection*{D1.12: $\epsilon$-$\delta$ continuity}
Let $f:\text{dom}(f)\rightarrow\mathbb{R}$
where $\text{dom}(f)\subset\mathbb{R}$.
Then $f$ is continuous at $\alpha\in\text{dom}(f)$ if:
\begin{align*}
    &\forall\epsilon>0;\exists\delta>0:
    |x-\alpha|<\delta \\
    &\implies
    |f(x)-f(\alpha)|<\epsilon.
\end{align*}

\subsubsection*{T: Continuity test}
$f$ is continuous at $\alpha$ if:
$$\lim_{x\rightarrow\alpha}f(x)=f(\alpha)$$
where the limit from left/right must both exist
and be equal to each other.

\subsubsection*{D: Uniform continuity}
$f$ is uniformly continuous on $I$ if:
\begin{align*}
    &\forall\epsilon>0;\exists\delta>0:
    \forall x,y\in I;
    |x-y|<\delta \\
    &\implies |f(x)-f(y)|<\epsilon.
\end{align*}

\subsubsection*{Remark}
$f$ is \textbf{not} uniformly continuous on $I$ \textbf{if{}f}:
\begin{align*}
    &\exists\epsilon>0;\exists (x_n)_{n\in\mathbb{N}}
    \wedge
    (y_n)_{n\in\mathbb{N}}\subset I: \\
    &\lim_{n\rightarrow\infty}|x_n-y_n|=0
    \hspace{0.05in}\wedge \\
    &|f(x_n)-f(y_n)|\geq\epsilon
    \hspace{0.05in}\text{for}\hspace{0.05in}
    \forall n\in\mathbb{N}.
\end{align*}
Functions on closed bounded intervals
are always uniformly continuous.

\subsubsection*{D: Differentiability}
$f$ is differentiable at $\alpha$ if:
$$f'(\alpha)=\lim_{h\rightarrow0}
\frac{f(\alpha+h)-f(\alpha)}{h}.$$

\subsubsection*{Remark}
Differentiability implies continuity.

\subsubsection*{T1.13: Intermediate value theorem}
Let $f$ be continuous on $[a,b]$. \\
If $f(a)f(b)<0$ then:
$$\exists c\in(a,b):f(c)=0.$$

\subsubsection*{T1.14: Extreme value theorem}
Let $f$ be continuous on $[a,b]$. Then $f$ is \textit{bounded}
and $\exists c,d\in[a,b]$ such that:
$$f(c)=\min\{f(x):x\in[a,b]\}$$
$$f(d)=\max\{f(x):x\in[a,b]\}.$$

\subsubsection*{T: Mean value theorem}
Let $f$ be continuous on $[a,b]$ \\
and differentiable on $(a,b)$.
Then:
$$\exists c\in(a,b):f'(c)=\frac{f(b)-f(a)}{b-a}.$$

\subsubsection*{D2.1: Pointwise convergence}
$f_n\rightarrow f$ pointwise on $E$ if:
$$f(x)=\lim_{n\rightarrow\infty}f_n(x).$$
Here $f_n:E\rightarrow\mathbb{R}$ and:
\begin{align*}
    &\forall x\in E;\forall\epsilon>0;
    \exists N\in\mathbb{N}:\forall n\geq N \\
    &\implies |f_n(x)-f(x)|<\epsilon.
\end{align*}

\subsubsection*{D2.2: Uniform convergence}
$f_n\rightarrow f$ uniformly on $E$ if:
\begin{align*}
    &\forall\epsilon>0;
    \exists N\in\mathbb{N}:\forall n\geq N
    \hspace{0.05in}\text{and}\hspace{0.05in}
    \forall x\in E \\
    &\implies |f_n(x)-f(x)|<\epsilon.
\end{align*}

\subsubsection*{P2.1}
The following statements are equivalent.
\begin{enumerate}
    \item $f_n\rightarrow f$ uniformly on $E$
    \item $\displaystyle\lim_{n\rightarrow\infty}
    \sup_{x\in E}|f_n(x)-f(x)|=0$
    \item $\exists a_n\rightarrow0$ s.t.
    $|f_n(x)-f(x)|\leq a_n$ for $\forall x\in E$.
\end{enumerate}

\subsubsection*{T2.1}
If $f_n$ is continuous on $E$ \textbf{and}
$f_n\rightarrow f$ uniformly on $E$
then $f$ is continuous on $E$.

\subsubsection*{Remark}
If $f$ is \underline{not continuous} on $E$
then $f_n$ \underline{cannot} be uniform on $E$.

\subsubsection*{T2.5: Weierstrass M-test}
Let $E\subset\mathbb{R}$ and
$f_k:E\rightarrow\mathbb{R}$. \\
$\exists M_k>0 : \displaystyle
\sum_{k=1}^{\infty}M_k<\infty$. \\
If $\forall k\in\mathbb{N}$
and $\forall x\in E$;
$|f_k(x)|\leq M_k$
then: \\
$\displaystyle\sum_{k=1}^{\infty}f_k(x)$
converges uniformly on $E$.

\subsubsection*{D: Power series}
Let $(a_n)$ be a real sequence and $c\in\mathbb{R}$. Then:
$$f_{PS}(x)=\sum_{n=0}^{\infty}a_n(x-c)^n$$
is a power series centered at $c$, \\
with \textbf{radius of convergence}:
$$R=\sup\{r\geq0:(a_n r^n)\hspace{0.05in}
\text{is bounded}\}$$
where $R=\infty$ implying that series \\
converges everywhere.

\subsubsection*{T3.1: Convergence of power series}
Let $0<R<\infty$.
If $|x-c|<R$ then
$f_{PS}(x)$ converges absolutely.

If $|x-c|>R$ then $f_{PS}(x)$ diverges.

\subsubsection*{T3.2: Continuity of power series}
Let $0<r<R$ where $R$ is the radius of
convergence of $f_{PS}(x)$.

Then for $|x-c|\leq r$, $f_{PS}(x)$
converges absolutely and uniformly
to a \underline{continuous} function $f(x)$.

\subsubsection*{L3.1}
$\displaystyle\sum_{n=1}^{\infty}a_n(x-c)^n$
and $\displaystyle\sum_{n=1}^{\infty}n a_n(x-c)^{n-1}$
have the same radius of convergence.

\subsubsection*{T: Root and ratio tests}
Let $\displaystyle S=\sum_{n=1}^{\infty}\alpha_n$
and consider:
\begin{enumerate}
    \item Ratio test: $\displaystyle\rho=\lim_{n\rightarrow\infty}
    \left|\frac{\alpha_{n+1}}{\alpha_n}\right|$

    \item Root test: $\displaystyle\rho=
    \lim_{n\rightarrow\infty}|\alpha_n|^{1/n}$.
\end{enumerate}
Then:
\begin{itemize}
    \item $\rho<1$: $S$ converges absolutely
    \item $\rho>1$: $S$ diverges
    \item $\rho=1$: test is inconclusive.
\end{itemize}

\subsubsection*{T3.3}
Let $R$ be the radius of convegence of
$f_{PS}(x)$. Then for $\forall x:|x-c|<R$,
$f_{PS}(x)$ is \textbf{infinitely differentiable} and:
$$f_{PS}(x)=\sum_{n=0}^{\infty}a_n(x-c)^n$$
$$a_n=\frac{f^{(n)}(c)}{n!}.$$

\subsubsection*{T: Taylor's theorem}
Let $f$ be $n$ times differentiable at $\alpha\in\mathbb{R}$
where $n\in\mathbb{N}$. Then:
\begin{align*}f(x)&=\sum_{k=1}^{n}\frac{f^{(k)}(\alpha)}{k!}
(x-\alpha)^k \\ &\quad+h_n(x)(x-\alpha)^n\end{align*}
where $\displaystyle\lim_{x\rightarrow\alpha}h_n(x)=0$.

\subsubsection*{Elementary expansions}
\begin{itemize}
    \item $\displaystyle e^x
    =\sum_{n=0}^{\infty}\frac{x^n}{n!}$

    \item $\displaystyle \sin x
    =\sum_{n=0}^{\infty}\frac{(-1)^n x^{2n+1}}{(2n+1)!}$

    \item $\displaystyle \cos x
    =\sum_{n=0}^{\infty}\frac{(-1)^n x^{2n}}{(2n)!}$
\end{itemize}

\subsubsection*{D: Characteristic functions}
Let $E\subset\mathbb{R}$.
The characteristic function is defined as
a real function such that:
$$\chi_E(x) =
    \left\{
    \begin{array}{ll}
        1  & \mbox{} x \in E \\
        0 & \mbox{otherwise.}
    \end{array}
\right.$$

\subsubsection*{D4.1 and D4.2: Step functions}
The step function with respect to
finite set $\{x_0, \dots, x_n\}$
for some $n \in \mathbb{N}$ is:
$$\phi(x) =
    \left\{
    \begin{array}{ll}
	0  &  x < x_0 \hspace{0.06in} \text{or} \hspace{0.06in} x > x_n \\
	c_j & x \in (x_{j-1}, x_j); \hspace{0.06in} 1 \leq j \leq n
    \end{array}
\right.$$
and its integral is defined as:
$$\int\phi=\sum_{j=1}^{n}c_j(x_j-x_{j-1}).$$

\subsubsection*{D4.3: Lebesgue integrable}
$f:I\rightarrow\mathbb{R}$ is Lebesgue integrable on $I$ if:
\begin{enumerate}
    \item $\displaystyle\sum_{j=1}^{\infty} |c_j| \lambda(J_j) < \infty$

    \item $\forall x\in I; f(x)=\displaystyle\sum_{j=1}^{\infty} |c_j| \chi_{J_j}(x)<\infty$
\end{enumerate}
Here $c_j \in \mathbb{R}$, $J_i \subset I$ and is bounded for 
$j \in \{1, 2, 3, \dots\}$. Then:
$$\int_I f = \sum_{j=1}^{\infty} |c_j| \lambda(J_j).$$

\subsubsection*{T4.1}
Let $c_j,d_j\in\mathbb{R}$ and $J_j,K_j$ be bounded intervals
where $j\in\{1,2,\dots\}$. Let:
$$\sum_{j=1}^{\infty}|c_j|\lambda(J_j)<\infty$$
$$\sum_{j=1}^{\infty}|d_j|\lambda(K_j)<\infty.$$
If:
\begin{align*}
    &\forall x;\sum_{j=1}^{\infty}c_j\chi_{J_j}(x)
    =\sum_{j=1}^{\infty}d_j\chi_{K_j}(x): \\
    &\sum_{j=1}^{\infty}|c_j|\chi_{J_j}(x)<\infty
    \hspace*{0.05in}\text{and} \\
    &\sum_{j=1}^{\infty}|d_j|\chi_{K_j}(x)<\infty
\end{align*}
then $\displaystyle\sum_{j=1}^{\infty}c_j\lambda(J_j)
=\sum_{j=1}^{\infty}d_j\lambda(K_j)$.

\subsubsection*{T4.2: Basic properties}
Let $f,g$ be integrable on $I$ and $\alpha,\beta\in\mathbb{R}$.
\begin{enumerate}
    \item $\alpha f+\beta g$ is integrable on $I$ and:
    $$\int_I(\alpha f+\beta g)
    =\alpha\int_I f+\beta\int_I g.$$

    \item If $f\geq g$ on $I$
    then $\displaystyle\int_I f\geq\int_I g$.

    \item $|f|$ is integrable on I and:
    $$\int_I |f|\geq\left|\int_I f\right|.$$

    \item If $f$ or $g$ is bounded on $I$
    then $fg$ is integrable on $I$.

    \item If $f\geq0$ and $\displaystyle\int_I f=0$, \\
    then $\forall h$ such that 
    $0\leq h\leq f$ is also integrable on $I$.
\end{enumerate}

\subsubsection*{T4.3}
Let $f_n$ be integrable on $I$ where:
$$\sum_{n=1}^{\infty}\int_I |f_n|<\infty.$$
\begin{enumerate}
    \item Let $f$ be defined as:
    \begin{align*}
        &\forall x\in I;f(x)=\sum_{n=1}^{\infty}f_n(x)
        : \\ &\sum_{n=1}^{\infty}|f_n(x)|<\infty.
    \end{align*}
    Then $f$ is integrable on $I$ and:
    $$\int_I f=\sum_{n=1}^{\infty}\int_I f_n.$$

    \item Let each $f_n\geq0$ and:
    $$\forall x\in I; f(x)=\sum_{n=1}^{\infty}f_n(x).$$
    Then $f$ is integrable on $I$ \textbf{if{}f}:
    $$\sum_{n=1}^{\infty}\int_I f_n<\infty.$$
\end{enumerate}

\subsubsection*{T4.4: MCT for integrals}
Let $f_n$ be \underline{monotone increasing}
sequence of functions on $I$ and that:
$$\forall x\in I;f(x)=\lim_{n\rightarrow\infty}f_n(x).$$
Then $f$ is integrable on $I$ \textbf{if{}f}:
$$\sup_{n\in\mathbb{N}}\int_I f_n
=\lim_{n\rightarrow\infty}\int_I f_n<\infty.$$
Furthermore:
$$\int_I f=\lim_{n\rightarrow\infty}\int_I f_n.$$

\subsubsection*{D4.4: Riemann integrable}
$f$ is Riemann-integrable on $[a,b]$ if:
\begin{align*}
    &\forall\epsilon>0; \exists\phi,
    \psi:\phi\leq f\leq\psi \\
    &\text{and}\hspace*{0.05in}
    \int\psi-\int\phi<\epsilon
\end{align*}
where $\phi$ and $\psi$ are step functions, \\
i.e. the bounded support of $f$.

\subsubsection*{T4.5}
$f$ is Riemann-integrable \underline{if and only if}:
$$\sup\left\{\int\phi:\phi\leq f\right\}
=\inf\left\{\int\psi:f\leq\psi\right\}$$
where $\phi$ and $\psi$ are step functions.

\subsubsection*{T4.6}
If $f$ is Riemann-integrable on $I$ then
$f$ is also Lebesgue-integrable on $I$.

\subsubsection*{Remark}
The converse of T4.6 is \underline{not true}.

\subsubsection*{Remark}
If $f$ is Riemann-integrable on $I$
then $|f|$ is also Riemann-integrable on $I$,
but reverse is not true!

\subsubsection*{L4.1}
Let $f$ be a bounded function with
bounded support on $[a,b]$.
The following statements are equivalent:
\begin{enumerate}
    \item $f$ is Riemann-integrable.

    \item $\forall\epsilon>0;
    \exists\{a=x_0<\dots<x_n=b\}:$
    $$\sum_{j=1}^{n}(M_j-m_j)(x_j-x_{j-1})<\epsilon$$
    where we define:
    $$M_j=\sup_{x\in(x_{j-1},x_j)}\Bigl\{f(x)\Bigr\}$$
    $$m_j=\inf_{x\in(x_{j-1},x_j)}\Bigl\{f(x)\Bigr\}$$
    and $n\in\mathbb{N}$.
    (i.e. finite partition)

    \item $\forall\epsilon>0;
    \exists\{a=x_0<\dots<x_n=b\}:$
    $$\sum_{j=1}^{n}\sup_{x,y\in I_j}
    |f(x)-f(y)|\lambda(I_j)<\epsilon$$
    where $I_j=(x_{j-1},x_j)$.
\end{enumerate}

\subsubsection*{T4.7}
Let $g:[a,b]\rightarrow\mathbb{R}$
and $f$ be such that $f(x)=g(x)$ if $x\in[a,b]$
and $f(x)=0$ otherwise.
\begin{enumerate}
    \item If $g$ is continuous on $[a,b]$
    then $f$ is Riemann-integrable.

    \item If $g$ is a monotone function
    then $f$ is Riemann-integrable.
\end{enumerate}

\subsubsection*{T4.8}
Let $J\subset I$.
\begin{enumerate}
    \item If $f$ is integrable on $I$ \\
    then $f$ is integrable on $J$.

    \item If $f$ is integrable on $J$
    and \\ for $\forall x\in I\backslash J; f(x)=0$ \\
    then $f$ is integrable on $I$.

    Furthermore: $\displaystyle\int_J f=\int_I f$.

    \item If $f$ is integrable on $I$ and
    $f(x)\geq0$ for $\forall x\in I$ then:
    $$\int_I f\geq\int_J f.$$

    \item Assume that $I$ can be written as the union
    of disjoint intervals $I_n$ and that
    $f$ is integrable on each $I_n$.

    Then $f$ is integrable on $I$ \textbf{if{}f}:
    $$\sum_{n=1}^{\infty}\int_{I_n}|f|<\infty.$$
    If this is true then:
    $$\int_I f=\sum_{n=1}^{\infty}\int_{I_n}|f|.$$
\end{enumerate}

\subsubsection*{T4.9}
If any \underline{two} of the following integrals exists:
$$\int_{a}^{b}f,\hspace*{0.3in}
\int_{b}^{c}f,\hspace*{0.3in}
\int_{a}^{c}f$$
then so does the third and:
$$\int_{a}^{b}f+\int_{b}^{c}f=\int_{a}^{c}f.$$

\subsubsection*{T4.10: FTC I}
Let $g$ be integrable on $I$ and let:
$$G(x)=\int_{x_0}^{x}g(s)\dd s$$
where $x,x_0\in I$.

If $g$ is continuous at $x$ then:
$$\dv{x}G(x)=g(x).$$

\subsubsection*{T4.11: FTC II}
Let $f'(x)$ be continuous on $I$. Then:
$$\int_{a}^{b}f'(x)\dd x=f(b)-f(a)$$
where $a,b\in I$.

\subsubsection*{L4.2: Fatoux's lemma}
Let $f_n\geq0$ be integrable on $I$ and:
$$\forall x\in I;f(x)=\liminf_{n\rightarrow\infty}f_n(x).$$
If $\displaystyle\liminf_{n\rightarrow\infty}
\int_I f_n<\infty$ then:
$$\int_I f\leq\liminf_{n\rightarrow\infty}\int_I f_n.$$

\subsubsection*{T4.12: Dominated convergence}
Let $f_n,g$ be integrable on $I$ and:
$$\forall x\in I; f(x)=\lim_{n\rightarrow\infty}f_n(x).$$
If $|f_n(x)|\leq g(x)$ for $\forall x\in I$ \\
then $f$ is integrable on $I$ and:
$$\int_I f=\lim_{n\rightarrow\infty}\int_I f_n.$$

\subsubsection*{T4.13}
Let $f_n$ be integrable on $(a,b)$ and that
$f_n\rightarrow f$ uniformly on $(a,b)$.

Then $f$ is integrable on $(a,b)$ and:
$$\int_{a}^{b}f=\lim_{n\rightarrow\infty}\int_{a}^{b}f_n.$$

\subsubsection*{D5.1: $L^2$ space}
$f\in L^2([a,b])$ if:
\begin{enumerate}
    \item $f:[a,b]\rightarrow\mathbb{C}$ is measurable

    \item $x\mapsto|f(x)|^2$
    is integrable:
    $$||f||_2^{\textcolor{red}{2}}\deq
    \int_{a}^{b}|f(x)|^2\dd x<\infty$$
\end{enumerate}
and $||f||_2$ is the $L^2$-norm of $f$.

\subsubsection*{Remark}
If $z\in\mathbb{C}$ then $z\bar{z}\deq|z|^2$.

\subsubsection*{D5.2: Inner products}
The inner product of $f,g\in L^2([a,b])$ is:
$$\langle f,g\rangle=
\int_{a}^{b}f(x)\overline{g(x)}\dd x.$$

\subsubsection*{T5.1: Cauchy-Schwarz inequality}
Let $f,g\in L^2([a,b])$. Then:
$$|\langle f,g\rangle|\leq||f||_2||g||_2.$$

\subsubsection*{C: Minkowski's inequality}
Let $f,g\in L^2([a,b])$. Then:
$$||f+g||_2\leq||f||_2+||g||_2.$$

\subsubsection*{D5.3: $L^2$ convergence}
$f_n\rightarrow f$ in $L^2$ if:
$$\lim_{n\rightarrow\infty}||f_n-f||_2=0.$$
Here $f,f_1,f_2,\ldots\in L^2([a,b])$.

\subsubsection*{D5.4: Orthonormal systems}
The sequence of functions $(\phi_n)_{n\in\mathbb{N}}$ 
in $L^2$ is an orthonormal system on $[a,b]$ if:
$$\langle\phi_m,\phi_n \rangle=\delta_{mn}.$$

\subsubsection*{T5.2}
Let $(\phi_n)_{n\in\mathbb{N}}$ be an
orthonormal system on $[a,b]$ with
\textbf{linear span} $X_n$. 

Assume that $f\in L^2$ and:
$$s_N(x)=\sum_{n=1}^{N}
\langle f,\phi_N\rangle\phi_n(x).$$
Then:
$$||f-s_N||_2\leq||f-g||_2$$
holds for $\forall g\in X_n$.

\subsubsection*{T5.3: Bessel's inequality}
Let $(\phi_n)_{n\in\mathbb{N}}$ be an
orthonormal system on $[a,b]$ and $f\in L^2$.
Then:
$$\sum_{n\in\mathbb{N}}|
\langle f,\phi_n\rangle|^2
\leq||f||_2^2.$$

\subsubsection*{C: Riemann-Lebesgue lemma}
Let $(\phi_n)_{n\in\mathbb{N}}$ be an
orthonormal system on $[a,b]$ and $f\in L^2$.
Then:
$$\lim_{n\rightarrow\infty}\langle f,\phi_n\rangle=0.$$

\subsubsection*{D5.5: Completeness}
The orthonormal system $(\phi_n)_{n\in\mathbb{N}}$ 
is complete if:
$$\sum_{n\in\mathbb{N}}|
\langle f,\phi_n\rangle|^2=||f||_2^2$$
for $\forall f\in L^2$.

\subsubsection*{T5.4}
Let $(\phi_n)_{n\in\mathbb{N}}$ be an
orthonormal system on $[a,b]$ and let
$(s_N)_{N\in\mathbb{N}}$ be a sequence of functions
where:
$$s_N(x)=\sum_{n=1}^{N}
\langle f,\phi_N\rangle\phi_n(x).$$
Then $(\phi_n)_{n\in\mathbb{N}}$
is complete \textbf{if{}f}:
$$\forall f\in L^2; s_N\rightarrow f
\hspace{0.05in}\text{in $L^2$}.$$

\subsubsection*{D5.6: Trigonometric polynomial}
Trigonometric polynomials are functions of form:
$$f(x)=\sum_{n\in\mathbb{Z}}
c_n e^{2\pi inx}$$
where $x\in\mathbb{R}$ and $c_n\in\mathbb{C}$.

\subsubsection*{L5.1}
$(e^{2\pi inx})_{n\in\mathbb{Z}}$ forms
an orthonormal system on $[0,1]$. Furthermore:
\begin{enumerate}
    \item $\displaystyle
    \int_{0}^{1}e^{2\pi inx}\dd x
    =\left\{\begin{array}{ll}
        0 & n\neq0 \\
        1 & n=0
    \end{array}\right.$

    \item If
    $\displaystyle f_{FS}
    =\sum_{n\in\mathbb{Z}}c_n e^{2\pi inx}$ then:
    $$c_n=\langle f,e^{2\pi inx}\rangle.$$
\end{enumerate}

\subsubsection*{D5.7 and D5.8: Fourier series}
The $n$th Fourier coefficient of integrable $1$-periodic $f$
where $n\in\mathbb{Z}$ is defined as:
$$\widehat{f}(n)=\langle f,\phi_n\rangle$$
and the Fourier series of $f$ is:
$$f_{FS}=\sum_{n\in\mathbb{Z}}\widehat{f}(n)
e^{2\pi inx}.$$
The Fourier partial sums is defined as:
$$S_N f(x)=\sum_{n=-N}^{N}
\widehat{f}(n)e^{2\pi inx}$$
where $N\in\mathbb{Z}$.

\subsubsection*{D5.9: Convolutions}
The convolution of $1$-periodic
functions $f,g\in L^2$ is:
$$f*g(x)=\int_{0}^{1}f(t)g(x-t)\dd t.$$

\subsubsection*{L5.2}
For $1$-periodic $f,g\in L^2$:
$f*g=g*f$.

\subsubsection*{L5.3: Dirichlet kernel}
The Dirichlet kernel is defined as:
\begin{align*}
    D_N(x)
    &=\sum_{n=-N}^{N}e^{2\pi inx} \\
    &=\frac{\sin(2N+1)\pi x}{\sin\pi x}
\end{align*}
where $N\in\mathbb{N}$.

\subsubsection*{L5.4: Fej\'er kernel}
The Fej\'er kernel is defined as:
\begin{align*}
    K_N(x)
    &=\frac{1}{N+1}
    \sum_{n=0}^{N}D_n(x) \\
    &=\frac{1}{N+1}
    \left[\frac{\sin(N+1)\pi x}
    {\sin\pi x}\right]^2
\end{align*}
where $N\in\mathbb{N}$.

\subsubsection*{T5.5: Fej\'er's theorem}
In the limit $N\rightarrow\infty$:
$$K_N*f\rightarrow f\hspace{0.05in}
\text{uniformly on $\mathbb{R}$}$$
where $f$ is $1$-periodic and continuous.

\subsubsection*{C}
For every $1$-periodic continuous $f$:
$$\exists(f_n)_{n\in\mathbb{N}}
:f_n\rightarrow f\hspace{0.05in}\text{uniformly on $D$}$$
for $f_n$ is a trigonometric polynomial
and domain $D$ subject to $f$.

\subsubsection*{D5.10: Approximation of unity}
A sequence of $1$-periodic integrable \\
$(k_n)_{n\in\mathbb{N}}$ is an approximation
of unity if for all $1$-periodic continuous $f$:
$$f*k_n\rightarrow f\hspace{0.05in}
\text{uniformly on $\mathbb{R}$}$$
or that:
$$\lim_{n\rightarrow\infty}
\sup_{x\in\mathbb{R}}
|f*k_n(x)-f(x)|=0.$$

\subsubsection*{T5.6}
Let $(k_n)_{n\in\mathbb{N}}$ be
a sequence of $1$-periodic integrable functions
that satisfies:
\begin{enumerate}
    \item $k_n(x)\geq0$ for $\forall x\in\mathbb{R}$.
    
    \item $\displaystyle
    \int_{-1/2}^{1/2}k_n(t)\dd t=1$

    \item $\forall\delta\in(0,\frac{1}{2}];
    \displaystyle\lim_{n\rightarrow\infty}
    \int_{-\delta}^{\delta}k_n(t)\dd t=1$.
\end{enumerate}
Then $(k_n)_{n\in\mathbb{N}}$
is an approximation of unity.

\subsubsection*{C}
The Fej\'er kernel $(K_N)_{N\in\mathbb{N}}$
is an approximation of unity.

\subsubsection*{L5.5}
If $f$ is $1$-periodic continuous
then:
$$\lim_{N\rightarrow\infty}||S_N f-f||_2=0.$$

\subsubsection*{T5.7}
For every $1$-periodic $f\in L^2$:
$$S_N f\rightarrow f
\hspace{0.05in}\text{in $L^2$}$$
or that the Fourier series of $f$
converges to $f$ 
in the $L^2$ sense.

\subsubsection*{C: Parseval's theorem}
Let $f,g\in L^2$ be $1$-periodic. Then:
$$\langle f,g\rangle=
\sum_{n\in\mathbb{Z}}\widehat{f}(n)
\overline{\widehat{g}(n)}$$
and in particular:
$$||f||_2^2=\sum_{n\in\mathbb{Z}}
|\widehat{f}(n)|^2.$$

\end{multicols*}

\end{document}