\pagestyle{fancy}
\fancyhead{}
\fancyhead[L]{Honours Analysis Workshop 6}
\fancyhead[R]{Winter 2023}

\section{Workshop 6}
\begin{enumerate}
    \item Consider real function $f=x^2$. We know that $f$ is continuous on $\mathbb{R}$, since it is a polynomial.
    So for $f:\mathbb{R}\rightarrow\mathbb{R}$ the $\epsilon-\delta$ definition states:

    $\forall \alpha \in \mathbb{R};$ $\forall\epsilon>0;$ $\exists\delta>0;$
    $\forall x \in \mathbb{R};$ \\
    $|x-\alpha|<\delta \implies |f(x)-f(\alpha)|<\epsilon$

    Now we set $\alpha>1$ and $\epsilon=1$. Find \textbf{best} possible $\delta = \delta (\epsilon)$. \\

    So whilst we can choose $1>\delta(\delta+2\alpha)$, this is certainly not the best bound.

    Consider this approach instead:
    \begin{align*}
        \alpha-\delta<x<\alpha+\delta &\implies \alpha^2-1<x^2<\alpha^2+1 \\
        &\implies \sqrt{\alpha^2-1}<x<\sqrt{\alpha^2+1}
    \end{align*}
    i.e. since we have an implication:
    $$\sqrt{\alpha^2-1}<\alpha-\delta<x<\alpha+\delta<\sqrt{\alpha^2+1}$$
    This is really 4 inequalities, and we need to choose the best 2. So:
    $$\sqrt{\alpha^2-1}<\alpha-\delta\implies\delta<\alpha-\sqrt{\alpha^2-1}$$
    $$\alpha+\delta<\sqrt{\alpha^2+1}\implies\delta<-\alpha+\sqrt{\alpha^2+1}$$
    We can prove that $-\alpha+\sqrt{\alpha^2+1}>-\alpha+\sqrt{\alpha^2+1}$
    by contradiction. Now for the lower bound:
    $$\sqrt{\alpha^2-1}<\alpha+\delta\implies-\alpha+\sqrt{\alpha^2-1}<\delta$$
    $$\alpha-\delta<\sqrt{\alpha^2+1}\implies\alpha-\sqrt{\alpha^2+1}<\delta$$
    By contradiction we have $\alpha-\sqrt{\alpha^2+1}>-\alpha+\sqrt{\alpha^2-1}$. Hence:
    $$\alpha-\sqrt{\alpha^2+1}<\delta<-\alpha+\sqrt{\alpha^2+1}$$
    ? our lower bound is wrong ?

    \newpage

    \textbf{Another approach}
    
    We begin from here:
    \begin{align*}
        \alpha-\delta<x<\alpha+\delta &\implies \alpha^2-1<x^2<\alpha^2+1 \\
        &\implies \sqrt{\alpha^2-1}<x<\sqrt{\alpha^2+1}.
    \end{align*}
    It is clear from a graph that the distance from any $x$ to $\alpha$ cannot exceed either $\alpha-\sqrt{\alpha^2-1}$ or $\sqrt{\alpha^2+1}-\alpha$ for our function to be continuous.
    \newpage

    \item Now define $f:[0, 1]\rightarrow\mathbb{R}$ with rule $f(x)=x^2$.

    Show: $\forall\epsilon>0;$ $\exists\delta=\epsilon/2;$
    $\forall x, \alpha \in [0, 1];$
    $|x-\alpha|<\delta \implies |f(x)-f(\alpha)|<\epsilon$ \\

    \begin{proof}
        We really should take the hint when given. Note that $\forall x, \alpha \in [0,1]$:
        $$|x+\alpha|<|x|+|\alpha|\leq2$$
        by the triangle inequality. (helpful to think of a triangle)
        
        Since polynomials are continuous we apply the $\epsilon-\delta$ continuity definition:
        
        $\forall\epsilon>0;$ $\exists\delta>0;$
        $\forall x, \alpha \in [0, 1];$
        $|x-\alpha|<\delta \implies |x^2-\alpha^2|<\epsilon$
    
        Consider the final line:
        \begin{align*}
            |x^2-\alpha^2| &= |x+\alpha||x-\alpha| \\
            &<2|x-\alpha|
        \end{align*}
        So if we choose $\delta=\frac{\epsilon}{2}$ given any $\epsilon$ then $|x-\alpha|<\epsilon$ and
        $$|x^2-\alpha^2|<\epsilon.$$
    \end{proof}

    \newpage

    \item Consider function $f:(0,\infty)\rightarrow\mathbb{R}$ with rule $f(x)=\frac{1}{x}$.
    
    Is this function \textbf{uniformly continuous}? \\

    So firstly uniform continuity only makes sense if our function is already continuous. Since $x=0$ is removed, our function is continuous and we may consider uniform continuity.

    Here we claim that $f$ is \textbf{not} uniformly continuous.

    Note that the following two notions of uniform continuity is equivalent:
    \begin{itemize}
        \item $\forall\epsilon>0;$ $\exists\delta>0;$
        $\forall x, y \in I;$
        $|x-y|<\delta \implies |f(x)-f(y)|<\epsilon$

        \item $\forall s_n, t_n \in I;$
        $\displaystyle\lim_{n\rightarrow\infty}|s_n - t_n|=0$
        $\implies\displaystyle\lim_{n\rightarrow\infty}|f(s_n) - f(t_n)|=0$
    \end{itemize}
    given function $f:I\rightarrow\mathbb{R}$. This makes our life easy. To disprove uniform continuity we just need to negate the second condition:
    $$\exists s_n, t_n \in I; \lim_{n\rightarrow\infty}|s_n - t_n|=0
    \hspace{0.1in}\text{and}\hspace{0.1in}\lim_{n\rightarrow\infty}|f(s_n) - f(t_n)|\neq0.$$
    Choose $s_n=\frac{1}{n}$ and $t_n=\frac{2}{n}$ and we are finished.

    \newpage

    \item Consider function $f:[a,\infty)\rightarrow\mathbb{R}$
    for $\boldsymbol{a>0}$ and $f(x)=\frac{1}{x}$.

    Is this function uniformly continuous? \\

    \begin{proof}
        We claim that $f$ is uniformly continuous.
        
        So we need:
        $$\forall\epsilon>0; \exists\delta>0; \forall x, y\in [a,\infty):
        |x-y|<\delta\implies|f(x)-f(y)|<\epsilon$$
        Substituting $f$ we have that $|x-y|<\delta\implies|\frac{1}{x}-\frac{1}{y}|<\epsilon$.

        Firstly without loss of generality define $x>y\geq a>0$.
        $$\therefore \frac{1}{a}>\frac{1}{y}>\frac{1}{x}\implies
        \frac{1}{a^2}>\frac{1}{xy}.$$
        Now consider:
        \begin{align*}
            |\frac{1}{x}-\frac{1}{y}|
            &= \frac{|x-y|}{xy} \\
            &< \frac{|x-y|}{a^2} \\
            &< \epsilon
        \end{align*}
        if we choose $\delta=a^2\epsilon$. Therefore:
        $$\forall\epsilon>0; \exists\delta=a^2\epsilon; \forall x, y\in [a,\infty):
        |x-y|<\delta\implies|\frac{1}{x}-\frac{1}{y}|<\epsilon,$$
        or that $f$ is uniformly continuous on $[a,\infty)$ where $a>0$.
    \end{proof}

    \newpage

    \textbf{Mean value theorem approach} \\
    \textbf{Not finished!} \\
    Firstly $f(x)=\frac{1}{x}$ is differentiable on $[a,\infty)$
    as we have the following limit:
    \begin{align*}
        f'(x)
        &= \lim_{h\rightarrow 0}\frac{f(x+h)-f(x)}{h} \\
        &= \lim_{h\rightarrow 0}\frac{\frac{1}{x+h}-\frac{1}{x}}{h} \\
        &= -\frac{1}{x^2}\hspace{0.1in}\text{if}\hspace{0.1in}x\neq0.
    \end{align*}
    Define $x>y\geq a>0$. By the mean value theorem we have:
    $$\forall x, y\in [a,\infty); \exists c\in [y,x];
    f'(c)=\frac{f(x)-f(y)}{x-y},$$
    or that
    \begin{align*}
        |\frac{1}{x}-\frac{1}{y}|
        &= \frac{1}{x}-\frac{1}{y} \\
        &=-\frac{1}{c^2}(x-y).
    \end{align*}
    Now for fixed $[y, x]$, the mean value theorem states that
    $c>\min\{x, y\}$ and hence $c>a$. (this holds $\forall x, y$) $\therefore \frac{1}{a^2}>\frac{1}{c^2}\implies-\frac{1}{a^2}>-\frac{1}{c^2}$
    and:
    \begin{align*}
        |\frac{1}{x}-\frac{1}{y}|
        &= \frac{1}{x}-\frac{1}{y} \\
        &=-\frac{1}{c^2}(x-y) \\
        &< -\frac{1}{a^2}(x-y).
    \end{align*}
    Since $x$ and $y$ are non-negative $x-y=|x-y|$ and:
    $$|\frac{1}{x}-\frac{1}{y}|<-\frac{1}{a^2}|x-y|=-\frac{\delta}{a^2}=\epsilon.$$
    Now pick $\delta=-a^2 \epsilon$ and we are finished.

    \newpage

    \item 5

    \newpage

    \item 6

    \newpage

    \item Prove that the following statements are equivalent:
    \begin{itemize}
        \item $f:I\rightarrow\mathbb{R}$ is uniformly continuous.
        
        i.e. $\forall\epsilon>0;$ $\exists\delta>0;$
        $\forall x, y \in I:$
        $|x-y|<\delta \implies |f(x)-f(y)|<\epsilon$

        \item $\forall s_n, t_n \in I:$
        $\displaystyle\lim_{n\rightarrow\infty}|s_n - t_n|=0$
        $\implies\displaystyle\lim_{n\rightarrow\infty}|f(s_n) - f(t_n)|=0$
    \end{itemize}
    \begin{proof}
        $\rightarrow$ direction
        
        Direct proof. Assume that $f$ is uniformly continuous:
        $$\forall\epsilon>0; \exists\delta>0; \forall x, y \in I: |x-y|<\delta
        \implies |f(x)-f(y)|<\epsilon.$$
        Also assume that:
        $$\forall s_n, t_n\in I: \lim_{n\rightarrow\infty}|s_n-t_n|=0.$$
        But this may also be written as:
        $$\forall\delta>0; \forall s_n, t_n\in I; \exists N\in\mathbb{N}: \forall n\geq N\implies |s_n - t_n|<\delta,$$
        and since the definition of uniform continuity holds $\forall x, y\in I$ we may set $x=s_n$ and $y=t_n$. Combining our assumptions we get:
        \begin{align*}
            \forall\epsilon>0; \exists\delta>0; \forall s_n, t_n\in I; \exists N\in\mathbb{N}: \forall n\geq N
            &\implies |s_n-t_n|<\delta \\
            &\implies |f(s_n)-f(t_n)|<\epsilon
        \end{align*}
        But really what we want is:
        $$\forall\epsilon>0; \exists N\in\mathbb{N}:
        \forall n\geq N\implies|f(s_n)-f(t_n)|<\epsilon$$
        Or that:
        $$\lim_{n\rightarrow\infty}|f(s_n)-f(t_n)|=0.$$
    \end{proof}

    \newpage
    
    \begin{proof}
        $\leftarrow$ direction

        Proof by contradiction. Assume that if:
        $$\forall s_n, t_n \in I: \lim_{n\rightarrow\infty}|s_n - t_n|=0\implies
        \lim_{n\rightarrow\infty}|f(s_n) - f(t_n)|=0,$$
        then $f$ is \textbf{not} uniformly continuous. i.e. that:
        $$\exists\epsilon>0; \forall\delta>0;\exists x, y\in I: |x-y|<\delta
        \hspace{0.07in}\text{and}\hspace{0.07in}|f(x)-f(y)|\geq\epsilon.$$
        So our first assumption gives us:
        $$\forall\epsilon>0; \exists N_1\in\mathbb{N}:
        \forall n\geq N_1\implies|f(s_n)-f(t_n)|<\epsilon$$
        and holds true $\forall s_n, t_n\in I$ with condition:
        $$\forall\delta>0; \exists N_2\in\mathbb{N}:
        \forall n\geq N_2\implies|s_n - t_n|<\delta.$$
        So taking $N=\max\{N_1, N_2\}$ and combining the previous two statements:
        \begin{align*}
            \forall\epsilon>0; \forall\delta>0; \forall s_n, t_n \in I;
            \exists N\in\mathbb{N}: \forall n\geq N
            &\implies |s_n - t_n|<\delta \\
            &\implies |f(s_n)-f(t_n)|<\epsilon
        \end{align*}
        The definition for \textbf{not} uniformly continuous only makes sense if $x$ and $y$ are sequences, since if they are real numbers then the following implies that they must be equal:
        $$\forall\delta>0; \exists x, y\in I : |x-y|<\delta,$$
        and we reach a contradiction from the implication 
        $\exists!\epsilon>0: 0\geq\epsilon$. So for sequences $x_n$ and $y_n$ the previous statement implies:
        $$\lim_{n\rightarrow\infty}|x_n-y_n|=0$$
        But we also assumed that $\forall s_n, t_n \in I: \displaystyle\lim_{n\rightarrow\infty}|s_n - t_n|=0$.
        
        This justifies setting $x_n=s_n$ and $y_n=t_n$ with condition $\forall n\geq N$.
        $$\therefore\exists\epsilon>0; \forall\delta>0;\exists s_n, t_n\in I; |s_n - t_n|<\delta
        \hspace{0.07in}\text{and}\hspace{0.07in}|f(s_n)-f(t_n)|\geq\epsilon.$$
        But clearly this means that:
        $$\lim_{n\rightarrow\infty}|f(s_n) - f(t_n)|\neq0.$$
        Then by truth tables $f$ must be uniformly continuous.
    \end{proof}
    
\end{enumerate}