\documentclass{article}
\usepackage{geometry}
\usepackage{amsmath}
\usepackage{amsfonts}
\usepackage{amssymb}
\usepackage{amsthm}
\usepackage{parskip}
\usepackage{multicol}
\usepackage{xcolor}
\usepackage{fancyhdr}
\usepackage{physics}
\usepackage{graphicx} % Required for inserting images
\usepackage{hyperref}
\usepackage{enumitem}
\usepackage{mathtools}
\usepackage[utf8]{inputenc}
\usepackage[T1]{fontenc}
\usepackage{centernot}

% commands
\newcommand{\deq}{\vcentcolon=}
\newcommand{\idd}{\text{đ}}
\newcommand{\nimplies}{\centernot\implies}
\newcommand{\vc}[1]{\boldsymbol{#1}}

% margin settings
\geometry{
    a4paper,
    left=7mm,
    right=7mm,
    top=2cm,
    bottom=7mm
}

% testing
\usepackage{blindtext}

% proof environments
\newtheorem{definition}{Definition}[section]
\newtheorem{theorem}{Theorem}[section]
\newtheorem{corollary}{Corollary}[theorem]
\newtheorem{lemma}[theorem]{Lemma}
\newtheorem*{remark}{Remark} % unnumbered remarks

% header and footer
\pagestyle{fancy}
\fancyfoot{} % removes footer
\fancyhf{}
\renewcommand{\headrulewidth}{0.5pt}
\fancyhead[L]{Honours algebra}
\fancyhead[R]{\thepage}

\begin{document}

\begin{multicols*}{3}
\noindent

\subsubsection*{D: Functions}
A function $f:X\rightarrow Y$ is an assignment
of an element of $Y$ to \underline{each} element of $X$.
\begin{enumerate}
    \item $f$ is \textbf{injective} if:
    \begin{align*}
        &\forall x_1,x_2\in X;
        f(x_1)=f(x_2) \\ &\implies x_1=x_2.
    \end{align*}

    \item $f$ is \textbf{surjective} if:
    $$\forall y\in Y;\exists x\in X: y=f(x).$$

    \item $f$ is \textbf{bijective}
    if it is injective and surjective.
\end{enumerate}

\subsubsection*{D: Groups}
A group $G$ is a set defined with:
\begin{enumerate}
    \item Composition operator ($\cdot$) \\
    such that $x\cdot y=xy$.
    
    \item $\forall x,y,z\in G$; $(xy)z=x(yz)$
    
    \item $\exists e\in G: ex=xe=x$ \\
    for $\forall x\in G$.
    
    \item $\exists x^{-1}\in G: xx^{-1}=x^{-1}x=e$ \\
    for $\forall x\in G$.
\end{enumerate}
$G$ is \textbf{Abelian} if 
$\forall x,y\in G; xy=yx$.

\subsubsection*{D1.2.1(i): Fields}
A field $F$ is a set defined with:
\begin{enumerate}
    \item Addition function $(+)$:
    \begin{align*}
        (+):F\times F\rightarrow F;
        (\lambda,\mu)\mapsto\lambda+\mu
    \end{align*}

    \item Multiplication function $(\cdot)$:
    \begin{align*}
        (\cdot):F\times F\rightarrow F;
        (\lambda,\mu)\mapsto\lambda\cdot\mu
    \end{align*}

    \item $\exists 0_F,1_F\in F$
    where $0_F\neq1_F$ such that 
    $(F,+)$ and $(F\backslash\{0_F\},\cdot)$
    form Abelian groups.

    \item $\exists (-\lambda)\in F:
    \lambda+(-\lambda)=0_F$

    \item $\exists (\lambda^{-1})\in F:
    \lambda\cdot(\lambda^{-1})=1_F$

    \item $\lambda(\mu+\nu)
    =\lambda\mu+\lambda\nu\in F$
\end{enumerate}

\subsubsection*{D1.2.1(ii): Vector spaces}
A vector space $V$ over a field $F$ is an Abelian group $V\deq(V,+)$
with mapping:
$$F\times V\rightarrow V:
(\lambda,\vc{v}\mapsto
\lambda\vc{v})$$
where for $\forall\lambda, \mu\in F$ and
$\forall\vc{v},\vc{w}\in V$:
\begin{enumerate}
    \item
    $\lambda(\vc{v}+\vc{w})
    =(\lambda\vc{v})+(\mu\vc{w})$

    \item $(\lambda+\mu)\vc{v}
    =(\lambda\vc{v})+(\mu\vc{w})$

    \item $\lambda(\mu\vc{v})=(\lambda\mu)\vc{v}$
    
    \item $1_F\vc{v}=\vc{v}$
\end{enumerate}
and is a $F$-vector space.

\subsubsection*{Remark}
Let $V$ be a $F$-vector space where $\vc{v}\in V$.
\begin{enumerate}
    \item $0\vc{v}=0$
    \item $(-1)\vc{v}=-\vc{v}$
    \item $\lambda\vc{0}=\vc{0}$
    for $\forall\lambda\in F$.
\end{enumerate}

\subsubsection*{D: Cartesian products}
The Cartesian product of sets $X_1,\dots,X_n$
is defined as:
$$X_1\times\cdots\times X_n
\deq\{(x_1,\dots,x_n):x_i\in X_i\}$$
where $1\leq i\leq n$.

The \underline{projection} of a Cartesian product is:
\begin{align*}
    \text{pr}_i &:X_1\times\cdots\times X_n
    \rightarrow X_i; \\
    &(x_1,\dots,x_n)\mapsto x_i
\end{align*}

\subsubsection*{D1.4.1: Vector subspaces}
A vector subspace $U$ of $F$-vector space $V$
has the follwing properties:
\begin{enumerate}
    \item $U\subset V$ and $\vc{0}\in U$.
    \item Let $\vc{u},\vc{v}\in U$
    and $\lambda\in F$. \\ Then
    $\vc{u}+\vc{v}\in U$ and $\lambda\vc{u}\in U$.
\end{enumerate}
and is also a vector space.

\subsubsection*{P1.4.5}
Let $T\subset V$ where $V$ is a $F$-vector space.
Then for all vector subspaces containing $T$,
there exists a \underline{smallest} vector subspace:
$$\text{span}(T)=\langle T\rangle_F\subset V$$
known as the vector subspace generated by $T$,
or the span of $T$.

\subsubsection*{D1.4.7: Generating set}
Let $T\subset V$ where $V$ is a $F$-vector space.
$T$ is a generating set of $V$ if:
$$\text{span}(T)=V$$
and is the linear combination of vectors in $T$
over field $F$.

\subsubsection*{D1.4.9: Power sets}
The power set of set $X$ is:
$$\mathcal{P}(X)\deq\{U:U\subseteq X\}.$$
Let $\mathcal{U}\subseteq\mathcal{P}(X)$. Then:
$$\bigcup_{U\in\mathcal{U}}U
\deq\{x\in X:(\exists U\in\mathcal{U}:x\in U)\}$$
$$\bigcap_{U\in\mathcal{U}}U
\deq\{x\in X:\forall U\in\mathcal{U};x\in U\}.$$

\subsubsection*{D1.5.1: Linear independence}
Let $V$ be a $F$-vector space and $L\subseteq V$. \\
$L$ is linearly independent if:
\begin{align*}
    &\alpha_1\vc{v}_1+\dots+\alpha_r\vc{v}_r=\vc{0} \\
    &\implies \alpha_1=\dots=\alpha_r=0
\end{align*}
where $\vc{v}_i\in L$.

\subsubsection*{D1.5.8: Basis}
A basis of a vector space $V$ is a linearly independent
generating set in $V$.

\subsubsection*{T1.5.11}
Let $V$ be a $F$-vector space.

Then $\{\vc{v}_1,\dots,\vc{v}_r\}$ is a basis of $V$
\textbf{if{}f}:
$$\Phi:F^r\rightarrow V;$$
$$(\alpha_1,\dots,\alpha_r)\mapsto
\alpha_1\vc{v}_1+\dots+\alpha_r\vc{v}_r$$
is a bijection.

\subsubsection*{T1.5.12}
Let $V$ be a vector space and $E\subseteq V$.
Then the following statements are equivalent:
\begin{enumerate}
    \item $E$ is a basis of $V$.
    
    \item $E$ is minimal among all generating sets,
    or that $E\hspace{0.02in}\backslash\hspace{0.02in}
    \{\vc{v}\}$ is not a basis for $\forall\vc{v}\in V$.

    \item $E$ is maximal among all linearly independent subsets.
    i.e. $E\cup\{\vc{v}\}$ is not linearly independent.
\end{enumerate}

\subsubsection*{C1.5.13}
Every finitely generated vector space has a finite basis.

\subsubsection*{T1.5.14}

\subsubsection*{D1.5.15}

\subsubsection*{T1.5.16}

\newcolumn

\subsubsection*{T1.6.1}

\subsubsection*{T1.6.2}

\subsubsection*{L1.6.3}

\subsubsection*{L1.6.4}

\subsubsection*{D1.6.5}

\subsubsection*{C1.6.7}

\subsubsection*{C1.6.8}

\subsubsection*{T1.6.10}

\end{multicols*}

\end{document}